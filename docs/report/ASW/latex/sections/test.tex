\section{6~--- Test}\label{test}

Il frontend prodotto è stato testato principalmente su Chrome sfruttando
i Chrome DevTools per testare la responsività del layout su diversi
dispositivi ma anche direttamente da mobile una volta messo online il
sito. Per quanto riguarda il backend sono stati effettuati sia unit test
per i singoli componenti sia test e2e principalmente da Swagger e in
integrazione con il frontend.

\subsection{6.1~--- Accessibilità (a11y)}\label{accessibilituxe0-a11y}

Per garantire l'accessibilità dell'interfaccia sono stati adottati due
approcci complementari: \textbf{analisi statica} durante lo sviluppo e
\textbf{test automatizzati} sulle pagine web a runtime.

\subsubsection{Analisi Statica con
ESLint}\label{analisi-statica-con-eslint}

Il progetto integra \textbf{eslint-plugin-vuejs-\allowbreak%
accessibility} nella configurazione ESLint, identificando potenziali violazioni delle linee guida WCAG direttamente durante lo sviluppo:

\begin{Shaded}
\begin{Highlighting}[]
\ImportTok{import}\NormalTok{ vuejsAccessibility }\ImportTok{from} \StringTok{\textquotesingle{}eslint{-}plugin{-}vuejs{-}accessibility\textquotesingle{}}

\ImportTok{export} \ImportTok{default} \FunctionTok{defineConfig}\NormalTok{([}
  \CommentTok{// ... altre configurazioni}
  \OperatorTok{...}\NormalTok{vuejsAccessibility}\OperatorTok{.}\AttributeTok{configs}\NormalTok{[}\StringTok{\textquotesingle{}flat/recommended\textquotesingle{}}\NormalTok{]}\OperatorTok{,}
\NormalTok{])}
\end{Highlighting}
\end{Shaded}

Il plugin verifica automaticamente la presenza di attributi \texttt{alt}
nelle immagini, la corretta associazione di label ai form e l'uso
appropriato di ruoli \texttt{ARIA} segnalando le violazioni come warning
o errori durante il linting.

\subsubsection{Test Automatizzati con Puppeteer e
Lighthouse}\label{test-automatizzati-con-puppeteer-e-lighthouse}

Oltre all'analisi statica, il sistema implementa test di accessibilità
dinamici tramite \textbf{Puppeteer} e \textbf{Lighthouse}. Lo script
\texttt{a11y-check-multi.js} esegue un audit automatizzato sulle pagine
specificate in fase di configurazione:

\begin{Shaded}
\begin{Highlighting}[]
\KeywordTok{const}\NormalTok{ config }\OperatorTok{=}\NormalTok{ \{}
\NormalTok{  baseUrl}\OperatorTok{:} \BuiltInTok{process}\OperatorTok{.}\AttributeTok{env}\OperatorTok{.}\AttributeTok{BASE\_URL} \OperatorTok{||} \StringTok{\textquotesingle{}http://localhost:5173/it\textquotesingle{}}\OperatorTok{,}
\NormalTok{  pages}\OperatorTok{:}\NormalTok{ [}
\NormalTok{    \{ name}\OperatorTok{:} \StringTok{\textquotesingle{}Home\textquotesingle{}}\OperatorTok{,}\NormalTok{ path}\OperatorTok{:} \StringTok{\textquotesingle{}/\textquotesingle{}}\NormalTok{ \}}\OperatorTok{,}
\NormalTok{    \{ name}\OperatorTok{:} \StringTok{\textquotesingle{}Explore\textquotesingle{}}\OperatorTok{,}\NormalTok{ path}\OperatorTok{:} \StringTok{\textquotesingle{}/explore\textquotesingle{}}\NormalTok{ \}}\OperatorTok{,}
\NormalTok{    \{ name}\OperatorTok{:} \StringTok{\textquotesingle{}Login\textquotesingle{}}\OperatorTok{,}\NormalTok{ path}\OperatorTok{:} \StringTok{\textquotesingle{}/login\textquotesingle{}}\NormalTok{ \}}\OperatorTok{,}
\NormalTok{    \{ name}\OperatorTok{:} \StringTok{\textquotesingle{}Register\textquotesingle{}}\OperatorTok{,}\NormalTok{ path}\OperatorTok{:} \StringTok{\textquotesingle{}/register\textquotesingle{}}\NormalTok{ \}}\OperatorTok{,}
\NormalTok{    \{ name}\OperatorTok{:} \StringTok{\textquotesingle{}Create Event\textquotesingle{}}\OperatorTok{,}\NormalTok{ path}\OperatorTok{:} \StringTok{\textquotesingle{}/create{-}event\textquotesingle{}}\NormalTok{ \}}\OperatorTok{,}
\NormalTok{    \{ name}\OperatorTok{:} \StringTok{\textquotesingle{}Event Details\textquotesingle{}}\OperatorTok{,}\NormalTok{ path}\OperatorTok{:} \StringTok{\textquotesingle{}/events/547a3b27{-}344a{-}4318{-}b17e{-}edf7cd14aee3\textquotesingle{}}\NormalTok{ \}}\OperatorTok{,}
\NormalTok{    \{}
\NormalTok{      name}\OperatorTok{:} \StringTok{\textquotesingle{}Organization Profile [Published Events Tab]\textquotesingle{}}\OperatorTok{,}
\NormalTok{      path}\OperatorTok{:} \StringTok{\textquotesingle{}/users/7dee946f{-}3ab9{-}41b5{-}92e9{-}ea6264d9dd35\#publishedEvents\textquotesingle{}}\OperatorTok{,}
\NormalTok{    \}}\OperatorTok{,}
\NormalTok{  ]}\OperatorTok{,}
\NormalTok{  minScore}\OperatorTok{:} \PreprocessorTok{parseInt}\NormalTok{(}\BuiltInTok{process}\OperatorTok{.}\AttributeTok{env}\OperatorTok{.}\AttributeTok{MIN\_A11Y\_SCORE} \OperatorTok{||} \StringTok{\textquotesingle{}80\textquotesingle{}}\NormalTok{)}\OperatorTok{,}
\NormalTok{  themeMode}\OperatorTok{:} \BuiltInTok{process}\OperatorTok{.}\AttributeTok{env}\OperatorTok{.}\AttributeTok{THEME\_MODE} \OperatorTok{||} \StringTok{\textquotesingle{}both\textquotesingle{}}\OperatorTok{,} \CommentTok{// \textquotesingle{}light\textquotesingle{}, \textquotesingle{}dark\textquotesingle{}, \textquotesingle{}both\textquotesingle{}}
\NormalTok{\}}
\end{Highlighting}
\end{Shaded}

Lo script:

\begin{enumerate}
\def\labelenumi{\arabic{enumi}.}
\tightlist{}
\item
  \textbf{Lancia Chrome in modalità
  headless}~tramite~\texttt{chrome-launcher}
\item
  \textbf{Connette Puppeteer}~per controllare il browser e impostare il
  tema
\item
  \textbf{Esegue Lighthouse}~sulla categoria accessibilità per ogni
  pagina
\item
  \textbf{Testa entrambi i temi}~(light e dark mode) per garantire
  l'accessibilità in tutte le condizioni
\item
  \textbf{Genera report dettagliati}~in formato HTML e un summary
  testuale
\end{enumerate}

\begin{Shaded}
\begin{Highlighting}[]
\KeywordTok{async} \KeywordTok{function} \FunctionTok{runLighthouseOnPage}\NormalTok{(url}\OperatorTok{,}\NormalTok{ port}\OperatorTok{,}\NormalTok{ themeName) \{}
  \KeywordTok{const}\NormalTok{ options }\OperatorTok{=}\NormalTok{ \{}
\NormalTok{    logLevel}\OperatorTok{:} \StringTok{\textquotesingle{}error\textquotesingle{}}\OperatorTok{,}
\NormalTok{    output}\OperatorTok{:}\NormalTok{ [}\StringTok{\textquotesingle{}json\textquotesingle{}}\OperatorTok{,} \StringTok{\textquotesingle{}html\textquotesingle{}}\NormalTok{]}\OperatorTok{,}
\NormalTok{    onlyCategories}\OperatorTok{:}\NormalTok{ [}\StringTok{\textquotesingle{}accessibility\textquotesingle{}}\NormalTok{]}\OperatorTok{,}
\NormalTok{    port}\OperatorTok{:}\NormalTok{ port}\OperatorTok{,}
\NormalTok{  \}}
  \KeywordTok{const}\NormalTok{ runnerResult }\OperatorTok{=} \ControlFlowTok{await} \FunctionTok{lighthouse}\NormalTok{(url}\OperatorTok{,}\NormalTok{ options)}
  \ControlFlowTok{return}\NormalTok{ \{}
\NormalTok{    lhr}\OperatorTok{:}\NormalTok{ runnerResult}\OperatorTok{.}\AttributeTok{lhr}\OperatorTok{,}
\NormalTok{    reportHtml}\OperatorTok{:}\NormalTok{ runnerResult}\OperatorTok{.}\AttributeTok{report}\NormalTok{[}\DecValTok{1}\NormalTok{]}\OperatorTok{,}
\NormalTok{    theme}\OperatorTok{:}\NormalTok{ themeName}\OperatorTok{,}
\NormalTok{  \}}
\NormalTok{\}}
\end{Highlighting}
\end{Shaded}

Il report generato include:

\begin{itemize}
\tightlist{}
\item
  \textbf{Score di accessibilità}~per ogni pagina (scala 0--100)
\item
  \textbf{Violazioni critiche}~con conteggio degli elementi coinvolti
\item
  \textbf{Top 5 problemi più frequenti}~nell'intera applicazione
\item
  \textbf{Status pass/fail}~basato sulla soglia minima configurata
  (default 80/100)
\end{itemize}

Per lanciare i test è stato necessario predisporre una modalità ad hoc,
eseguibile con \texttt{npm\ run\ prod:a11y} che prevede login automatico
come organizzazione e disabilita le guardie lato frontend così da
permettere la corretta visualizzazione di tutte le pagine.

Lo script può poi essere eseguito tramite \texttt{npm\ run\ a11y:multi}
e restituisce un exit code non-zero in caso di pagine sotto la soglia
minima, permettendo l'integrazione in pipeline CI/CD\@.

Al momento, dato che il processo di seed iniziale genera ogni volta id
casuali, è necessario ricontrollare i link nella configuazione ad ogni
deploy con seeding.

\subsection{6.2~--- Euristiche di Nielsen}\label{euristiche-di-nielsen}

Per offrire una migliore usabilità e user experience, il sistema è stato
sottoposto alla valutazione delle euristiche di Nielsen. Di seguito le
considerazioni emerse:

\begin{enumerate}
\def\labelenumi{\arabic{enumi}.}
\tightlist{}
\item
  \textbf{Visibilità dello stato del sistema}: per segnalare attività in
  corso viene mostrata un'icona animata di caricamento. Lo stato online
  degli utenti è visibile in tempo reale nella chat tramite un apposito
  label, e le notifiche push informano l'utente di nuovi messaggi, like
  ricevuti o eventi pubblicati dalle organizzazioni seguite.
\item
  \textbf{Corrispondenza tra sistema e mondo reale}: la terminologia e
  le icone utilizzate sono in linea con le convenzioni dei social
  network (cuore per i like, fumetto per la chat, campanella per le
  notifiche). I concetti di ``evento'', ``biglietto'' e
  ``organizzazione'' rispecchiano il dominio reale.
\item
  \textbf{Controllo e libertà per l'utente}: l'applicazione non presenta
  percorsi obbligati. L'utente può navigare liberamente tra le sezioni,
  tornare indietro in qualsiasi momento e le operazioni critiche (come
  l'eliminazione di un evento) richiedono conferma esplicita.
\item
  \textbf{Consistenza e standard}: l'applicazione mantiene un linguaggio
  visivo uniforme grazie al framework Quasar. I pulsanti primari,
  secondari e di pericolo hanno stili distintivi e coerenti in tutto il
  sistema. La palette colori è stata definita in fase di design e
  applicata uniformemente, inclusa la modalità dark.
\item
  \textbf{Prevenzione dall'errore}: i form implementano validazione in
  tempo reale (formato di email, password e campi obbligatori in
  generale) con feedback immediato. Le navigation guards impediscono
  l'accesso a pagine non autorizzate, reindirizzando automaticamente
  l'utente (es. alla pagina di login se necessaria autenticazione).
\item
  \textbf{Riconoscimento anziché ricordo}: i layout sono consistenti tra
  le pagine. La barra di navigazione e i tab mantengono la stessa
  posizione, le card degli eventi hanno struttura uniforme e le icone
  sono autoesplicative senza necessità di tooltip.
\item
  \textbf{Flessibilità ed efficienza d'uso}: per gli utenti esperti sono
  disponibili scorciatoie da tastiera globali: \texttt{Ctrl/Cmd\ +\ D}
  per il toggle della dark mode, \texttt{Ctrl/Cmd\ +\ H} per tornare
  alla home, \texttt{Ctrl/Cmd\ +\ P} per aprire il profilo e
  \texttt{Ctrl/Cmd\ +\ E} per accedere alla creazione eventi.
\item
  \textbf{Design e estetica minimalista}: il design segue il principio
  KISS con approccio mobile-first. Ogni pagina presenta poche azioni ben
  distinte: la home mostra gli eventi in evidenza, la pagina explore
  permette la ricerca, il profilo gestisce le informazioni personali.
\item
  \textbf{Aiuto all'utente}: i messaggi di errore sono contestuali e
  descrittivi. Le notifiche toast informano l'utente dell'esito delle
  operazioni e i campi dei form mostrano label di errore specifiche
  (e.g.~``Email non valida'', ``Password troppo corta'').
\item
  \textbf{Documentazione}: vista la semplicità d'uso derivante dalle
  scelte di design, non è stata necessaria documentazione esterna. Le
  scorciatoie da tastiera disponibili (\texttt{Ctrl/Cmd\ +\ D/H/P/E})
  sono documentate internamente nel
  composable~\texttt{useKeyboardShortcuts}, pronte per essere esposte in
  una futura sezione ``Keyboard Shortcuts'' nelle impostazioni.
\end{enumerate}

\subsection{6.3~--- Test di Usabilità}\label{test-di-usabilituxe0}

Durante lo sviluppo dell'applicazione, questa è stata fatta provare a
diversi utenti, per ricevere di volta in volta dei feedback utili a
migliorare lo sviluppo e la resa finale del prodotto.

Ai soggetti del test non è stata fornita nessuna linea guida, ci si è
limitati a richiedere una particolare azione da svolgere al fine di
osservare come gli utenti avrebbero cercato di eseguire i compiti
richiesti vedendo per la prima volta il sistema.

\textbf{Test per Organizzazioni}: fra i compiti assegnati hanno figurato
la creazione di eventi mediante l'apposito editor, che si è rivelato
semplice ed intuitivo.

\textbf{Test per Membri}: fra i compiti figurava la ricerca degli
eventi, l'acquisto dei biglietti, la modifica del profilo e di
effettuare una recensione per un evento.

In generale gli utenti hanno trovato la disposizione dei vari elementi
sostanzialmente adeguata. Le osservazioni sollevate erano perlopiù su
aspetti stilistici e sono state incorporate nelle successive iterazioni
di sviluppo.
