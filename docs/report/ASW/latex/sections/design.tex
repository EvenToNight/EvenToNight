\section{3~--- Design}\label{design}

\subsection{3.1~--- Metodologia
progettuale}\label{metodologia-progettuale}

Lo sviluppo della piattaforma è stato condotto seguendo l'approccio
\emph{User Centered Design} (UCD), con l'obiettivo di progettare un
sistema conforme ai principi HCI ottimizzando l'esperienza utente.

Per garantire la centralità dell'utente durante il design e lo sviluppo,
non potendo coinvolgere utenti reali per l'intero ciclo progettuale,
sono state adottate tecniche di virtualizzazione degli utenti promosse
da UCD, quali la metodologia \textbf{\emph{Personas}} combinata con gli
\textbf{scenari d'uso}.

\subsubsection{Analisi dei target user}\label{analisi-dei-target-user}

Dopo aver delineato i requisiti di business della piattaforma, il team
di sviluppo ha individuato il target di riferimento. La piattaforma è
progettata per tre tipologie di utenti: utenti non registrati, utenti
registrati e utenti registrati come organizzazione.

Per rappresentare le caratteristiche e i bisogni di ciascun gruppo di
utenti del sistema, sono state create delle \emph{Personas}. Per ogni
\emph{Personas} è stato inoltre simulato uno scenario d'uso della
piattaforma, in modo da evidenziare le diverse modalità di interazione
con il sistema e come questo possa rispondere efficacemente alle
esigenze degli utenti.

\emph{Personas}: Carlo

Carlo è un adulto appassionato di eventi culturali e spettacoli, ma
fatica a scoprire nuove attività tramite i canali di informazione
tradizionali. Non ama condividere i propri dati sul web e vuole trovare
rapidamente eventi interessanti a cui partecipare, da solo o con amici e
familiari.

Carlo ha bisogno di uno strumento semplice e immediato che gli permetta
di esplorare gli eventi in programma che lo interessano, capire
rapidamente orari, luoghi e dettagli principali, senza dover registrarsi
o inserire informazioni personali.

Scenario d'uso:

Carlo visita il sito della piattaforma senza registrarsi, scorre gli
eventi in programma e consulta foto, descrizioni e informazioni pratiche
come orario e luogo.

Grazie all'interfaccia chiara e semplice, Carlo può farsi un'idea
immediata di quali eventi potrebbero interessargli e pianificare
eventuali uscite. Pur senza account, ottiene tutte le informazioni
necessarie e apprezza poter scoprire eventi diversi rispetto a quelli
tradizionali, senza condividere dati personali.

\emph{Personas}: Francesca

Francesca ha 22 anni e si è appena trasferita in una città universitaria
vivace. Frequenta il primo anno della magistrale in Comunicazione
Digitale e Marketing, divide il suo tempo principalmente tra lezioni e
studio, e sente il desiderio di vivere esperienze che la aiutino a
integrarsi nella nuova città.

Usa costantemente lo smartphone per restare aggiornata su eventi e
tendenze locali. È attratta da contenuti visivi e immediati, come foto
con una breve descrizione, che le permettono di percepire rapidamente
l'atmosfera di un evento. Apre diverse app e social network alla ricerca
di eventi, scorrendo post e pagine locali. Questo processo la stanca: le
informazioni sono spesso sparse e incomplete e non sempre riesce a
trovare esperienze che le interessano.

Francesca cerca uno strumento che le permetta di scoprire eventi nella
nuova città in cui vive, in modo intuitivo e veloce.

Scenario d'uso:

Alla sera, terminato di studiare, Francesca accede alla piattaforma dal
suo smartphone ed esplora i prossimi eventi in programma filtrando
quelli vicino a lei.

Quando il post di un evento cattura la sua attenzione, controlla la
pagina dell'organizzazione per vedere gli eventi passati e leggere le
recensioni, così da capire il tipo di esperienze che l'organizzazione
propone e valutare se l'evento possa essere affidabile e di suo
gradimento. Essendo in una città nuova, queste informazioni la aiutano a
scegliere con maggiore sicurezza.

Grazie alla piattaforma, Francesca riesce a trovare rapidamente eventi
vicino a lei compatibili con i suoi interessi, ottenendo tutte le
informazioni necessarie in un unico luogo e acquistando i biglietti in
modo semplice e immediato come è abituata a fare in altre piattaforme.

\emph{Personas}: Emma Lopez

Emma ha 27 anni e vive a Madrid. Sta organizzando un weekend in Italia
con le sue amiche e vuole scoprire locali, eventi musicali autentici e
buoni ristoranti italiani. Pianifica con attenzione ogni dettaglio ed è
importante per lei trovare attività che soddisfino gli interessi di
tutte le sue amiche.

Per cercare eventi e locali, Emma usa principalmente il computer. Naviga
tra diversi siti, ma non ha un modo comodo e veloce per salvare e
confrontare gli eventi che le interessano. Spesso deve tradurre siti
italiani per capirne i dettagli. Si affida alle foto per farsi un'idea
dell'evento, ma non sa sempre come scegliere quali eventi siano davvero
interessanti. Vorrebbe prenotare qualcosa per assicurarsi un posto e
godersi la vacanza senza rischiare di perdere gli eventi più popolari.

Scenario d'uso:

Emma si registra sulla piattaforma che trova direttamente in lingua
spagnola, essendo in Spagna. Dal suo computer esplora gli eventi
filtrando in base alle sue preferenze e a quelle delle sue amiche, per
selezionare quelli più adatti al gruppo. Può salvare direttamente i pos
degli eventi che le interessano e vedere quanti like hanno ricevuto,
aiutandola a capire quali sono più popolari. Una volta convinta,
acquista i biglietti per lei e per le sue amiche, assicurandosi i posti.

Emma è contenta di poter consultare tutti gli eventi in programma in un
unico sito già tradotto nella propria lingua e sfruttare le funzionalità
della piattaforma per salvare, gestire, filtrare e prenotare i suoi
eventi.

\emph{Personas}: Simone

Simone ha 34 anni e gestisce un locale poco conosciuto in periferia.
Crede nelle potenzialità del suo locale: l'ambiente è bello e
accogliente, ma fatica a farlo conoscere. Il sito web del locale riceve
poche visite e non permette di capire quanti utenti siano interessati
agli eventi. Nel locale Simone organizza principalmente cene, ma gli
piacerebbe collaborare con organizzatori di spettacoli dal vivo o show
da ospitare nel suo locale per aumentare la visibilità e attrarre più
clienti.

Simone ha bisogno di uno strumento che gli permetta di promuovere il suo
locale, monitorare facilmente l'interesse degli utenti e collaborare con
altri organizzatori in modo semplice ed efficace.

Scenario d'uso:

Simone si registra come organizzazione sulla piattaforma e crea il
profilo del suo locale, rendendolo visibile a tutti gli utenti. Ogni
volta che riceve un nuovo follower o che un suo evento ottiene un like,
Simone riceve una notifica che gli fornisce un feedback immediato
sull'interesse degli utenti. La piattaforma gli consente inoltre di
collaborare con altre organizzazioni, definendo un calendario ricco di
eventi per i prossimi mesi presso il suo locale.

Grazie alla piattaforma, Simone vede finalmente il suo locale apprezzato
e frequentato.

Dall'analisi dei target user, attraverso le \emph{Personas} e i relativi
scenari d'uso, sono emersi i principali task che gli utenti desiderano
svolgere sulla piattaforma. Queste informazioni hanno guidato la
progettazione dell'interfaccia grafica.

\subsection{3.2~--- Progettazione dell'Interfaccia
Grafica}\label{progettazione-dellinterfaccia-grafica}

In questa sezione viene descritta la progettazione dell'interfaccia
grafica dell'applicazione.

L'obiettivo principale è stato quello di definire un design chiaro,
intuitivo e coerente, in grado di guidare l'utente attraverso i
principali flussi funzionali della piattaforma.

La sezione comprende:

\begin{itemize}
\tightlist{}
\item
  la presentazione dei \textbf{mockup} delle schermate principali;
\item
  una panoramica degli \textbf{storyboard}, illustrando i principali
  flussi di interazione dell'utente.
\end{itemize}

\subsubsection{Mockup}\label{mockup}

Per la fase iniziale di progettazione dell'interfaccia grafica sono
stati creati dei \textbf{mockup}, con l'obiettivo di definire una prima
proposta del possibile \textbf{look \& feel} dell'applicazione, prima di
passare all'implementazione vera e propria.

Seguendo un approccio \textbf{agile}, i mockup sono stati
successivamente sostituiti da \textbf{demo funzionanti incrementali},
permettendo di testare e validare progressivamente le funzionalità
dell'applicazione.

I mockup sono stati realizzati utilizzando
\href{https://www.figma.com/}{Figma}. In particolare, sono state
sviluppate le schermate delle \textbf{tre aree principali}
dell'applicazione:

\begin{itemize}
\tightlist{}
\item
  \textbf{Home}
\item
  \textbf{Esplora}
\item
  \textbf{Profilo}
\end{itemize}

\paragraph{Approccio Mobile-First}\label{approccio-mobile-first}

La progettazione è stata effettuata seguendo l'approccio
\textbf{mobile-first}, prendendo come modello di riferimento l'iPhone
16.

Questo approccio è stato scelto principalmente per due motivi:

\begin{enumerate}
\def\labelenumi{\arabic{enumi}.}
\tightlist{}
\item
  \textbf{Vincoli più restrittivi}: il formato mobile costringe a dare
  priorità ai contenuti più importanti e a semplificare la navigazione,
  promuovendo un'interfaccia chiara e intuitiva che rispetta il
  principio \textbf{KISS}.
\item
  \textbf{Target principale da smartphone}: la maggior parte degli
  utenti accederà all'app tramite telefono, quindi è stata data priorità
  all'ottimizzazione del design per questo genere di dispositivi.
\end{enumerate}

Nel design è stato inoltre seguito il principio \textbf{DRY,}
riutilizzando ove possibile gli stessi componenti per garantire un
aspetto coerente e familiare dell'applicazione.

Ad ogni modo il design è stato pensato e realizzato anche per essere
responsive ed avere successivamente una buona resa anche su schermi
desktop.

\paragraph{Home}\label{home}

Di seguito è riportato il mockup per la schermata home. Questa è la
schermata iniziale proposta all'utente, da cui potrà da subito cercare
degli eventi o semplicemente vedere gli eventi proposti in vetrina. Da
questa schermata l'utente potrà anche accedere o registrarsi alla
piattaforma.

\begin{figure}[H]
\centering
\includegraphics[width=0.3\textwidth]{/mockup/Mockup-iPhone_16-Home_1.jpg}
\hfill
\includegraphics[width=0.3\textwidth]{/mockup/Mockup-iPhone_16-Home_2.jpg}
\hfill
\includegraphics[width=0.3\textwidth]{/mockup/Mockup-iPhone_16-Home_3.jpg}
\end{figure}

Inizialmente era stata anche proposta una versione alternativa con un
diverso sistema di navigazione, che però è stata successivamente
scartata vista la scarsa integrazione con il design dell'applicazione,
in particolare in combinazione con la schermata \textbf{Esplora}.

\begin{figure}[H]
\centering
\includegraphics[width=0.3\textwidth]{/mockup/Mockup-iPhone_16-Home-Alternative.jpg}
\end{figure}

\paragraph{Esplora}\label{esplora}

Di seguito è riportato il design della sezione esplora. In questa
sezione è possibile andare a visualizzare tutti gli eventi presenti
sulla piattaforma e cercare anche tutti gli utenti per visualizzarne il
profilo, seguirli e contattare le organizzazioni.

\begin{figure}[H]
\centering
\includegraphics[width=0.3\textwidth]{/mockup/Mockup-iPhone_16-Explore.jpg}
\hfill
\includegraphics[width=0.3\textwidth]{/mockup/Mockup-iPhone_16-Explore_Events.jpg}
\hfill
\includegraphics[width=0.3\textwidth]{/mockup/Mockup-iPhone_16-Explore_Organizations.jpg}
\end{figure}

\paragraph{Profilo}\label{profilo}

Da questa schermata l'utente avrà accesso alle sue informazioni, potrà
modificare il suo profilo e le sue preferenze.

\begin{figure}[H]
\centering
\includegraphics[width=0.3\textwidth]{/mockup/Mockup-iPhone_16-User_Profile.jpg}
\end{figure}

\subsubsection{Storyboard}\label{storyboard}

Di seguito sono riportati alcuni esempi di interazione con
l'applicazione, che illustrano i principali flussi di utilizzo.

In particolare, vengono mostrati i seguenti casi d'uso:

\begin{itemize}
\tightlist{}
\item
  \textbf{Esplorare la piattaforma}
\item
  \textbf{Login e Registrazione}
\item
  \textbf{Creare un evento}
\item
  \textbf{Partecipare ad un evento}
\item
  \textbf{Recensire un evento}
\item
  \textbf{Contattare un'organizzazione}
\end{itemize}

Le storyboard sono presentate direttamente utilizzando la piattaforma
sviluppata, mostrando le principali interazioni dell'utente.

Nel progettare i flussi di navigazione si è sempre tenuto conto della
\textbf{regola dei tre click}, cercando di rendere le principali
funzionalità accessibili in pochi passaggi. Negli esempi riportati di
seguito, alcune dinamiche di navigazione secondarie o ripetitive sono
state omesse.

\subsubsection{Esplorare la piattaforma}\label{esplorare-la-piattaforma}

L'utente che apre la piattaforma può iniziare ad esplorarla, in
particolare può scoprire gli eventi proposti o cercarli direttamente
tramite la barra di ricerca. Nel caso in cui voglia visualizzare
maggiori risultati può andare in una pagina dedicata all'esplorazione
dei contenuti della piattaforma dove è possibile filtrare gli eventi e
cercare in maniera più comoda utenti e organizzazioni.

\begin{figure}[H]
\centering
\pandocbounded{\includegraphics[keepaspectratio,alt={Storyboard Esplora}]{/storyboard/Storyboard-Explore.png}}
\caption{Storyboard Esplora}
\end{figure}

\subsubsection{Login e Registrazione}\label{login-e-registrazione}

Dopo aver aperto l'applicazione, l'utente per accedere alle funzionalità
aggiuntive che la piattaforma offre può accedere o registrarsi.

\begin{figure}[H]
\centering
\pandocbounded{\includegraphics[keepaspectratio,alt={Storyboard Login}]{/storyboard/Storyboard-Login.png}}
\caption{Storyboard Login}
\end{figure}

\subsubsection{Creare un evento}\label{creare-un-evento}

Un'organizzazione che si è registrata sulla piattaforma ha la
possibilità di creare degli eventi, la creazione avviene attraverso un
form in cui inserire tutti i vari dati. Inoltre è possibile anche
temporaneamente creare una bozza dell'evento e continuare a modificarla
successivamente.

\begin{figure}[H]
\centering
\pandocbounded{\includegraphics[keepaspectratio,alt={Storyboard Crea Evento}]{/storyboard/Storyboard-Create-Event.png}}
\caption{Storyboard Crea Evento}
\end{figure}

\subsubsection{Partecipare ad un evento}\label{partecipare-ad-un-evento}

Un utente registrato sulla piattaforma ha la possibilità di acquistare i
biglietti per i diversi eventi e visualizzarli in seguito, i biglietti
conterranno un QR code che può essere usato dalle organizzazioni per
verificarli.

\begin{figure}[H]
\centering
\pandocbounded{\includegraphics[keepaspectratio,alt={Storyboard Compra Biglietto}]{/storyboard/Storyboard-Buy-Ticket.png}}
\caption{Storyboard Compra Biglietto}
\end{figure}

\subsubsection{Recensire un evento}\label{recensire-un-evento}

In seguito alla partecipazione ad un evento, un utente può decidere di
lasciare una sua recensione. Una volta lasciata non ne può lasciare
altre per lo stesso evento ma può modificarla o eliminarla.

\begin{figure}[H]
\centering
\pandocbounded{\includegraphics[keepaspectratio,alt={Storyboard Recensione}]{/storyboard/Storyboard-Reviews.png}}
\caption{Storyboard Recensione}
\end{figure}

\subsubsection{Contattare
un'organizzazione}\label{contattare-unorganizzazione}

Un utente registrato può avere la necessità di contattare
un'organizzazione per chiedere maggiori informazioni o per eventuali
problemi.

\begin{figure}[H]
\centering
\pandocbounded{\includegraphics[keepaspectratio,alt={Storyboard Chat}]{/storyboard/Storyboard-Chat.png}}
\caption{Storyboard Chat}
\end{figure}

\subsection{3.3~--- Dominio}\label{dominio}

Durante la fase iniziale di analisi sono state individuate le entità
significative del dominio.

Il sistema si basa su due tipologie di utenti, membri e organizzazioni,
che condividono una base comune ma possiedono funzionalità differenti.

I membri rappresentano gli utenti finali della piattaforma e
interagiscono con gli eventi pubblicati dalle organizzazioni. Possono
esplorare gli eventi applicando filtri quali data, città, prezzo,
esprimere interesse tramite like e partecipare agli eventi acquistandone
i relativi biglietti. Dopo aver partecipato all'evento, gli utenti
possono lasciare recensioni, contribuendo alla valutazione e alla
popolarità dell'organizzazione.

Inoltre, possono seguire altri utenti e comunicare direttamente con le
organizzazioni tramite messaggi.

Le organizzazioni, oltre alle funzionalità comuni ai membri, hanno la
possibilità di creare e gestire eventi, definendone le informazioni e
rendendoli disponibili sulla piattaforma. Possono inoltre ricevere
recensioni da utenti che hanno partecipato ai loro eventi.

Tutti gli utenti ricevono delle notifiche a seguito di eventi di dominio
che li riguardano.

Da questa analisi emergono le principali entità del dominio, tra cui
utenti, eventi, biglietti, notifiche e interazioni, che costituiscono la
base per il design delle API e dell'architettura del sistema.

\subsection{3.4~--- Design delle risorse}\label{design-delle-risorse}

A partire dall'analisi del dominio, sono state definite come principali
risorse gli utenti e gli eventi.

Oltre alle risorse principali, il sistema modella ulteriori risorse
rilevanti, come:

\begin{itemize}
\tightlist{}
\item
  \textbf{tickets}, associati agli eventi
\item
  \textbf{conversations}, per la comunicazione tra utenti
\item
  \textbf{interactions}, che aggregano like, review e follow
\item
  \textbf{notifications,} associate agli utenti
\end{itemize}

Questo approccio permette di mantenere un modello REST coerente e
facilmente estendibile.

\subsubsection{/users}\label{users}

La risorsa /users rappresenta gli utenti del sistema ed è modellata
secondo l'archetipo \textbf{collection}, mentre /users/\{userId\}
rappresenta una risorsa \textbf{document}.

Ad ogni utente sono associate anche ulteriori \textbf{sotto-risorse}
(altre \textbf{collection}), tra cui:

\begin{itemize}
\tightlist{}
\item
  /users/\{userId\}/likes
\item
  /users/\{userId\}/reviews
\item
  /users/\{userId\}/followers
\item
  /users/\{userId\}/following
\item
  /users/\{userId\}/events
\item
  /users/\{userId\}/conversations
\item
  /users/\{userId\}/notifications
\end{itemize}

Alcuni endpoint seguono l'archetipo del \textbf{controller} in quanto
escono dal classico paradigma RESTfull indicando delle azioni, ad
esempio:~--- /users/login~--- /users/register

\subsubsection{/events}\label{events}

La risorsa /events rappresenta l'insieme degli eventi disponibili nel
sistema (archetipo \textbf{collection}), mentre /events/\{eventId\}
rappresenta un singolo evento (\textbf{document}).

Ad ogni evento sono associate anche ulteriori \textbf{sotto-risorse}
(altre \textbf{collection}), tra cui:

\begin{itemize}
\tightlist{}
\item
  /events/\{eventId\}/participants
\item
  /events/\{eventId\}/likes
\item
  /events/\{eventId\}/reviews
\item
  /events/\{eventId\}/tickets
\end{itemize}

Su alcuni endpoint vengono utilizzati dei query params per filtrare i
dati delle collections ed effettuare una richiesta paginata tramite
limit e offset, ad esempio:

\begin{itemize}
\tightlist{}
\item
  /events/search?query=sunset\&limit=10\&offset=0\&orderBy=date
\end{itemize}

\subsection{3.5~--- Design delle API}\label{design-delle-api}

Le API sono state progettate seguendo i principi del REST API Design,
con particolare attenzione all'uso corretto dei metodi HTTP,
all'utilizzo di sostantivi negli url e al rispetto degli archetipi REST\@.

Alcuni esempi di API implementate seguendo i principi citati:

\begin{figure}[H]
\centering
\pandocbounded{\includegraphics[keepaspectratio,alt={Alcune API per la collection Users}]{/api/user-api.png}}
\caption{Alcune API per la collection Users}
\end{figure}

\begin{figure}[H]
\centering
\pandocbounded{\includegraphics[keepaspectratio,alt={Alcune API per la collection Events}]{/api/events-api.png}}
\caption{Alcune API per la collection Events}
\end{figure}

\begin{figure}[H]
\centering
\pandocbounded{\includegraphics[keepaspectratio,alt={Alcune API per la collection Ticket-Types}]{/api/ticket_types-api.png}}
\caption{Alcune API per la collection Ticket-Types}
\end{figure}

Ogni microservizio espone una documentazione \textbf{Swagger/OpenAPI},
che descrive endpoint, input e possibili risposte HTTP\@.

Swagger è stato utilizzato sia come strumento di documentazione che come
supporto al testing manuale delle API\@. Di seguito il link
\url{https://eventonight.github.io/EvenToNight/openAPI/}

\subsection{3.6~--- Architettura}\label{architettura}

L'architettura del sistema è basata su un insieme di microservizi
indipendenti, ciascuno responsabile di un sotto-dominio applicativo
specifico.

Il frontend rappresenta il punto di accesso per gli utenti e comunica
esclusivamente con i servizi backend tramite API REST e socket, senza
accesso diretto ai database.

Ogni servizio infatti possiede la propria logica di business, entità
comuni a più servizi (es. gli eventi) vengono rappresentati in maniera
diversa in ognuno di essi.

Ogni servizio è containerizzato tramite Docker, comunica con gli altri
servizi principalmente tramite messaggi asincroni ed espone un insieme
coerente di API REST\@.

\begin{figure}[H]
\centering
\pandocbounded{\includegraphics[keepaspectratio,alt={Panoramica dell' architectura}]{/architecture/architecture-overview.png}}
\caption{Panoramica dell' architectura}
\end{figure}

Le risorse individuate nella fase precedente sono state organizzate e
distribuite nei vari servizi, di seguito una breve descizione.

\begin{figure}[H]
\centering
\pandocbounded{\includegraphics[keepaspectratio,alt={Archiettura del servizio Users}]{/architecture/users-service.png}}
\caption{Archiettura del servizio Users}
\end{figure}

Il servizio \emph{users} è responsabile della gestione delle risorse
utente e di una parte delle relative sotto-risorse. Per ogni utente, il
sistema gestisce l'auenticazione e le informazioni riguardanti l'accoun
(e.g username, email, interessi) e il profilo (e.g.~nome, bio).

\begin{figure}[H]
\centering
\pandocbounded{\includegraphics[keepaspectratio,alt={Archiettura del servizio Events}]{/architecture/events-service.png}}
\caption{Archiettura del servizio Events}
\end{figure}

Il servizio \emph{events} è responsabile della gestione delle risorse
eventi e delle loro informazioni. Gestisce sia gli aspetti di creazione
degli eventi sia il recupero degli eventi filtrati in base a
caratteristiche specificate (e.g.popolari, interessi).

\begin{figure}[H]
\centering
\pandocbounded{\includegraphics[keepaspectratio,alt={Archiettura del servizio Interactions}]{/architecture/interactions-service.png}}
\caption{Archiettura del servizio Interactions}
\end{figure}

Il servizio \emph{interactions} è responsabile di alcune sotto-risorse
degli utenti e degli eventi. In particolare, gestisce tutte le
informazioni riguardanti le interazioni che un utente può avere con un
evento o con un altro utente. Tra le interazioni con gli eventi sono
stati implementati i likes, le reviews e le partecipazioni, mentre con
gli altri utenti è stato implementato un sistema di following.

\begin{figure}[H]
\centering
\pandocbounded{\includegraphics[keepaspectratio,alt={Archiettura del servizio Chat}]{/architecture/chat-service.png}}
\caption{Archiettura del servizio Chat}
\end{figure}

Il servizio \emph{chat} è responsabile di alcune sotto-risorse degli
utenti. In particolare, gestisce tutte le informazioni riguardanti le
conversazioni che l'utente può avere con altri utenti.

\begin{figure}[H]
\centering
\pandocbounded{\includegraphics[keepaspectratio,alt={Archiettura del servizio Payments}]{/architecture/payments-service.png}}
\caption{Archiettura del servizio Payments}
\end{figure}

Il servizio payments è responsabile di alcune risorse degli utenti e
degli eventi, in particolare della gestione dei biglietti che gli utenti
possono acquistare per partecipare agli eventi.

\begin{figure}[H]
\centering
\pandocbounded{\includegraphics[keepaspectratio,alt={Archiettura del servizio Notifications}]{/architecture/notifications-service.png}}
\caption{Archiettura del servizio Notifications}
\end{figure}

Il servizio \emph{notifications} è responsabile di alcune sotto-risorse
degli utenti, in particolare gestisce tutte le informazioni riguardanti
le notifiche che un utente riceve. Le notifiche possono essere di tipi
diversi, quali like ricevuto, review ricevuta, nuovo follower o nuovo
evento creato da un'organizzazione seguita. Si occupa anche di
notificare l'arrivo di un nuovo messaggio, gestito poi dal servizio
chat.

\subsection{3.7~--- Comunicazione: RabbitMQ e
Socket.IO}\label{comunicazione-rabbitmq-e-socket.io}

Per la comunicazione tra servizi abbiamo utilizzato un broker di
messaggi, in particolare RabbitMQ\@. Ogni servizio definisce alcuni
\emph{domain event} \emph{(e.g.~nuovo evento pubblicato, like aggiunto,
utente creato)} e li comunica sull'exchange \emph{eventonight}, al quale
sono connesse e in ascolto le code dei vari servizi.

Per funzionalità che richiedono aggiornamento in tempo reale (come le
notifiche e le chat) abbiamo utilizzato una comunicazione basata su
socket centralizzandola nel servizio notifications in modo da aprire una
sola connessione con ciascun client. Il servizio notifications tramite
RabbitMQ ascolta tutti i \emph{domain event} che vogliamo comunicare
real-time, li elabora e quando necessario li inoltra nei canali degli
utenti connessi interessati.
