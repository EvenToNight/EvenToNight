\section{5~--- Codice}\label{codice}

L'architettura del progetto \textbf{EvenToNight} si basa su una
\textbf{Single Page Application (SPA)} per il frontend e un layer di
\textbf{microservizi backend} esposti attraverso \textbf{Traefik} come
reverse-proxy/API gateway.

\subsection{5.1~--- Frontend}\label{frontend}

Il frontend è stato sviluppato con \textbf{Vue 3} utilizzando la
\textbf{Composition API} e \textbf{TypeScript}, sfruttando il framework
\textbf{Quasar} per i componenti UI\@.

L'applicazione fa uso delle principali funzionalità di Vue come
\texttt{provide/\allowbreak{} inject}, \texttt{props/\allowbreak{} emit}, \texttt{defineModel},
\texttt{defineExpose}, \texttt{watchers} e in alcune situazioni la
direttiva \texttt{:key} per innescare il refresh dei componenti.

\subsubsection{Struttura del Progetto}\label{struttura-del-progetto}

\begin{Shaded}
\begin{Highlighting}[]
\NormalTok{/src/}
\NormalTok{├── api/                 \# Layer di astrazione API con supporto mock}
\NormalTok{│   ├── adapters/        \# Adapter per allinare i dati ricevuti dalle API}
\NormalTok{│   ├── mock{-}services/   \# Implementazioni mock per sviluppo}
\NormalTok{│   ├── services/        \# Implementazioni API dei servizi}
\NormalTok{│   └── client.ts        \# Client HTTP con gestione JWT}
\NormalTok{├── components/          \# Componenti Vue riutilizzabili}
\NormalTok{├── composables/         \# Funzionalità riutilizzabili dai vari componenti}
\NormalTok{├── i18n/                \# File di lingua}
\NormalTok{├── layouts/             \# Layout condivisi}
\NormalTok{├── router/              \# Configurazione routing e guards}
\NormalTok{├── stores/              \# Pinia store}
\NormalTok{└── views/               \# Pagine dell\textquotesingle{}applicazione}
\end{Highlighting}
\end{Shaded}

\subsubsection{Gestione dello Stato con
Pinia}\label{gestione-dello-stato-con-pinia}

Lo store Pinia gestisce l'autenticazione dell'utente e i token JWT\@. Il
sistema implementa il refresh automatico dei token prima della scadenza:

\begin{Shaded}
\begin{Highlighting}[]
\KeywordTok{interface}\NormalTok{ Tokens \{}
\NormalTok{  accesToken}\OperatorTok{:}\NormalTok{ AccessToken}\OperatorTok{,}
\NormalTok{  refreshToken}\OperatorTok{:}\NormalTok{ RefreshToken}\OperatorTok{,}
\NormalTok{  refreshExpiresAt}\OperatorTok{:} \DataTypeTok{number}\OperatorTok{,}
\NormalTok{\}}

\ImportTok{export} \KeywordTok{const}\NormalTok{ useAuthStore }\OperatorTok{=} \FunctionTok{defineStore}\NormalTok{(}\StringTok{\textquotesingle{}auth\textquotesingle{}}\OperatorTok{,}\NormalTok{ () }\KeywordTok{=\textgreater{}}\NormalTok{ \{}
  \KeywordTok{const}\NormalTok{ user }\OperatorTok{=} \FunctionTok{ref}\OperatorTok{\textless{}}\NormalTok{User }\OperatorTok{|} \DataTypeTok{null}\OperatorTok{\textgreater{}}\NormalTok{(}\KeywordTok{null}\NormalTok{)}
  \KeywordTok{const}\NormalTok{ tokens }\OperatorTok{=}\NormalTok{ ref}\OperatorTok{\textless{}}\NormalTok{Tokens }\OperatorTok{|} \DataTypeTok{null}\OperatorTok{\textgreater{}}\NormalTok{ (}\KeywordTok{null}\NormalTok{)}

  \KeywordTok{const}\NormalTok{ isAuthenticated }\OperatorTok{=} \FunctionTok{computed}\NormalTok{(() }\KeywordTok{=\textgreater{}}\NormalTok{ \{}
    \ControlFlowTok{if}\NormalTok{ (}\OperatorTok{!}\NormalTok{tokens}\OperatorTok{.}\AttributeTok{value} \OperatorTok{||} \OperatorTok{!}\NormalTok{user}\OperatorTok{.}\AttributeTok{value}\NormalTok{) }\ControlFlowTok{return} \KeywordTok{false}
    \ControlFlowTok{return}\NormalTok{ tokens}\OperatorTok{.}\AttributeTok{refreshExpiresAt}\OperatorTok{.}\AttributeTok{value} \OperatorTok{\textgreater{}} \BuiltInTok{Date}\OperatorTok{.}\FunctionTok{now}\NormalTok{()}
\NormalTok{  \})}

  \CommentTok{// Set data to local storage to avoid data loss on page refresh}
  \KeywordTok{const}\NormalTok{ setAuthData }\OperatorTok{=} \KeywordTok{async}\NormalTok{ (authData}\OperatorTok{:}\NormalTok{ LoginResponse) }\KeywordTok{=\textgreater{}}\NormalTok{ \{}
    \FunctionTok{setTokens}\NormalTok{(authData)}
    \FunctionTok{setUser}\NormalTok{(authData}\OperatorTok{.}\AttributeTok{user}\NormalTok{)}
    \FunctionTok{setupAutoRefresh}\NormalTok{()}
    \ControlFlowTok{await}\NormalTok{ api}\OperatorTok{.}\AttributeTok{notifications}\OperatorTok{.}\FunctionTok{connect}\NormalTok{(authData}\OperatorTok{.}\AttributeTok{user}\OperatorTok{.}\AttributeTok{id}\OperatorTok{,}\NormalTok{ authData}\OperatorTok{.}\AttributeTok{accessToken}\NormalTok{) }\CommentTok{//Connect to Socket}
\NormalTok{  \}}

  \CommentTok{// Restores auth data from session storage (if any)}
  \KeywordTok{const}\NormalTok{ refreshCurrentSessionUserData }\OperatorTok{=}\NormalTok{ () }\KeywordTok{=\textgreater{}}\NormalTok{ \{\}}

  \CommentTok{// Auto{-}refresh 5 minutes before expiration}
  \KeywordTok{const}\NormalTok{ setupAutoRefresh }\OperatorTok{=}\NormalTok{ () }\KeywordTok{=\textgreater{}}\NormalTok{ \{}
    \KeywordTok{const}\NormalTok{ refreshTime }\OperatorTok{=}\NormalTok{ tokens}\OperatorTok{.}\AttributeTok{value}\OperatorTok{.}\AttributeTok{refreshExpiresAt} \OperatorTok{{-}} \BuiltInTok{Date}\OperatorTok{.}\FunctionTok{now}\NormalTok{() }\OperatorTok{{-}} \DecValTok{5} \OperatorTok{*} \DecValTok{60} \OperatorTok{*} \DecValTok{1000}
    \ControlFlowTok{if}\NormalTok{ (refreshTime }\OperatorTok{\textgreater{}} \DecValTok{0}\NormalTok{) \{}
      \PreprocessorTok{setTimeout}\NormalTok{(() }\KeywordTok{=\textgreater{}} \FunctionTok{refreshAccessToken}\NormalTok{()}\OperatorTok{,}\NormalTok{ refreshTime)}
\NormalTok{    \}}
\NormalTok{  \}}
\NormalTok{\})}
\end{Highlighting}
\end{Shaded}

\subsubsection{Layer API e Mocking
Strategy}\label{layer-api-e-mocking-strategy}

Un aspetto fondamentale dello sviluppo è stato il \textbf{layer di
astrazione API}, che ha permesso di prototipare il frontend
indipendentemente dalla disponibilità dei microservizi backend.

Il sistema utilizza una variabile d'ambiente per utilizzare API reali o
mock:

\begin{Shaded}
\begin{Highlighting}[]
\KeywordTok{const}\NormalTok{ useRealApi}\OperatorTok{:} \DataTypeTok{boolean} \OperatorTok{=}\NormalTok{ import}\OperatorTok{.}\AttributeTok{meta}\OperatorTok{.}\AttributeTok{env}\OperatorTok{.}\AttributeTok{VITE\_USE\_MOCK\_API} \OperatorTok{===} \StringTok{\textquotesingle{}false\textquotesingle{}}

\ImportTok{export} \KeywordTok{const}\NormalTok{ api }\OperatorTok{=}\NormalTok{ \{}
\NormalTok{  events}\OperatorTok{:}\NormalTok{ useRealApi }\OperatorTok{?} \FunctionTok{createEventsApi}\NormalTok{(}\FunctionTok{createEventsClient}\NormalTok{()) }\OperatorTok{:}\NormalTok{ mockEventsApi}\OperatorTok{,}
\NormalTok{  chat}\OperatorTok{:}\NormalTok{ useRealApi }\OperatorTok{?} \FunctionTok{createChatApi}\NormalTok{(}\FunctionTok{createChatClient}\NormalTok{()) }\OperatorTok{:}\NormalTok{ mockChatApi}\OperatorTok{,}
\NormalTok{  notifications}\OperatorTok{:}\NormalTok{ useRealApi}
    \OperatorTok{?} \FunctionTok{createNotificationsApi}\NormalTok{(}\FunctionTok{createNotificationsClient}\NormalTok{())}
    \OperatorTok{:}\NormalTok{ mockNotificationsApi}\OperatorTok{,}
  \CommentTok{// other services}
\NormalTok{\}}
\end{Highlighting}
\end{Shaded}

Il client HTTP gestisce centralmente l'injection del token JWT e il refresh automatico in caso di 401:

\begin{Shaded}
\begin{Highlighting}[]
\KeywordTok{const}\NormalTok{ token }\OperatorTok{=}\NormalTok{ tokenProvider}\OperatorTok{?.}\NormalTok{()}
\ControlFlowTok{if}\NormalTok{ (token) \{}
\NormalTok{  headers[}\StringTok{\textquotesingle{}Authorization\textquotesingle{}}\NormalTok{] }\OperatorTok{=} \VerbatimStringTok{\textasciigrave{}Bearer }\SpecialCharTok{$\{}\NormalTok{token}\SpecialCharTok{\}}\VerbatimStringTok{\textasciigrave{}}
\NormalTok{\}}

\CommentTok{// Refresh after receiving 401}
\ControlFlowTok{if}\NormalTok{ (response}\OperatorTok{.}\AttributeTok{status} \OperatorTok{===} \DecValTok{401} \OperatorTok{\&\&} \OperatorTok{!}\NormalTok{isRetry }\OperatorTok{\&\&}\NormalTok{ onTokenExpired) \{}
  \KeywordTok{const}\NormalTok{ refreshed }\OperatorTok{=} \ControlFlowTok{await} \FunctionTok{onTokenExpired}\NormalTok{()}
  \ControlFlowTok{if}\NormalTok{ (refreshed) \{}
    \ControlFlowTok{return} \KeywordTok{this}\OperatorTok{.}\FunctionTok{request}\NormalTok{(endpoint}\OperatorTok{,}\NormalTok{ options}\OperatorTok{,} \KeywordTok{true}\NormalTok{)}
\NormalTok{  \}}
\NormalTok{\}}
\end{Highlighting}
\end{Shaded}

Questo approccio, seguendo il principio \textbf{Dependency Inversion
(DIP)}, ha permesso di sviluppare componenti UI indipendenti
dall'implementazione concreta delle API e ha facilitato il testing.

\subsubsection{Router e Navigation
Guards}\label{router-e-navigation-guards}

Il sistema di routing utilizza due livelli: un livello root per gestire
redirect e rotte speciali, e un livello nested sotto \texttt{/:locale}
per tutte le rotte localizzate. Questa separazione è stata necessaria
perché alcune rotte (come quella codificata nel QR code nei biglietti)
non richiedono il prefisso della lingua.

\begin{Shaded}
\begin{Highlighting}[]

\KeywordTok{const}\NormalTok{ router }\OperatorTok{=} \FunctionTok{createRouter}\NormalTok{(\{}
\NormalTok{  history}\OperatorTok{:} \FunctionTok{createWebHistory}\NormalTok{(import}\OperatorTok{.}\AttributeTok{meta}\OperatorTok{.}\AttributeTok{env}\OperatorTok{.}\AttributeTok{BASE\_URL}\NormalTok{)}\OperatorTok{,}
\NormalTok{  routes}\OperatorTok{:}\NormalTok{ [}
\NormalTok{    \{}
\NormalTok{      path}\OperatorTok{:} \StringTok{\textquotesingle{}/\textquotesingle{}}\OperatorTok{,}
\NormalTok{      redirect}\OperatorTok{:}\NormalTok{ () }\KeywordTok{=\textgreater{}} \VerbatimStringTok{\textasciigrave{}/}\SpecialCharTok{$\{}\FunctionTok{getInitialLocale}\NormalTok{()}\SpecialCharTok{\}}\VerbatimStringTok{\textasciigrave{}}\OperatorTok{,}
\NormalTok{    \}}\OperatorTok{,}
\NormalTok{    \{}
\NormalTok{      path}\OperatorTok{:} \StringTok{\textquotesingle{}/verify/:ticketId\textquotesingle{}}\OperatorTok{,}
\NormalTok{      redirect}\OperatorTok{:}\NormalTok{ (to) }\KeywordTok{=\textgreater{}} \VerbatimStringTok{\textasciigrave{}/}\SpecialCharTok{$\{}\FunctionTok{getInitialLocale}\NormalTok{()}\SpecialCharTok{\}}\VerbatimStringTok{/verify/}\SpecialCharTok{$\{}\NormalTok{to}\OperatorTok{.}\AttributeTok{params}\OperatorTok{.}\AttributeTok{ticketId}\SpecialCharTok{\}}\VerbatimStringTok{\textasciigrave{}}\OperatorTok{,}
\NormalTok{    \}}\OperatorTok{,}
\NormalTok{    \{}
\NormalTok{      path}\OperatorTok{:} \StringTok{\textquotesingle{}/:locale\textquotesingle{}}\OperatorTok{,}
\NormalTok{      component}\OperatorTok{:}\NormalTok{ LocaleWrapper}\OperatorTok{,}
\NormalTok{      children}\OperatorTok{:}\NormalTok{ [}
\NormalTok{        \{ path}\OperatorTok{:} \StringTok{\textquotesingle{}\textquotesingle{}}\OperatorTok{,}\NormalTok{ name}\OperatorTok{:} \StringTok{\textquotesingle{}home\textquotesingle{}}\OperatorTok{,}\NormalTok{ component}\OperatorTok{:}\NormalTok{ Home \}}\OperatorTok{,}
\NormalTok{        \{ path}\OperatorTok{:} \StringTok{\textquotesingle{}login\textquotesingle{}}\OperatorTok{,}\NormalTok{ name}\OperatorTok{:} \StringTok{\textquotesingle{}login\textquotesingle{}}\OperatorTok{,}\NormalTok{ component}\OperatorTok{:}\NormalTok{ () }\KeywordTok{=\textgreater{}} \ImportTok{import}\NormalTok{(}\StringTok{\textquotesingle{}../views/AuthView.vue\textquotesingle{}}\NormalTok{)}\OperatorTok{,}\NormalTok{ beforeEnter}\OperatorTok{:}\NormalTok{ requireGuest \}}\OperatorTok{,}
\NormalTok{        \{ path}\OperatorTok{:} \StringTok{\textquotesingle{}events/:id\textquotesingle{}}\OperatorTok{,}\NormalTok{ name}\OperatorTok{:} \StringTok{\textquotesingle{}event{-}details\textquotesingle{}}\OperatorTok{,}\NormalTok{ component}\OperatorTok{:}\NormalTok{ () }\KeywordTok{=\textgreater{}} \ImportTok{import}\NormalTok{(}\StringTok{\textquotesingle{}../views/EventDetailsView.vue\textquotesingle{}}\NormalTok{)}\OperatorTok{,}\NormalTok{ beforeEnter}\OperatorTok{:}\NormalTok{ requireNotDraft \}}\OperatorTok{,}
\NormalTok{        \{ path}\OperatorTok{:} \StringTok{\textquotesingle{}create{-}event\textquotesingle{}}\OperatorTok{,}\NormalTok{ name}\OperatorTok{:} \StringTok{\textquotesingle{}create{-}event\textquotesingle{}}\OperatorTok{,}\NormalTok{ component}\OperatorTok{:}\NormalTok{ () }\KeywordTok{=\textgreater{}} \ImportTok{import}\NormalTok{(}\StringTok{\textquotesingle{}../views/CreateEventView.vue\textquotesingle{}}\NormalTok{)}\OperatorTok{,}\NormalTok{ beforeEnter}\OperatorTok{:} \FunctionTok{requireRole}\NormalTok{(}\StringTok{\textquotesingle{}organization\textquotesingle{}}\NormalTok{) \}}\OperatorTok{,}
        \CommentTok{// ... other routes}
\NormalTok{      ]}\OperatorTok{,}
\NormalTok{    \}}\OperatorTok{,}
\NormalTok{  ]}\OperatorTok{,}
\NormalTok{\})}

\CommentTok{// Global guard for i18n synchronization}
\NormalTok{router}\OperatorTok{.}\FunctionTok{beforeEach}\NormalTok{((to}\OperatorTok{,}\NormalTok{ \_from}\OperatorTok{,}\NormalTok{ next) }\KeywordTok{=\textgreater{}}\NormalTok{ \{}
  \KeywordTok{const}\NormalTok{ locale }\OperatorTok{=}\NormalTok{ to}\OperatorTok{.}\AttributeTok{params}\OperatorTok{.}\AttributeTok{locale} \ImportTok{as} \DataTypeTok{string}

  \CommentTok{// Redirect if locale is not supported}
  \ControlFlowTok{if}\NormalTok{ (locale }\OperatorTok{\&\&} \OperatorTok{!}\NormalTok{SUPPORTED\_LOCALES}\OperatorTok{.}\FunctionTok{includes}\NormalTok{(locale)) \{}
    \ControlFlowTok{return} \FunctionTok{next}\NormalTok{(}\VerbatimStringTok{\textasciigrave{}/}\SpecialCharTok{$\{}\NormalTok{DEFAULT\_LOCALE}\SpecialCharTok{\}$\{}\NormalTok{to}\OperatorTok{.}\AttributeTok{path}\OperatorTok{.}\FunctionTok{substring}\NormalTok{(locale}\OperatorTok{.}\AttributeTok{length} \OperatorTok{+} \DecValTok{1}\NormalTok{)}\SpecialCharTok{\}}\VerbatimStringTok{\textasciigrave{}}\NormalTok{)}
\NormalTok{  \}}

  \CommentTok{// Restore saved locale preference on navigation}
  \KeywordTok{const}\NormalTok{ savedLocale }\OperatorTok{=}\NormalTok{ localStorage}\OperatorTok{.}\FunctionTok{getItem}\NormalTok{(}\StringTok{\textquotesingle{}user{-}locale\textquotesingle{}}\NormalTok{)}
  \ControlFlowTok{if}\NormalTok{ (savedLocale }\OperatorTok{\&\&}\NormalTok{ locale }\OperatorTok{\&\&}\NormalTok{ locale }\OperatorTok{!==}\NormalTok{ savedLocale }\OperatorTok{\&\&}\NormalTok{ SUPPORTED\_LOCALES}\OperatorTok{.}\FunctionTok{includes}\NormalTok{(savedLocale)) \{}
    \ControlFlowTok{return} \FunctionTok{next}\NormalTok{(\{}
\NormalTok{      name}\OperatorTok{:}\NormalTok{ to}\OperatorTok{.}\AttributeTok{name} \ImportTok{as} \DataTypeTok{string}\OperatorTok{,}
\NormalTok{      params}\OperatorTok{:}\NormalTok{ \{ }\OperatorTok{...}\NormalTok{to}\OperatorTok{.}\AttributeTok{params}\OperatorTok{,}\NormalTok{ locale}\OperatorTok{:}\NormalTok{ savedLocale \}}\OperatorTok{,}
\NormalTok{      query}\OperatorTok{:}\NormalTok{ to}\OperatorTok{.}\AttributeTok{query}\OperatorTok{,}
\NormalTok{      replace}\OperatorTok{:} \KeywordTok{true}\OperatorTok{,}
\NormalTok{    \})}
\NormalTok{  \}}

  \CommentTok{// Sync i18n with URL locale}
  \ControlFlowTok{if}\NormalTok{ (locale }\OperatorTok{\&\&}\NormalTok{ i18n}\OperatorTok{.}\AttributeTok{global}\OperatorTok{.}\AttributeTok{locale}\OperatorTok{.}\AttributeTok{value} \OperatorTok{!==}\NormalTok{ locale) \{}
\NormalTok{    i18n}\OperatorTok{.}\AttributeTok{global}\OperatorTok{.}\AttributeTok{locale}\OperatorTok{.}\AttributeTok{value} \OperatorTok{=}\NormalTok{ locale }\ImportTok{as}\NormalTok{ Locale}
\NormalTok{    localStorage}\OperatorTok{.}\FunctionTok{setItem}\NormalTok{(}\StringTok{\textquotesingle{}user{-}locale\textquotesingle{}}\OperatorTok{,}\NormalTok{ locale)}
\NormalTok{  \}}

  \FunctionTok{next}\NormalTok{()}
\NormalTok{\})}
\end{Highlighting}
\end{Shaded}

Le \textbf{navigation guards} proteggono le rotte in base
all'autenticazione e ai ruoli:

\begin{Shaded}
\begin{Highlighting}[]
\ImportTok{export} \KeywordTok{const}\NormalTok{ requireAuth }\OperatorTok{=}\NormalTok{ (to}\OperatorTok{,}\NormalTok{ \_from}\OperatorTok{,}\NormalTok{ next) }\KeywordTok{=\textgreater{}}\NormalTok{ \{}
  \KeywordTok{const}\NormalTok{ authStore }\OperatorTok{=} \FunctionTok{useAuthStore}\NormalTok{()}
  \ControlFlowTok{if}\NormalTok{ (}\OperatorTok{!}\NormalTok{authStore}\OperatorTok{.}\AttributeTok{isAuthenticated}\NormalTok{) \{}
    \FunctionTok{next}\NormalTok{(\{ name}\OperatorTok{:}\NormalTok{ LOGIN\_ROUTE\_NAME}\OperatorTok{,}\NormalTok{ query}\OperatorTok{:}\NormalTok{ \{ redirect}\OperatorTok{:}\NormalTok{ to}\OperatorTok{.}\AttributeTok{fullPath}\NormalTok{ \} \})}
\NormalTok{  \} }\ControlFlowTok{else}\NormalTok{ \{}
    \FunctionTok{next}\NormalTok{()}
\NormalTok{  \}}
\NormalTok{\}}

\ImportTok{export} \KeywordTok{const}\NormalTok{ requireRole }\OperatorTok{=}\NormalTok{ (role}\OperatorTok{:} \DataTypeTok{string}\NormalTok{) }\KeywordTok{=\textgreater{}}\NormalTok{ \{}
  \ControlFlowTok{return}\NormalTok{ (to}\OperatorTok{,} \ImportTok{from}\OperatorTok{,}\NormalTok{ next) }\KeywordTok{=\textgreater{}}\NormalTok{ \{}
    \KeywordTok{const}\NormalTok{ authStore }\OperatorTok{=} \FunctionTok{useAuthStore}\NormalTok{()}
    \ControlFlowTok{if}\NormalTok{ (authStore}\OperatorTok{.}\AttributeTok{user}\OperatorTok{?.}\AttributeTok{role} \OperatorTok{!==}\NormalTok{ role) \{}
      \FunctionTok{next}\NormalTok{(\{ name}\OperatorTok{:}\NormalTok{ FORBIDDEN\_ROUTE\_NAME \})}
\NormalTok{    \} }\ControlFlowTok{else}\NormalTok{ \{}
      \FunctionTok{next}\NormalTok{()}
\NormalTok{    \}}
\NormalTok{  \}}
\NormalTok{\}}
\end{Highlighting}
\end{Shaded}

Le guards guidano l'utente nell'utilizzo dell'applicazione, impedendo
l'accesso a pagine non autorizzate e reindirizzandolo automaticamente
(ad esempio, se un organizzazione vuole creare un evento ma non ha
effettuato l'accesso, prima si ha un redirect alla pagina di login).

\subsubsection{Comunicazione Real-time con
WebSocket}\label{comunicazione-real-time-con-websocket}

La comunicazione in tempo reale è gestita tramite
\href{http://socket.io/}{\textbf{Socket.IO}} per le notifiche e i
messaggi chat:

\begin{Shaded}
\begin{Highlighting}[]
\NormalTok{socket }\OperatorTok{=} \FunctionTok{io}\NormalTok{(url}\OperatorTok{,}\NormalTok{ \{}
\NormalTok{  auth}\OperatorTok{:}\NormalTok{ \{ token}\OperatorTok{,}\NormalTok{ userId \}}\OperatorTok{,}
\NormalTok{  reconnection}\OperatorTok{:} \KeywordTok{true}\OperatorTok{,}
\NormalTok{  reconnectionAttempts}\OperatorTok{:} \DecValTok{5}\OperatorTok{,}
\NormalTok{  transports}\OperatorTok{:}\NormalTok{ [}\StringTok{\textquotesingle{}websocket\textquotesingle{}}\OperatorTok{,} \StringTok{\textquotesingle{}polling\textquotesingle{}}\NormalTok{]}\OperatorTok{,}
\NormalTok{\})}

\NormalTok{socket}\OperatorTok{.}\FunctionTok{on}\NormalTok{(}\StringTok{\textquotesingle{}connect\textquotesingle{}}\OperatorTok{,}\NormalTok{ () }\KeywordTok{=\textgreater{}}\NormalTok{ \{}
\NormalTok{  handlers}\OperatorTok{.}\FunctionTok{forEach}\NormalTok{((\{ handler}\OperatorTok{,}\NormalTok{ eventType \}) }\KeywordTok{=\textgreater{}}\NormalTok{ \{}
\NormalTok{    socket}\OperatorTok{?.}\FunctionTok{on}\NormalTok{(eventType}\OperatorTok{,}\NormalTok{ handler)}
\NormalTok{  \})}
\NormalTok{\})}
\end{Highlighting}
\end{Shaded}

Gli eventi gestiti includono:

\begin{itemize}
\tightlist{}
\item
  \texttt{user-online} / \texttt{user-offline}~--- Stato online degli
  utenti
\item
  \texttt{new-message}~--- Nuovi messaggi in chat
\item
  \texttt{like-received} / \texttt{follow-received}~--- Notifiche di
  interazione
\item
  \texttt{new-event-published}~--- Nuovi eventi pubblicati da utenti
  seguiti
\end{itemize}

\subsubsection{Composables
Riutilizzabili}\label{composables-riutilizzabili}

I composables incapsulano logiche riutilizzabili tra i componenti. Un
esempio significativo è \texttt{useInfiniteScroll} per la paginazione:

\begin{Shaded}
\begin{Highlighting}[]
\ImportTok{export} \KeywordTok{function} \FunctionTok{useInfiniteScroll}\OperatorTok{\textless{}}\NormalTok{R}\OperatorTok{\textgreater{}}\NormalTok{(config}\OperatorTok{:}\NormalTok{ InfiniteScrollConfiguration}\OperatorTok{\textless{}}\NormalTok{R}\OperatorTok{\textgreater{}}\NormalTok{) \{}
  \KeywordTok{const}\NormalTok{ items}\OperatorTok{:}\NormalTok{ Ref}\OperatorTok{\textless{}}\NormalTok{R[]}\OperatorTok{\textgreater{}} \OperatorTok{=} \FunctionTok{ref}\NormalTok{([])}
  \KeywordTok{const}\NormalTok{ hasMore }\OperatorTok{=} \FunctionTok{ref}\NormalTok{(}\KeywordTok{true}\NormalTok{)}
  \KeywordTok{const}\NormalTok{ loading }\OperatorTok{=} \FunctionTok{ref}\NormalTok{(}\KeywordTok{true}\NormalTok{)}

  \KeywordTok{const}\NormalTok{ onLoad }\OperatorTok{=} \KeywordTok{async}\NormalTok{ (\_index}\OperatorTok{:} \DataTypeTok{number}\OperatorTok{,}\NormalTok{ done}\OperatorTok{:}\NormalTok{ (stop}\OperatorTok{?:} \DataTypeTok{boolean}\NormalTok{) }\KeywordTok{=\textgreater{}} \DataTypeTok{void}\NormalTok{) }\KeywordTok{=\textgreater{}}\NormalTok{ \{}
    \ControlFlowTok{if}\NormalTok{ (}\OperatorTok{!}\NormalTok{hasMore}\OperatorTok{.}\AttributeTok{value}\NormalTok{) \{}
      \FunctionTok{done}\NormalTok{(}\KeywordTok{true}\NormalTok{)}
      \ControlFlowTok{return}
\NormalTok{    \}}
    \ControlFlowTok{await} \FunctionTok{loadItems}\NormalTok{(}\KeywordTok{true}\NormalTok{)}
    \FunctionTok{done}\NormalTok{(}\OperatorTok{!}\NormalTok{hasMore}\OperatorTok{.}\AttributeTok{value}\NormalTok{)}
\NormalTok{  \}}

  \ControlFlowTok{return}\NormalTok{ \{ items}\OperatorTok{,}\NormalTok{ hasMore}\OperatorTok{,}\NormalTok{ loading}\OperatorTok{,}\NormalTok{ onLoad}\OperatorTok{,}\NormalTok{ reload \}}
\NormalTok{\}}
\end{Highlighting}
\end{Shaded}

Altri composables includono:

\begin{itemize}
\tightlist{}
\item
  \texttt{useUserProfile}~--- Logica profilo utente (isOwnProfile,
  isOrganization)
\item
  \texttt{useDarkMode}~--- Gestione tema chiaro/scuro con persistenza
\item
  \texttt{useTranslation}~--- Traduzioni con prefisso automatico
\end{itemize}

\subsubsection{Layout Condivisi}\label{layout-condivisi}

Sono stati definiti anche layout riutilizzabili per garantire
consistenza nell'interfaccia:

\begin{itemize}
\tightlist{}
\item
  \textbf{NavigationWithSearch}: Utilizzato in home ed explore, serve a
  gestire la comparsa della barra di ricerca nella barra di navigazione
  dal momento che esce dalla viewport e viceversa.
\item
  \textbf{TwoColumnLayout}: Utilizzato in chat e impostazioni, con
  supporto mobile che mostra una colonna alla volta
\end{itemize}

\subsubsection{Internazionalizzazione
(i18n)}\label{internazionalizzazione-i18n}

L'applicazione supporta 5 lingue (en, es, fr, it, de). Le traduzioni
sono generate automaticamente in CI a partire dal sorgente inglese:

\begin{Shaded}
\begin{Highlighting}[]
\KeywordTok{const}\NormalTok{ localeModules }\OperatorTok{=}\NormalTok{ import}\OperatorTok{.}\AttributeTok{meta}\OperatorTok{.}\FunctionTok{glob}\NormalTok{(}\StringTok{\textquotesingle{}./locales/*.ts\textquotesingle{}}\OperatorTok{,}\NormalTok{ \{ eager}\OperatorTok{:} \KeywordTok{true}\NormalTok{ \})}

\KeywordTok{const}\NormalTok{ i18n }\OperatorTok{=} \FunctionTok{createI18n}\NormalTok{(\{}
\NormalTok{  legacy}\OperatorTok{:} \KeywordTok{false}\OperatorTok{,}
\NormalTok{  locale}\OperatorTok{:}\NormalTok{ DEFAULT\_LOCALE}\OperatorTok{,}
\NormalTok{  fallbackLocale}\OperatorTok{:}\NormalTok{ DEFAULT\_LOCALE}\OperatorTok{,}
\NormalTok{  messages}\OperatorTok{,}
\NormalTok{\})}
\end{Highlighting}
\end{Shaded}

L'utilizzo nei componenti avviene tramite il composable useTranslation:

\begin{Shaded}
\begin{Highlighting}[]

\KeywordTok{const}\NormalTok{ \{ t \} }\OperatorTok{=} \FunctionTok{useTranslation}\NormalTok{(}\StringTok{\textquotesingle{}components.cards.EventCard\textquotesingle{}}\NormalTok{)}
\end{Highlighting}
\end{Shaded}

Inoltre il selettore delle lingue nelle impostazioni del profilo
utilizza l'API nativa \texttt{Intl.DisplayNames} per mostrare ogni
lingua nel proprio nome nativo (es. ``Italiano'', ``Français'',
``Deutsch''), migliorando l'accessibilità per gli utenti:

\begin{Shaded}
\begin{Highlighting}[]
\KeywordTok{const}\NormalTok{ getLanguageInfo }\OperatorTok{=}\NormalTok{ (code}\OperatorTok{:} \DataTypeTok{string}\NormalTok{)}\OperatorTok{:}\NormalTok{ LanguageOption }\KeywordTok{=\textgreater{}}\NormalTok{ \{}
  \KeywordTok{const}\NormalTok{ nativeNames }\OperatorTok{=} \KeywordTok{new} \BuiltInTok{Intl}\OperatorTok{.}\FunctionTok{DisplayNames}\NormalTok{([code]}\OperatorTok{,}\NormalTok{ \{ type}\OperatorTok{:} \StringTok{\textquotesingle{}language\textquotesingle{}}\NormalTok{ \})}
  \ControlFlowTok{return}\NormalTok{ \{}
\NormalTok{    code}\OperatorTok{,}
\NormalTok{    nativeName}\OperatorTok{:}\NormalTok{ nativeNames}\OperatorTok{.}\FunctionTok{of}\NormalTok{(code) }\OperatorTok{||}\NormalTok{ code}\OperatorTok{.}\FunctionTok{toUpperCase}\NormalTok{()}\OperatorTok{,}
\NormalTok{    flag}\OperatorTok{:} \FunctionTok{getFlagEmoji}\NormalTok{(code)}\OperatorTok{,}
\NormalTok{  \}}
\NormalTok{\}}
\end{Highlighting}
\end{Shaded}

L'internazionalizzazione si estende anche al backend: il servizio
Payments genera i \textbf{biglietti PDF nella lingua dell'utente}, con
traduzioni dedicate.

Il backend predispone inoltre la gestione dei prezzi con le valute e un
sistema di conversione con caching, attualmente non utilizzato dal
frontend ma pronto per future estensioni:

\begin{Shaded}
\begin{Highlighting}[]
\ImportTok{export} \KeywordTok{class}\NormalTok{ CurrencyConverter \{}
  \KeywordTok{private} \KeywordTok{static} \KeywordTok{readonly}\NormalTok{ CACHE\_DURATION }\OperatorTok{=} \DecValTok{24} \OperatorTok{*} \DecValTok{60} \OperatorTok{*} \DecValTok{60} \OperatorTok{*} \DecValTok{1000} \CommentTok{// 24 hours}

  \KeywordTok{static} \KeywordTok{async} \FunctionTok{convertAmount}\NormalTok{(amount}\OperatorTok{:} \DataTypeTok{number}\OperatorTok{,}\NormalTok{ fromCurrency}\OperatorTok{:} \DataTypeTok{string}\OperatorTok{,}\NormalTok{ toCurrency}\OperatorTok{:} \DataTypeTok{string}\NormalTok{)}\OperatorTok{:} \BuiltInTok{Promise}\OperatorTok{\textless{}}\DataTypeTok{number}\OperatorTok{\textgreater{}}\NormalTok{ \{}
    \KeywordTok{const}\NormalTok{ fromRates }\OperatorTok{=} \ControlFlowTok{await} \KeywordTok{this}\OperatorTok{.}\FunctionTok{fetchRatesForCurrency}\NormalTok{(}\ImportTok{from}\NormalTok{)}
    \ControlFlowTok{return} \KeywordTok{this}\OperatorTok{.}\AttributeTok{converter}\OperatorTok{.}\FunctionTok{convert}\NormalTok{(amount}\OperatorTok{,} \ImportTok{from}\OperatorTok{,}\NormalTok{ to}\OperatorTok{,}\NormalTok{ fromRates)}
\NormalTok{  \}}
\NormalTok{\}}
\end{Highlighting}
\end{Shaded}

\textbf{Limiti attuali}: I contenuti user-generated come le descrizioni
degli eventi non sono attualmente tradotti e vengono memorizzati nella
lingua in cui sono stati inseriti dall'organizzazione. Un'estensione
futura potrebbe prevedere la possibilità di inserire descrizioni in più
lingue, con fallback alla lingua originale quando la traduzione non è
disponibile.

\subsubsection{Geolocalizzazione degli
eventi}\label{geolocalizzazione-degli-eventi}

Per l'inserimento della posizione durante la creazione di un evento è
stata utilizzata l'API pubblica di \textbf{Nominatim} (OpenStreetMap)
per la ricerca e il geocoding degli indirizzi. La scelta di
OpenStreetMap è stata dettata dalla sua natura open-source e
dall'assenza di costi di utilizzo.

Poiché la maggior parte degli utenti utilizza Google Maps per la
navigazione, i dati ricevuti da Nominatim vengono elaborati per generare
link compatibili con Google Maps, permettendo all'utente di cliccare
sulla posizione dell'evento e aprirla direttamente nell'applicazione di
navigazione.

\begin{Shaded}
\begin{Highlighting}[]
\ImportTok{export} \KeywordTok{const}\NormalTok{ extractLocationMapsLink }\OperatorTok{=}\NormalTok{ (location}\OperatorTok{:}\NormalTok{ LocationData)}\OperatorTok{:} \DataTypeTok{string} \KeywordTok{=\textgreater{}}\NormalTok{ \{}
  \KeywordTok{const}\NormalTok{ query }\OperatorTok{=} \VerbatimStringTok{\textasciigrave{}}\SpecialCharTok{$\{}\NormalTok{location}\OperatorTok{.}\AttributeTok{name}\SpecialCharTok{\}}\VerbatimStringTok{,}\SpecialCharTok{$\{}\NormalTok{location}\OperatorTok{.}\AttributeTok{road}\SpecialCharTok{\}}\VerbatimStringTok{,}\SpecialCharTok{$\{}\NormalTok{location}\OperatorTok{.}\AttributeTok{city}\SpecialCharTok{\}}\VerbatimStringTok{,}\SpecialCharTok{$\{}\NormalTok{location}\OperatorTok{.}\AttributeTok{country}\SpecialCharTok{\}}\VerbatimStringTok{\textasciigrave{}}
  \ControlFlowTok{return} \VerbatimStringTok{\textasciigrave{}https://www.google.com/maps/search/?api=1\&query=}\SpecialCharTok{$\{}\PreprocessorTok{encodeURIComponent}\NormalTok{(query)}\SpecialCharTok{\}}\VerbatimStringTok{\textasciigrave{}}
\NormalTok{\}}
\end{Highlighting}
\end{Shaded}

\subsection{\texorpdfstring{\textbf{5.2
Backend}}{5.2 Backend}}\label{backend}

Il backend è composto da \textbf{7 microservizi}: 2 sviluppati in
\textbf{Scala 3} con il framework \textbf{Cask} (Users ed Events), 3 in
\textbf{NestJS} (Interactions, Chat e Payments) e 2 in
\textbf{Express.js} (Notifications e Media).

Ogni servizio ha la propria istanza \textbf{MongoDB} dedicata e comunica
con gli altri tramite \textbf{RabbitMQ} con topic exchange. Il servizio
Notifications gestisce le connessioni WebSocket tramite
\textbf{Socket.IO} per le notifiche real-time e il tracciamento dello
stato online degli utenti.

\subsubsection{Struttura del Progetto}\label{struttura-del-progetto-1}

Essendo i microservizi eterogenei segue una descrizione di massima
rappresentativa della struttura delle varie implementazioni, viene usata
la terminologia dello stack MEVN ma alcuni componenti potrebbero avere
un nome/implementazione diversa (e.i. in Nest il concetto di router e
controller viene unito in un unico componente rispetto ad express).

\begin{Shaded}
\begin{Highlighting}[]
\NormalTok{/src/}
\NormalTok{├── presentation/        \# Layer di ingresso al servizio}
\NormalTok{├── application/         \# Layer contenenti le logiche applicative}
\NormalTok{├── domain/              \# Modello dei dati}
\NormalTok{└── infrastructure/      \# Dipendenze del dominio}
\end{Highlighting}
\end{Shaded}

Più nel dettaglio, nel primo layer di presentazione troviamo i router
che definiscono le rotte (path + metodo http) supportate da ciascun
servizio. In application sono specificati i diversi DTO usati per
validare i dati in ingresso agli endpoint e strutturare i dati di
risposta degli stessi e i controller delle varie rotte. Nel dominio sono
definiti i modelli delle entità di interesse per il particolare servizio
e in infrastructure troviamo le varie dipendenze esterne, tipicamente le
implementazioni dei connectors a database e message-broker

\subsubsection{Autenticazione JWT}\label{autenticazione-jwt}

L'autenticazione è basata su \textbf{JWT}. Il servizio Users integra
\textbf{Keycloak} come identity provider per la gestione delle
credenziali utente (registrazione, login, gestione delle sessioni).
Keycloak genera i token JWT firmati e il servizio Users espone un
endpoint \texttt{/public-keys} che restituisce le chiavi pubbliche
necessarie per la validazione.

Gli altri microservizi recuperano le chiavi pubbliche da questo endpoint
e le salvano in cache localmente. Quando ricevono una richiesta
autenticata, estraggono il token dall'header \texttt{Authorization}, ne
verificano la firma e decodificano i claim.

Oltre alle REST API, l'autenticazione è stata implementata anche per le
connessioni WebSocket: il token viene passato durante l'handshake della
connessione Socket.IO e validato prima di stabilire il canale. Questo
permette di inviare notifiche con dati (oltre che solo segnali) in modo
sicuro.

\subsubsection{Express Middlewares}\label{express-middlewares}

Per gestire le richieste HTTP in \textbf{Express} sono stati utilizzati
i middleware, in particolare è stato definito un middleware per
l'autenticazione:

\begin{Shaded}
\begin{Highlighting}[]
\ImportTok{export} \KeywordTok{function} \FunctionTok{createAuthMiddleware}\NormalTok{(options}\OperatorTok{:}\NormalTok{ \{ optional}\OperatorTok{?:} \DataTypeTok{boolean}\NormalTok{ \} }\OperatorTok{=}\NormalTok{ \{\}) \{}
  \ControlFlowTok{return}\NormalTok{ (req}\OperatorTok{:}\NormalTok{ Request}\OperatorTok{,}\NormalTok{ res}\OperatorTok{:}\NormalTok{ Response}\OperatorTok{,}\NormalTok{ next}\OperatorTok{:}\NormalTok{ NextFunction)}\OperatorTok{:} \DataTypeTok{void} \KeywordTok{=\textgreater{}}\NormalTok{ \{}
    \KeywordTok{const}\NormalTok{ token }\OperatorTok{=}\NormalTok{ req}\OperatorTok{.}\AttributeTok{headers}\OperatorTok{.}\AttributeTok{authorization}\OperatorTok{?.}\FunctionTok{replace}\NormalTok{(}\StringTok{"Bearer "}\OperatorTok{,} \StringTok{""}\NormalTok{)}

    \ControlFlowTok{if}\NormalTok{ (}\OperatorTok{!}\NormalTok{token) \{}
      \ControlFlowTok{if}\NormalTok{ (options}\OperatorTok{.}\AttributeTok{optional}\NormalTok{) }\ControlFlowTok{return} \FunctionTok{next}\NormalTok{()}
      \ControlFlowTok{return}\NormalTok{ res}\OperatorTok{.}\FunctionTok{status}\NormalTok{(}\DecValTok{401}\NormalTok{)}\OperatorTok{.}\FunctionTok{json}\NormalTok{(\{ error}\OperatorTok{:} \StringTok{"No token provided"}\NormalTok{ \})}
\NormalTok{    \}}

    \ControlFlowTok{try}\NormalTok{ \{}
      \KeywordTok{const}\NormalTok{ payload }\OperatorTok{=} \ControlFlowTok{await}\NormalTok{ JwtService}\OperatorTok{.}\FunctionTok{verifyToken}\NormalTok{(token)}
\NormalTok{      req}\OperatorTok{.}\AttributeTok{userId} \OperatorTok{=}\NormalTok{ payload}\OperatorTok{.}\AttributeTok{user\_id}
      \FunctionTok{next}\NormalTok{()}
\NormalTok{    \} }\ControlFlowTok{catch}\NormalTok{ (error) \{}
\NormalTok{      res}\OperatorTok{.}\FunctionTok{status}\NormalTok{(}\DecValTok{401}\NormalTok{)}\OperatorTok{.}\FunctionTok{json}\NormalTok{(\{ error}\OperatorTok{:} \StringTok{"Authentication failed"}\NormalTok{ \})}
\NormalTok{    \}}
\NormalTok{  \}}
\NormalTok{\}}

\ImportTok{export} \KeywordTok{const}\NormalTok{ authMiddleware }\OperatorTok{=} \FunctionTok{createAuthMiddleware}\NormalTok{()}
\ImportTok{export} \KeywordTok{const}\NormalTok{ optionalAuthMiddleware }\OperatorTok{=} \FunctionTok{createAuthMiddleware}\NormalTok{(\{ optional}\OperatorTok{:} \KeywordTok{true}\NormalTok{ \})}
\end{Highlighting}
\end{Shaded}

Le routes utilizzano poi questo middleware per proteggere gli endpoint:

\begin{Shaded}
\begin{Highlighting}[]
\ImportTok{export} \KeywordTok{function} \FunctionTok{createNotificationRoutes}\NormalTok{(controller}\OperatorTok{:}\NormalTok{ NotificationController)}\OperatorTok{:}\NormalTok{ Router \{}
  \KeywordTok{const}\NormalTok{ router }\OperatorTok{=} \FunctionTok{Router}\NormalTok{()}
\NormalTok{  router}\OperatorTok{.}\FunctionTok{use}\NormalTok{(authMiddleware)}

\NormalTok{  router}\OperatorTok{.}\FunctionTok{get}\NormalTok{(}\StringTok{"/"}\OperatorTok{,}\NormalTok{ (req}\OperatorTok{,}\NormalTok{ res}\OperatorTok{,}\NormalTok{ next) }\KeywordTok{=\textgreater{}}
\NormalTok{    controller}\OperatorTok{.}\FunctionTok{getNotificationsByUserId}\NormalTok{(req}\OperatorTok{,}\NormalTok{ res}\OperatorTok{,}\NormalTok{ next))}
\NormalTok{  router}\OperatorTok{.}\FunctionTok{get}\NormalTok{(}\StringTok{"/unread{-}count"}\OperatorTok{,}\NormalTok{ (req}\OperatorTok{,}\NormalTok{ res}\OperatorTok{,}\NormalTok{ next) }\KeywordTok{=\textgreater{}}
\NormalTok{    controller}\OperatorTok{.}\FunctionTok{getUnreadCount}\NormalTok{(req}\OperatorTok{,}\NormalTok{ res}\OperatorTok{,}\NormalTok{ next))}
    \CommentTok{//other routes...}

  \ControlFlowTok{return}\NormalTok{ router}
\NormalTok{\}}
\end{Highlighting}
\end{Shaded}

A livello applicativo vengono inoltre utilizzati i middleware standard
di Express:~\texttt{cors()}~per gestire le richieste cross-origin
(necessario per le chiamate dal frontend) e~\texttt{express.json()}~per
il parsing automatico del body JSON.

\subsubsection{Authenticated WebSocket Gateway con
Socket.IO}\label{authenticated-websocket-gateway-con-socket.io}

Il gateway \textbf{Socket.IO} gestisce l'autenticazione delle
connessioni WebSocket e la distribuzione delle notifiche:

\begin{Shaded}
\begin{Highlighting}[]
\ImportTok{export} \KeywordTok{class}\NormalTok{ SocketIOGateway }\KeywordTok{implements}\NormalTok{ NotificationGateway \{}
  \KeywordTok{private}\NormalTok{ userSockets}\OperatorTok{:} \BuiltInTok{Map}\OperatorTok{\textless{}}\DataTypeTok{string}\OperatorTok{,} \BuiltInTok{Set}\OperatorTok{\textless{}}\DataTypeTok{string}\OperatorTok{\textgreater{}\textgreater{}} \OperatorTok{=} \KeywordTok{new} \BuiltInTok{Map}\NormalTok{()}

  \KeywordTok{private} \FunctionTok{setupAuthMiddleware}\NormalTok{()}\OperatorTok{:} \DataTypeTok{void}\NormalTok{ \{}
    \KeywordTok{this}\OperatorTok{.}\AttributeTok{io}\OperatorTok{.}\FunctionTok{use}\NormalTok{(}\KeywordTok{async}\NormalTok{ (socket}\OperatorTok{:} \BuiltInTok{Socket}\OperatorTok{,}\NormalTok{ next) }\KeywordTok{=\textgreater{}}\NormalTok{ \{}
      \KeywordTok{const}\NormalTok{ token }\OperatorTok{=}\NormalTok{ socket}\OperatorTok{.}\AttributeTok{handshake}\OperatorTok{.}\AttributeTok{auth}\OperatorTok{.}\AttributeTok{token} \OperatorTok{||}
\NormalTok{                    socket}\OperatorTok{.}\AttributeTok{handshake}\OperatorTok{.}\AttributeTok{headers}\OperatorTok{.}\AttributeTok{authorization}\OperatorTok{?.}\FunctionTok{replace}\NormalTok{(}\StringTok{"Bearer "}\OperatorTok{,} \StringTok{""}\NormalTok{)}

      \ControlFlowTok{if}\NormalTok{ (}\OperatorTok{!}\NormalTok{token) \{}
        \ControlFlowTok{return} \FunctionTok{next}\NormalTok{(}\KeywordTok{new} \BuiltInTok{Error}\NormalTok{(}\StringTok{"Authentication error: No token provided"}\NormalTok{))}
\NormalTok{      \}}

      \ControlFlowTok{try}\NormalTok{ \{}
        \KeywordTok{const}\NormalTok{ payload }\OperatorTok{=} \ControlFlowTok{await}\NormalTok{ JwtService}\OperatorTok{.}\FunctionTok{verifyToken}\NormalTok{(token)}
\NormalTok{        socket}\OperatorTok{.}\AttributeTok{data}\OperatorTok{.}\AttributeTok{userId} \OperatorTok{=}\NormalTok{ payload}\OperatorTok{.}\AttributeTok{user\_id}
        \FunctionTok{next}\NormalTok{()}
\NormalTok{      \} }\ControlFlowTok{catch}\NormalTok{ (err) \{}
        \FunctionTok{next}\NormalTok{(}\KeywordTok{new} \BuiltInTok{Error}\NormalTok{(}\StringTok{"Authentication error: Invalid token"}\NormalTok{))}
\NormalTok{      \}}
\NormalTok{    \})}
\NormalTok{  \}}

  \FunctionTok{sendNotificationToUser}\NormalTok{(userId}\OperatorTok{:} \DataTypeTok{string}\OperatorTok{,}\NormalTok{ notification}\OperatorTok{:} \DataTypeTok{any}\NormalTok{)}\OperatorTok{:} \BuiltInTok{Promise}\OperatorTok{\textless{}}\DataTypeTok{void}\OperatorTok{\textgreater{}}\NormalTok{ \{}
    \KeywordTok{const}\NormalTok{ topic }\OperatorTok{=} \KeywordTok{this}\OperatorTok{.}\FunctionTok{getTopicFromNotificationType}\NormalTok{(notification}\OperatorTok{.}\AttributeTok{type}\NormalTok{)}
    \KeywordTok{this}\OperatorTok{.}\AttributeTok{io}\OperatorTok{.}\FunctionTok{to}\NormalTok{(}\VerbatimStringTok{\textasciigrave{}user:}\SpecialCharTok{$\{}\NormalTok{userId}\SpecialCharTok{\}}\VerbatimStringTok{\textasciigrave{}}\NormalTok{)}\OperatorTok{.}\FunctionTok{emit}\NormalTok{(topic}\OperatorTok{,}\NormalTok{ notification)}
    \ControlFlowTok{return} \BuiltInTok{Promise}\OperatorTok{.}\FunctionTok{resolve}\NormalTok{()}
\NormalTok{  \}}

  \FunctionTok{broadcastUserOnline}\NormalTok{(userId}\OperatorTok{:} \DataTypeTok{string}\NormalTok{)}\OperatorTok{:} \DataTypeTok{void}\NormalTok{ \{}
    \KeywordTok{this}\OperatorTok{.}\AttributeTok{io}\OperatorTok{.}\FunctionTok{except}\NormalTok{(}\VerbatimStringTok{\textasciigrave{}user:}\SpecialCharTok{$\{}\NormalTok{userId}\SpecialCharTok{\}}\VerbatimStringTok{\textasciigrave{}}\NormalTok{)}\OperatorTok{.}\FunctionTok{emit}\NormalTok{(}\StringTok{"user{-}online"}\OperatorTok{,}\NormalTok{ \{}
\NormalTok{      userId}\OperatorTok{,}
\NormalTok{      timestamp}\OperatorTok{:} \KeywordTok{new} \BuiltInTok{Date}\NormalTok{()}
\NormalTok{    \})}
\NormalTok{  \}}

  \FunctionTok{broadcastUserOffline}\NormalTok{(userId}\OperatorTok{:} \DataTypeTok{string}\NormalTok{)}\OperatorTok{:} \DataTypeTok{void}\NormalTok{ \{}
    \KeywordTok{this}\OperatorTok{.}\AttributeTok{io}\OperatorTok{.}\FunctionTok{emit}\NormalTok{(}\StringTok{"user{-}offline"}\OperatorTok{,}\NormalTok{ \{}
\NormalTok{      userId}\OperatorTok{,}
\NormalTok{      timestamp}\OperatorTok{:} \KeywordTok{new} \BuiltInTok{Date}\NormalTok{()}
\NormalTok{    \})}
\NormalTok{  \}}
\NormalTok{\}}
\end{Highlighting}
\end{Shaded}

\subsubsection{Endpoint con autenticazione
opzionale}\label{endpoint-con-autenticazione-opzionale}

Alcuni endpoint utilizzano l'\textbf{autenticazione opzionale}: sono
accessibili sia da utenti autenticati che non, in questo caso la
risposta può variare pur mantenendo il formato consistente. Ad esempio,
l'endpoint per ottenere il profilo di un utente restituisce solo i dati
pubblici (username e informazioni del profilo) se la richiesta non è
autenticata, mentre include anche i dati privati dell'account
(e.g.~email, interessi) se l'utente autenticato sta richiedendo le
proprie informazioni:

\begin{Shaded}
\begin{Highlighting}[]
\NormalTok{@cask}\OperatorTok{.}\FunctionTok{get}\OperatorTok{(}\StringTok{"/:userId"}\OperatorTok{)}
  \KeywordTok{def} \FunctionTok{getUser}\OperatorTok{(}\NormalTok{userId}\OperatorTok{:} \ExtensionTok{String}\OperatorTok{,}\NormalTok{ req}\OperatorTok{:} \ExtensionTok{Request}\OperatorTok{):} \ExtensionTok{Response}\OperatorTok{[}\ExtensionTok{String}\OperatorTok{]} \OperatorTok{=}
\NormalTok{    userService}\OperatorTok{.}\FunctionTok{getUserById}\OperatorTok{(}\NormalTok{userId}\OperatorTok{)} \ControlFlowTok{match}
      \ControlFlowTok{case} \FunctionTok{Left}\OperatorTok{(}\NormalTok{err}\OperatorTok{)} \OperatorTok{=\textgreater{}} \ExtensionTok{Response}\OperatorTok{(}\NormalTok{err}\OperatorTok{,} \DecValTok{404}\OperatorTok{)}
      \ControlFlowTok{case} \FunctionTok{Right}\OperatorTok{(}\NormalTok{role}\OperatorTok{,}\NormalTok{ user}\OperatorTok{)} \OperatorTok{=\textgreater{}}
        \KeywordTok{val}\NormalTok{ isOwner}\OperatorTok{:} \ExtensionTok{Boolean} \OperatorTok{=} \FunctionTok{authenticateAndAuthorize}\OperatorTok{(}\NormalTok{req}\OperatorTok{,}\NormalTok{ userId}\OperatorTok{).}\NormalTok{isRight}
        \KeywordTok{val}\NormalTok{ json }\OperatorTok{=} \ControlFlowTok{if}\NormalTok{ isOwner then}
\NormalTok{            user}\OperatorTok{.}\FunctionTok{toOwnedUserDTO}\OperatorTok{(}\NormalTok{userId}\OperatorTok{,}\NormalTok{ role}\OperatorTok{).}\NormalTok{asJson}
        \ControlFlowTok{else}
\NormalTok{          user}\OperatorTok{.}\FunctionTok{toUserDTO}\OperatorTok{(}\NormalTok{userId}\OperatorTok{,}\NormalTok{ role}\OperatorTok{).}\NormalTok{asJson}
        \ExtensionTok{Response}\OperatorTok{(}\NormalTok{json}\OperatorTok{.}\NormalTok{spaces2}\OperatorTok{,} \DecValTok{200}\OperatorTok{,} \BuiltInTok{Seq}\OperatorTok{(}\StringTok{"Content{-}Type"} \OperatorTok{{-}\textgreater{}} \StringTok{"application/json"}\OperatorTok{))}
\end{Highlighting}
\end{Shaded}
