\section{7~--- Deployment}\label{deployment}

Il deploy dell'applicazione è stato automatizzato tramite Docker e
Docker Compose: tutti i servizi vengono eseguiti come container
indipendenti e lanciati tramite uno script centralizzato.

\subsection{7.1~--- Installazione}\label{installazione}

\subsubsection{1. Clonare il repository}\label{clonare-il-repository}

\begin{Shaded}
\begin{Highlighting}[]
\FunctionTok{git}\NormalTok{ clone https://github.com/EvenToNight/EvenToNight.git}
\BuiltInTok{cd}\NormalTok{ EvenToNight}
\end{Highlighting}
\end{Shaded}

\subsubsection{2. Configurare le variabili
d'ambiente}\label{configurare-le-variabili-dambiente}

\begin{Shaded}
\begin{Highlighting}[]
\FunctionTok{cp}\NormalTok{ .env.template .env}
\CommentTok{\# Modificare .env e compilare tutti i campi richiesti}
\CommentTok{\# Nota: Se si utilizza il flag {-}{-}no{-}deps, le chiavi Stripe possono contenere valori arbitrari. Tutti i valori devono essere compilati.}
\end{Highlighting}
\end{Shaded}

\subsubsection{3. Avviare l'applicazione}\label{avviare-lapplicazione}

\paragraph{Opzione A:\@ Utilizzo di immagini pre-compilate da ghcr.io
(Consigliato)}\label{opzione-a-utilizzo-di-immagini-pre-compilate-da-ghcr.io-consigliato}

\textbf{Download delle immagini:}

\begin{Shaded}
\begin{Highlighting}[]
\ExtensionTok{./scripts/composeApplication.sh}\NormalTok{ pull}
\end{Highlighting}
\end{Shaded}

\textbf{Download delle immagini con seeding del database:}

\begin{Shaded}
\begin{Highlighting}[]
\ExtensionTok{./scripts/composeApplication.sh} \AttributeTok{{-}{-}init{-}db}\NormalTok{ pull}
\end{Highlighting}
\end{Shaded}

\textbf{Deploy dell'applicazione:}

\begin{Shaded}
\begin{Highlighting}[]
\ExtensionTok{./scripts/composeApplication.sh}\NormalTok{ up }\AttributeTok{{-}d} \AttributeTok{{-}{-}wait}
\end{Highlighting}
\end{Shaded}

\textbf{Deploy con seeding del database:}

\begin{Shaded}
\begin{Highlighting}[]
\ExtensionTok{./scripts/composeApplication.sh} \AttributeTok{{-}{-}init{-}db}\NormalTok{ up }\AttributeTok{{-}d} \AttributeTok{{-}{-}wait}
\end{Highlighting}
\end{Shaded}

\textbf{Deploy in modalità sviluppo} (con porte mappate sull'host e
dashboard per database/RabbitMQ/Traefik):

\begin{Shaded}
\begin{Highlighting}[]
\ExtensionTok{./scripts/composeApplication.sh} \AttributeTok{{-}{-}init{-}db} \AttributeTok{{-}{-}dev}\NormalTok{ up }\AttributeTok{{-}d} \AttributeTok{{-}{-}wait}
\end{Highlighting}
\end{Shaded}

\paragraph{Opzione B:\@ Build locale}\label{opzione-b-build-locale}

Aggiungere il flag \texttt{-\/-build} per compilare i servizi localmente
invece di utilizzare le immagini pre-compilate:

\begin{Shaded}
\begin{Highlighting}[]
\CommentTok{\# Build e deploy}
\ExtensionTok{./scripts/composeApplication.sh}\NormalTok{ up }\AttributeTok{{-}{-}build} \AttributeTok{{-}d} \AttributeTok{{-}{-}wait}

\CommentTok{\# Build e deploy con seeding}
\ExtensionTok{./scripts/composeApplication.sh} \AttributeTok{{-}{-}init{-}db}\NormalTok{ up }\AttributeTok{{-}{-}build} \AttributeTok{{-}d} \AttributeTok{{-}{-}wait}
\end{Highlighting}
\end{Shaded}

\paragraph{Flag aggiuntivi}\label{flag-aggiuntivi}

\textbf{\texttt{-\/-no-deps}}: Esclude le dipendenze esterne (Stripe)

È possibile aggiungere \texttt{-\/-no-deps} a qualsiasi comando di
deploy per escludere i servizi esterni:

\begin{Shaded}
\begin{Highlighting}[]
\CommentTok{\# Deploy senza dipendenze esterne}
\ExtensionTok{./scripts/composeApplication.sh} \AttributeTok{{-}{-}no{-}deps}\NormalTok{ up }\AttributeTok{{-}d} \AttributeTok{{-}{-}wait}

\CommentTok{\# Deploy con seeding ma senza dipendenze esterne}
\ExtensionTok{./scripts/composeApplication.sh} \AttributeTok{{-}{-}init{-}db} \AttributeTok{{-}{-}no{-}deps}\NormalTok{ up }\AttributeTok{{-}d} \AttributeTok{{-}{-}wait}
\end{Highlighting}
\end{Shaded}

\textbf{Nota:} Quando si utilizza \texttt{-\/-no-deps}, le chiavi Stripe \textbf{(STRIPE\_SECRET\_KEY, STRIPE\_PUBLISHABLE\_KEY, STRIPE\_WEBHOOK\_SECRET)} in \texttt{.env} possono contenere valori arbitrari.

\paragraph{Configurazione Stripe}\label{configurazione-stripe}

\textbf{Per i pagamenti Stripe in ambiente locale} (richiesto solo se
NON si utilizza \texttt{-\/-no-deps}):

\begin{Shaded}
\begin{Highlighting}[]
\ExtensionTok{./services/payments/scripts/local{-}webhooks.sh}
\end{Highlighting}
\end{Shaded}

Questo script deve essere eseguito per inoltrare i webhook di Stripe
all'ambiente locale.

Per maggiori informazioni sull'utilizzo della modalità sandbox,
consultare la \href{https://docs.stripe.com/testing}{documentazione
Stripe}.

\subsubsection{Setup alternativo}\label{setup-alternativo}

Utilizzare Gradle per configurare l'intero ambiente con seeding e
listener Stripe:

\begin{Shaded}
\begin{Highlighting}[]
\ExtensionTok{./gradlew}\NormalTok{ setupApplicationEnvironment}
\end{Highlighting}
\end{Shaded}

\subsubsection{Teardown}\label{teardown}

\textbf{Arresto dell'applicazione:}

\begin{Shaded}
\begin{Highlighting}[]
\ExtensionTok{./scripts/composeApplication.sh}\NormalTok{ down}
\end{Highlighting}
\end{Shaded}

\textbf{Arresto e rimozione dei volumi:}

\begin{Shaded}
\begin{Highlighting}[]
\ExtensionTok{./scripts/composeApplication.sh}\NormalTok{ down }\AttributeTok{{-}v}
\end{Highlighting}
\end{Shaded}

\begin{center}\rule{0.5\linewidth}{0.5pt}\end{center}

Come ulteriore alternativa, è possibile visualizzare il sito già in
produzione al link \url{https://eventonight.site/it}
