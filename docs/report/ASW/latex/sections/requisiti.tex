\section{2~--- Requisiti}\label{requisiti}

Di seguito sono riportati i principali requisiti che l'applicazione deve
soddisfare.

\subsection{2.1~--- Requisiti di Business}\label{requisiti-di-business}

\begin{itemize}
\tightlist{}
\item
  La piattaforma consente alle organizzazioni di creare e pubblicare
  post relativi agli eventi da loro promossi.
\item
  Gli utenti possono utilizzare la piattaforma come punto di riferimento
  per scoprire eventi nelle vicinanze, in base alla località e ai propri
  interessi.
\item
  Il sistema abilita la vendita online dei biglietti degli eventi
  fornendo alle organizzazioni uno strumento per monetizzare le proprie
  attività.
\end{itemize}

\subsection{2.2~--- Requisiti Funzionali}\label{requisiti-funzionali}

\textbf{Tipologie di utenti supportate dal sistema:}

\begin{itemize}
\tightlist{}
\item
  Utenti non registrati.
\item
  Utenti registrati, che possono fruire dei contenuti della piattaforma.
\item
  Utenti registrati come organizzazioni, che possono creare eventi,
  vendere biglietti e fruire dei contenuti della piattaforma.
\end{itemize}

\textbf{Per tutti gli utenti:}

\begin{itemize}
\tightlist{}
\item
  Visualizzare la schermata Home con le modalità di interazione: ricerca
  eventi, visualizzazione eventi popolari, prossimi eventi e nuove
  aggiunte.
\item
  Visualizzare dalla schermata Esplora tutti gli eventi pubblicati sulla
  piattaforma, tutti gli utenti registrati e applicare filtri di
  ricerca.
\end{itemize}

\textbf{Per utenti registrati:}

\begin{itemize}
\tightlist{}
\item
  Ricevere un feed di eventi personalizzato, basato sugli interessi
  specificati.
\item
  Mettere e togliere like a un evento.
\item
  Mettere e togliere follow a un membro e a un'organizzazione.
\item
  Acquistare biglietti per gli eventi.
\item
  Lasciare una recensione dopo la partecipazione a un evento.
\item
  Contattare direttamente le organizzazioni all'interno della
  piattaforma per richiedere supporto.
\item
  Ricevere notifiche su:

  \begin{itemize}
  \tightlist{}
  \item
    nuovo follower.
  \item
    pubblicazione di nuovo evento da parte di organizzazione seguita.
  \item
    nuovo messaggio.
  \end{itemize}
\end{itemize}

\textbf{Per utenti registrati come organizzazioni:}

\begin{itemize}
\tightlist{}
\item
  Creare eventi, scegliendo se renderli pubblici o salvarli come bozza.
\item
  Specificare collaboratori durante la creazione degli eventi.
\item
  Ricevere notifiche su like e recensioni ai propri eventi.
\end{itemize}

\subsection{2.3~--- Requisiti Non
Funzionali}\label{requisiti-non-funzionali}

\begin{itemize}
\tightlist{}
\item
  Accessibilità: l'interfaccia grafica deve essere accessibile.
\item
  Portabilità: l'applicazione risulta responsive per adattarsi a schermi
  di diverse dimensioni pc/tablet/mobile.
\item
  Deployability: il sistema in automatico deve aggiornarsi alla versione
  dell'ultima release.
\item
  Availability: il sistema deve essere tollerante ai guasti per
  garantire la disponibilità, deve poter effettuare un recupero
  automatico in caso di errore e prevedere la ridondanza dei componenti
  critici per assicurare la continuità del servizio.
\item
  Sicurezza: gli utenti del sistema devono autenticarsi per verificare
  la loro identità e saranno poi autorizzati ad accedere alle risorse in
  base alle regole definite. Inoltre per assicurare la confidenzialità
  delle password queste saranno salvate in modo cifrato.
\item
  Robustezza: l'applicazione deve gestire input errati e generare errori
  coerenti.
\item
  Affidabilità: l'applicazione deve essere stabile, evitando crash.
\item
  Manutenibilità: il codice deve essere ben strutturato e ben
  documentato.
\item
  Estendibilità: il progetto deve favorire la personalizzazione e
  l'aggiunta di funzionalità.
\end{itemize}

Inoltre si è scelto, anche per esigenze didattiche, di aggiungere come
requisito architetturale lo sviluppo del sistema con un'architettura a
microservizi.
