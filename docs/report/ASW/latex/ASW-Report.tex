\documentclass{report}
\usepackage[utf8]{inputenc}
\usepackage[T1]{fontenc}
\usepackage[italian]{babel}
\usepackage{graphicx}
\usepackage{hyperref}
\usepackage{listings}
\usepackage{xcolor}
\usepackage{fancyvrb}
\usepackage{fvextra}
\usepackage{framed}
\usepackage{newunicodechar}
\usepackage{float}
\usepackage[export]{adjustbox}
\usepackage{mdframed}

% Definizione caratteri Unicode per box-drawing
\newunicodechar{├}{\texttt{|--}}
\newunicodechar{─}{\texttt{-}}
\newunicodechar{└}{\texttt{`--}}
\newunicodechar{│}{\texttt{|}}

% Definizione comandi pandoc
\providecommand{\tightlist}{\setlength{\itemsep}{0pt}\setlength{\parskip}{0pt}}
\newcommand{\pandocbounded}[1]{\resizebox{\ifdim\width>\textwidth\textwidth\else\width\fi}{!}{#1}}
\DefineVerbatimEnvironment{Highlighting}{Verbatim}{commandchars=\\\{\},fontsize=\small,breaklines,breakanywhere}
\definecolor{shadecolor}{RGB}{248,248,250}
\definecolor{framecolor}{RGB}{230,230,235}

\mdfdefinestyle{codebox}{
  backgroundcolor=shadecolor,
  linecolor=framecolor,
  linewidth=0.5pt,
  roundcorner=3pt,
  innerleftmargin=4pt,
  innerrightmargin=4pt,
  innertopmargin=6pt,
  innerbottommargin=6pt,
  skipabove=10pt,
  skipbelow=10pt
}

\newenvironment{Shaded}{%
  \begin{mdframed}[style=codebox]%
}{%
  \end{mdframed}%
}
\newcommand{\NormalTok}[1]{#1}
\newcommand{\KeywordTok}[1]{\textcolor{blue}{#1}}
\newcommand{\StringTok}[1]{\textcolor{red}{#1}}
\newcommand{\CommentTok}[1]{\textcolor{gray}{#1}}
\newcommand{\OperatorTok}[1]{\textcolor{violet}{#1}}
\newcommand{\FunctionTok}[1]{\textcolor{blue}{#1}}
\newcommand{\ExtensionTok}[1]{\textcolor{teal}{#1}}
\newcommand{\DecValTok}[1]{\textcolor{orange}{#1}}
\newcommand{\BuiltInTok}[1]{\textcolor{cyan}{#1}}
\newcommand{\ControlFlowTok}[1]{\textcolor{magenta}{#1}}
\newcommand{\VariableTok}[1]{\textcolor{olive}{#1}}
\newcommand{\SpecialCharTok}[1]{\textcolor{red}{#1}}
\newcommand{\DataTypeTok}[1]{\textcolor{purple}{#1}}
\newcommand{\ImportTok}[1]{\textcolor{blue}{#1}}
\newcommand{\AttributeTok}[1]{\textcolor{teal}{#1}}
\newcommand{\RegionMarkerTok}[1]{#1}
\newcommand{\InformationTok}[1]{\textcolor{gray}{#1}}
\newcommand{\WarningTok}[1]{\textcolor{red}{\textbf{#1}}}
\newcommand{\AlertTok}[1]{\textcolor{red}{\textbf{#1}}}
\newcommand{\ErrorTok}[1]{\textcolor{red}{\textbf{#1}}}
\newcommand{\VerbatimStringTok}[1]{\textcolor{red}{#1}}
\newcommand{\PreprocessorTok}[1]{\textcolor{magenta}{#1}}

\title{
    EvenToNight \\
    \large Applicazioni e Servizi Web
}

\author{
    Federico Bravetti \{federico.bravetti2@studio.unibo.it\} \and
    Tommaso Brini \{tommaso.brini@studio.unibo.it\} \and
    Alice Alfonsi \{alice.alfonsi2@studio.unibo.it\}
}

\date{\today}

\usepackage{natbib}

% Configurazione hyperref
\hypersetup{
    colorlinks=true,
    linkcolor=blue,
    filecolor=magenta,
    urlcolor=cyan,
}

% Path per le immagini
\graphicspath{{../public/}}

% Limita automaticamente le immagini troppo grandi
\let\oldincludegraphics\includegraphics%
\renewcommand{\includegraphics}[2][]{\oldincludegraphics[#1,max width=\textwidth,max height=0.85\textheight,keepaspectratio]{#2}}

% Configurazione listings per codice
\lstset{
    basicstyle=\ttfamily\small,
    breaklines=true,
    frame=single,
    numbers=left,
    numberstyle=\tiny,
    keywordstyle=\color{blue},
    commentstyle=\color{gray},
    stringstyle=\color{red}
}

\begin{document}

\maketitle

\tableofcontents

\chapter{Introduzione}

Questo progetto ha portato alla realizzazione di una piattaforma
digitale chiamata \textbf{EvenToNight}, pensata per mettere in contatto
organizzazioni che promuovono eventi sociali e utenti interessati a
scoprirli e parteciparvi.

La piattaforma è caratterizzata da un'interfaccia in stile social
network, rendendo così l'esperienza di utilizzo semplice, intuitiva e
coinvolgente.

L'applicazione consente agli utenti di:

\begin{itemize}
\item
  \textbf{Esplorare} eventi culturali, musicali e sociali
\item
  \textbf{Acquistare biglietti} direttamente dalla piattaforma
\item
  \textbf{Interagire} con le organizzazioni attraverso recensioni e chat
\item
  \textbf{Salvare} e tenere traccia degli eventi di interesse
\end{itemize}

Per le organizzazioni, la piattaforma offre:

\begin{itemize}
\item
  \textbf{Visibilità} per i propri eventi
\item
  \textbf{Gestione} completa di eventi e biglietteria
\item
  \textbf{Analytics} sull'engagement degli utenti (like, follower)
\item
  \textbf{Comunicazione} diretta con i partecipanti
\end{itemize}

Il progetto è stato sviluppato adottando un'architettura a
microservizi, garantendo scalabilità, manutenibilità e separazione
delle responsabilità.

\chapter{Requisiti}
\section{2~--- Requisiti}\label{requisiti}

Di seguito sono riportati i principali requisiti che l'applicazione deve
soddisfare.

\subsection{2.1~--- Requisiti di Business}\label{requisiti-di-business}

\begin{itemize}
\tightlist{}
\item
  La piattaforma consente alle organizzazioni di creare e pubblicare
  post relativi agli eventi da loro promossi.
\item
  Gli utenti possono utilizzare la piattaforma come punto di riferimento
  per scoprire eventi nelle vicinanze, in base alla località e ai propri
  interessi.
\item
  Il sistema abilita la vendita online dei biglietti degli eventi
  fornendo alle organizzazioni uno strumento per monetizzare le proprie
  attività.
\end{itemize}

\subsection{2.2~--- Requisiti Funzionali}\label{requisiti-funzionali}

\textbf{Tipologie di utenti supportate dal sistema:}

\begin{itemize}
\tightlist{}
\item
  Utenti non registrati.
\item
  Utenti registrati, che possono fruire dei contenuti della piattaforma.
\item
  Utenti registrati come organizzazioni, che possono creare eventi,
  vendere biglietti e fruire dei contenuti della piattaforma.
\end{itemize}

\textbf{Per tutti gli utenti:}

\begin{itemize}
\tightlist{}
\item
  Visualizzare la schermata Home con le modalità di interazione: ricerca
  eventi, visualizzazione eventi popolari, prossimi eventi e nuove
  aggiunte.
\item
  Visualizzare dalla schermata Esplora tutti gli eventi pubblicati sulla
  piattaforma, tutti gli utenti registrati e applicare filtri di
  ricerca.
\end{itemize}

\textbf{Per utenti registrati:}

\begin{itemize}
\tightlist{}
\item
  Ricevere un feed di eventi personalizzato, basato sugli interessi
  specificati.
\item
  Mettere e togliere like a un evento.
\item
  Mettere e togliere follow a un membro e a un'organizzazione.
\item
  Acquistare biglietti per gli eventi.
\item
  Lasciare una recensione dopo la partecipazione a un evento.
\item
  Contattare direttamente le organizzazioni all'interno della
  piattaforma per richiedere supporto.
\item
  Ricevere notifiche su:

  \begin{itemize}
  \tightlist{}
  \item
    nuovo follower.
  \item
    pubblicazione di nuovo evento da parte di organizzazione seguita.
  \item
    nuovo messaggio.
  \end{itemize}
\end{itemize}

\textbf{Per utenti registrati come organizzazioni:}

\begin{itemize}
\tightlist{}
\item
  Creare eventi, scegliendo se renderli pubblici o salvarli come bozza.
\item
  Specificare collaboratori durante la creazione degli eventi.
\item
  Ricevere notifiche su like e recensioni ai propri eventi.
\end{itemize}

\subsection{2.3~--- Requisiti Non
Funzionali}\label{requisiti-non-funzionali}

\begin{itemize}
\tightlist{}
\item
  Accessibilità: l'interfaccia grafica deve essere accessibile.
\item
  Portabilità: l'applicazione risulta responsive per adattarsi a schermi
  di diverse dimensioni pc/tablet/mobile.
\item
  Deployability: il sistema in automatico deve aggiornarsi alla versione
  dell'ultima release.
\item
  Availability: il sistema deve essere tollerante ai guasti per
  garantire la disponibilità, deve poter effettuare un recupero
  automatico in caso di errore e prevedere la ridondanza dei componenti
  critici per assicurare la continuità del servizio.
\item
  Sicurezza: gli utenti del sistema devono autenticarsi per verificare
  la loro identità e saranno poi autorizzati ad accedere alle risorse in
  base alle regole definite. Inoltre per assicurare la confidenzialità
  delle password queste saranno salvate in modo cifrato.
\item
  Robustezza: l'applicazione deve gestire input errati e generare errori
  coerenti.
\item
  Affidabilità: l'applicazione deve essere stabile, evitando crash.
\item
  Manutenibilità: il codice deve essere ben strutturato e ben
  documentato.
\item
  Estendibilità: il progetto deve favorire la personalizzazione e
  l'aggiunta di funzionalità.
\end{itemize}

Inoltre si è scelto, anche per esigenze didattiche, di aggiungere come
requisito architetturale lo sviluppo del sistema con un'architettura a
microservizi.


\chapter{Design}
\section{3~--- Design}\label{design}

\subsection{3.1~--- Metodologia
progettuale}\label{metodologia-progettuale}

Lo sviluppo della piattaforma è stato condotto seguendo l'approccio
\emph{User Centered Design} (UCD), con l'obiettivo di progettare un
sistema conforme ai principi HCI ottimizzando l'esperienza utente.

Per garantire la centralità dell'utente durante il design e lo sviluppo,
non potendo coinvolgere utenti reali per l'intero ciclo progettuale,
sono state adottate tecniche di virtualizzazione degli utenti promosse
da UCD, quali la metodologia \textbf{\emph{Personas}} combinata con gli
\textbf{scenari d'uso}.

\subsubsection{Analisi dei target user}\label{analisi-dei-target-user}

Dopo aver delineato i requisiti di business della piattaforma, il team
di sviluppo ha individuato il target di riferimento. La piattaforma è
progettata per tre tipologie di utenti: utenti non registrati, utenti
registrati e utenti registrati come organizzazione.

Per rappresentare le caratteristiche e i bisogni di ciascun gruppo di
utenti del sistema, sono state create delle \emph{Personas}. Per ogni
\emph{Personas} è stato inoltre simulato uno scenario d'uso della
piattaforma, in modo da evidenziare le diverse modalità di interazione
con il sistema e come questo possa rispondere efficacemente alle
esigenze degli utenti.

\emph{Personas}: Carlo

Carlo è un adulto appassionato di eventi culturali e spettacoli, ma
fatica a scoprire nuove attività tramite i canali di informazione
tradizionali. Non ama condividere i propri dati sul web e vuole trovare
rapidamente eventi interessanti a cui partecipare, da solo o con amici e
familiari.

Carlo ha bisogno di uno strumento semplice e immediato che gli permetta
di esplorare gli eventi in programma che lo interessano, capire
rapidamente orari, luoghi e dettagli principali, senza dover registrarsi
o inserire informazioni personali.

Scenario d'uso:

Carlo visita il sito della piattaforma senza registrarsi, scorre gli
eventi in programma e consulta foto, descrizioni e informazioni pratiche
come orario e luogo.

Grazie all'interfaccia chiara e semplice, Carlo può farsi un'idea
immediata di quali eventi potrebbero interessargli e pianificare
eventuali uscite. Pur senza account, ottiene tutte le informazioni
necessarie e apprezza poter scoprire eventi diversi rispetto a quelli
tradizionali, senza condividere dati personali.

\emph{Personas}: Francesca

Francesca ha 22 anni e si è appena trasferita in una città universitaria
vivace. Frequenta il primo anno della magistrale in Comunicazione
Digitale e Marketing, divide il suo tempo principalmente tra lezioni e
studio, e sente il desiderio di vivere esperienze che la aiutino a
integrarsi nella nuova città.

Usa costantemente lo smartphone per restare aggiornata su eventi e
tendenze locali. È attratta da contenuti visivi e immediati, come foto
con una breve descrizione, che le permettono di percepire rapidamente
l'atmosfera di un evento. Apre diverse app e social network alla ricerca
di eventi, scorrendo post e pagine locali. Questo processo la stanca: le
informazioni sono spesso sparse e incomplete e non sempre riesce a
trovare esperienze che le interessano.

Francesca cerca uno strumento che le permetta di scoprire eventi nella
nuova città in cui vive, in modo intuitivo e veloce.

Scenario d'uso:

Alla sera, terminato di studiare, Francesca accede alla piattaforma dal
suo smartphone ed esplora i prossimi eventi in programma filtrando
quelli vicino a lei.

Quando il post di un evento cattura la sua attenzione, controlla la
pagina dell'organizzazione per vedere gli eventi passati e leggere le
recensioni, così da capire il tipo di esperienze che l'organizzazione
propone e valutare se l'evento possa essere affidabile e di suo
gradimento. Essendo in una città nuova, queste informazioni la aiutano a
scegliere con maggiore sicurezza.

Grazie alla piattaforma, Francesca riesce a trovare rapidamente eventi
vicino a lei compatibili con i suoi interessi, ottenendo tutte le
informazioni necessarie in un unico luogo e acquistando i biglietti in
modo semplice e immediato come è abituata a fare in altre piattaforme.

\emph{Personas}: Emma Lopez

Emma ha 27 anni e vive a Madrid. Sta organizzando un weekend in Italia
con le sue amiche e vuole scoprire locali, eventi musicali autentici e
buoni ristoranti italiani. Pianifica con attenzione ogni dettaglio ed è
importante per lei trovare attività che soddisfino gli interessi di
tutte le sue amiche.

Per cercare eventi e locali, Emma usa principalmente il computer. Naviga
tra diversi siti, ma non ha un modo comodo e veloce per salvare e
confrontare gli eventi che le interessano. Spesso deve tradurre siti
italiani per capirne i dettagli. Si affida alle foto per farsi un'idea
dell'evento, ma non sa sempre come scegliere quali eventi siano davvero
interessanti. Vorrebbe prenotare qualcosa per assicurarsi un posto e
godersi la vacanza senza rischiare di perdere gli eventi più popolari.

Scenario d'uso:

Emma si registra sulla piattaforma che trova direttamente in lingua
spagnola, essendo in Spagna. Dal suo computer esplora gli eventi
filtrando in base alle sue preferenze e a quelle delle sue amiche, per
selezionare quelli più adatti al gruppo. Può salvare direttamente i pos
degli eventi che le interessano e vedere quanti like hanno ricevuto,
aiutandola a capire quali sono più popolari. Una volta convinta,
acquista i biglietti per lei e per le sue amiche, assicurandosi i posti.

Emma è contenta di poter consultare tutti gli eventi in programma in un
unico sito già tradotto nella propria lingua e sfruttare le funzionalità
della piattaforma per salvare, gestire, filtrare e prenotare i suoi
eventi.

\emph{Personas}: Simone

Simone ha 34 anni e gestisce un locale poco conosciuto in periferia.
Crede nelle potenzialità del suo locale: l'ambiente è bello e
accogliente, ma fatica a farlo conoscere. Il sito web del locale riceve
poche visite e non permette di capire quanti utenti siano interessati
agli eventi. Nel locale Simone organizza principalmente cene, ma gli
piacerebbe collaborare con organizzatori di spettacoli dal vivo o show
da ospitare nel suo locale per aumentare la visibilità e attrarre più
clienti.

Simone ha bisogno di uno strumento che gli permetta di promuovere il suo
locale, monitorare facilmente l'interesse degli utenti e collaborare con
altri organizzatori in modo semplice ed efficace.

Scenario d'uso:

Simone si registra come organizzazione sulla piattaforma e crea il
profilo del suo locale, rendendolo visibile a tutti gli utenti. Ogni
volta che riceve un nuovo follower o che un suo evento ottiene un like,
Simone riceve una notifica che gli fornisce un feedback immediato
sull'interesse degli utenti. La piattaforma gli consente inoltre di
collaborare con altre organizzazioni, definendo un calendario ricco di
eventi per i prossimi mesi presso il suo locale.

Grazie alla piattaforma, Simone vede finalmente il suo locale apprezzato
e frequentato.

Dall'analisi dei target user, attraverso le \emph{Personas} e i relativi
scenari d'uso, sono emersi i principali task che gli utenti desiderano
svolgere sulla piattaforma. Queste informazioni hanno guidato la
progettazione dell'interfaccia grafica.

\subsection{3.2~--- Progettazione dell'Interfaccia
Grafica}\label{progettazione-dellinterfaccia-grafica}

In questa sezione viene descritta la progettazione dell'interfaccia
grafica dell'applicazione.

L'obiettivo principale è stato quello di definire un design chiaro,
intuitivo e coerente, in grado di guidare l'utente attraverso i
principali flussi funzionali della piattaforma.

La sezione comprende:

\begin{itemize}
\tightlist{}
\item
  la presentazione dei \textbf{mockup} delle schermate principali;
\item
  una panoramica degli \textbf{storyboard}, illustrando i principali
  flussi di interazione dell'utente.
\end{itemize}

\subsubsection{Mockup}\label{mockup}

Per la fase iniziale di progettazione dell'interfaccia grafica sono
stati creati dei \textbf{mockup}, con l'obiettivo di definire una prima
proposta del possibile \textbf{look \& feel} dell'applicazione, prima di
passare all'implementazione vera e propria.

Seguendo un approccio \textbf{agile}, i mockup sono stati
successivamente sostituiti da \textbf{demo funzionanti incrementali},
permettendo di testare e validare progressivamente le funzionalità
dell'applicazione.

I mockup sono stati realizzati utilizzando
\href{https://www.figma.com/}{Figma}. In particolare, sono state
sviluppate le schermate delle \textbf{tre aree principali}
dell'applicazione:

\begin{itemize}
\tightlist{}
\item
  \textbf{Home}
\item
  \textbf{Esplora}
\item
  \textbf{Profilo}
\end{itemize}

\paragraph{Approccio Mobile-First}\label{approccio-mobile-first}

La progettazione è stata effettuata seguendo l'approccio
\textbf{mobile-first}, prendendo come modello di riferimento l'iPhone
16.

Questo approccio è stato scelto principalmente per due motivi:

\begin{enumerate}
\def\labelenumi{\arabic{enumi}.}
\tightlist{}
\item
  \textbf{Vincoli più restrittivi}: il formato mobile costringe a dare
  priorità ai contenuti più importanti e a semplificare la navigazione,
  promuovendo un'interfaccia chiara e intuitiva che rispetta il
  principio \textbf{KISS}.
\item
  \textbf{Target principale da smartphone}: la maggior parte degli
  utenti accederà all'app tramite telefono, quindi è stata data priorità
  all'ottimizzazione del design per questo genere di dispositivi.
\end{enumerate}

Nel design è stato inoltre seguito il principio \textbf{DRY,}
riutilizzando ove possibile gli stessi componenti per garantire un
aspetto coerente e familiare dell'applicazione.

Ad ogni modo il design è stato pensato e realizzato anche per essere
responsive ed avere successivamente una buona resa anche su schermi
desktop.

\paragraph{Home}\label{home}

Di seguito è riportato il mockup per la schermata home. Questa è la
schermata iniziale proposta all'utente, da cui potrà da subito cercare
degli eventi o semplicemente vedere gli eventi proposti in vetrina. Da
questa schermata l'utente potrà anche accedere o registrarsi alla
piattaforma.

\begin{figure}[H]
\centering
\includegraphics[width=0.3\textwidth]{/mockup/Mockup-iPhone_16-Home_1.jpg}
\hfill
\includegraphics[width=0.3\textwidth]{/mockup/Mockup-iPhone_16-Home_2.jpg}
\hfill
\includegraphics[width=0.3\textwidth]{/mockup/Mockup-iPhone_16-Home_3.jpg}
\end{figure}

Inizialmente era stata anche proposta una versione alternativa con un
diverso sistema di navigazione, che però è stata successivamente
scartata vista la scarsa integrazione con il design dell'applicazione,
in particolare in combinazione con la schermata \textbf{Esplora}.

\begin{figure}[H]
\centering
\includegraphics[width=0.3\textwidth]{/mockup/Mockup-iPhone_16-Home-Alternative.jpg}
\end{figure}

\paragraph{Esplora}\label{esplora}

Di seguito è riportato il design della sezione esplora. In questa
sezione è possibile andare a visualizzare tutti gli eventi presenti
sulla piattaforma e cercare anche tutti gli utenti per visualizzarne il
profilo, seguirli e contattare le organizzazioni.

\begin{figure}[H]
\centering
\includegraphics[width=0.3\textwidth]{/mockup/Mockup-iPhone_16-Explore.jpg}
\hfill
\includegraphics[width=0.3\textwidth]{/mockup/Mockup-iPhone_16-Explore_Events.jpg}
\hfill
\includegraphics[width=0.3\textwidth]{/mockup/Mockup-iPhone_16-Explore_Organizations.jpg}
\end{figure}

\paragraph{Profilo}\label{profilo}

Da questa schermata l'utente avrà accesso alle sue informazioni, potrà
modificare il suo profilo e le sue preferenze.

\begin{figure}[H]
\centering
\includegraphics[width=0.3\textwidth]{/mockup/Mockup-iPhone_16-User_Profile.jpg}
\end{figure}

\subsubsection{Storyboard}\label{storyboard}

Di seguito sono riportati alcuni esempi di interazione con
l'applicazione, che illustrano i principali flussi di utilizzo.

In particolare, vengono mostrati i seguenti casi d'uso:

\begin{itemize}
\tightlist{}
\item
  \textbf{Esplorare la piattaforma}
\item
  \textbf{Login e Registrazione}
\item
  \textbf{Creare un evento}
\item
  \textbf{Partecipare ad un evento}
\item
  \textbf{Recensire un evento}
\item
  \textbf{Contattare un'organizzazione}
\end{itemize}

Le storyboard sono presentate direttamente utilizzando la piattaforma
sviluppata, mostrando le principali interazioni dell'utente.

Nel progettare i flussi di navigazione si è sempre tenuto conto della
\textbf{regola dei tre click}, cercando di rendere le principali
funzionalità accessibili in pochi passaggi. Negli esempi riportati di
seguito, alcune dinamiche di navigazione secondarie o ripetitive sono
state omesse.

\subsubsection{Esplorare la piattaforma}\label{esplorare-la-piattaforma}

L'utente che apre la piattaforma può iniziare ad esplorarla, in
particolare può scoprire gli eventi proposti o cercarli direttamente
tramite la barra di ricerca. Nel caso in cui voglia visualizzare
maggiori risultati può andare in una pagina dedicata all'esplorazione
dei contenuti della piattaforma dove è possibile filtrare gli eventi e
cercare in maniera più comoda utenti e organizzazioni.

\begin{figure}[H]
\centering
\pandocbounded{\includegraphics[keepaspectratio,alt={Storyboard Esplora}]{/storyboard/Storyboard-Explore.png}}
\caption{Storyboard Esplora}
\end{figure}

\subsubsection{Login e Registrazione}\label{login-e-registrazione}

Dopo aver aperto l'applicazione, l'utente per accedere alle funzionalità
aggiuntive che la piattaforma offre può accedere o registrarsi.

\begin{figure}[H]
\centering
\pandocbounded{\includegraphics[keepaspectratio,alt={Storyboard Login}]{/storyboard/Storyboard-Login.png}}
\caption{Storyboard Login}
\end{figure}

\subsubsection{Creare un evento}\label{creare-un-evento}

Un'organizzazione che si è registrata sulla piattaforma ha la
possibilità di creare degli eventi, la creazione avviene attraverso un
form in cui inserire tutti i vari dati. Inoltre è possibile anche
temporaneamente creare una bozza dell'evento e continuare a modificarla
successivamente.

\begin{figure}[H]
\centering
\pandocbounded{\includegraphics[keepaspectratio,alt={Storyboard Crea Evento}]{/storyboard/Storyboard-Create-Event.png}}
\caption{Storyboard Crea Evento}
\end{figure}

\subsubsection{Partecipare ad un evento}\label{partecipare-ad-un-evento}

Un utente registrato sulla piattaforma ha la possibilità di acquistare i
biglietti per i diversi eventi e visualizzarli in seguito, i biglietti
conterranno un QR code che può essere usato dalle organizzazioni per
verificarli.

\begin{figure}[H]
\centering
\pandocbounded{\includegraphics[keepaspectratio,alt={Storyboard Compra Biglietto}]{/storyboard/Storyboard-Buy-Ticket.png}}
\caption{Storyboard Compra Biglietto}
\end{figure}

\subsubsection{Recensire un evento}\label{recensire-un-evento}

In seguito alla partecipazione ad un evento, un utente può decidere di
lasciare una sua recensione. Una volta lasciata non ne può lasciare
altre per lo stesso evento ma può modificarla o eliminarla.

\begin{figure}[H]
\centering
\pandocbounded{\includegraphics[keepaspectratio,alt={Storyboard Recensione}]{/storyboard/Storyboard-Reviews.png}}
\caption{Storyboard Recensione}
\end{figure}

\subsubsection{Contattare
un'organizzazione}\label{contattare-unorganizzazione}

Un utente registrato può avere la necessità di contattare
un'organizzazione per chiedere maggiori informazioni o per eventuali
problemi.

\begin{figure}[H]
\centering
\pandocbounded{\includegraphics[keepaspectratio,alt={Storyboard Chat}]{/storyboard/Storyboard-Chat.png}}
\caption{Storyboard Chat}
\end{figure}

\subsection{3.3~--- Dominio}\label{dominio}

Durante la fase iniziale di analisi sono state individuate le entità
significative del dominio.

Il sistema si basa su due tipologie di utenti, membri e organizzazioni,
che condividono una base comune ma possiedono funzionalità differenti.

I membri rappresentano gli utenti finali della piattaforma e
interagiscono con gli eventi pubblicati dalle organizzazioni. Possono
esplorare gli eventi applicando filtri quali data, città, prezzo,
esprimere interesse tramite like e partecipare agli eventi acquistandone
i relativi biglietti. Dopo aver partecipato all'evento, gli utenti
possono lasciare recensioni, contribuendo alla valutazione e alla
popolarità dell'organizzazione.

Inoltre, possono seguire altri utenti e comunicare direttamente con le
organizzazioni tramite messaggi.

Le organizzazioni, oltre alle funzionalità comuni ai membri, hanno la
possibilità di creare e gestire eventi, definendone le informazioni e
rendendoli disponibili sulla piattaforma. Possono inoltre ricevere
recensioni da utenti che hanno partecipato ai loro eventi.

Tutti gli utenti ricevono delle notifiche a seguito di eventi di dominio
che li riguardano.

Da questa analisi emergono le principali entità del dominio, tra cui
utenti, eventi, biglietti, notifiche e interazioni, che costituiscono la
base per il design delle API e dell'architettura del sistema.

\subsection{3.4~--- Design delle risorse}\label{design-delle-risorse}

A partire dall'analisi del dominio, sono state definite come principali
risorse gli utenti e gli eventi.

Oltre alle risorse principali, il sistema modella ulteriori risorse
rilevanti, come:

\begin{itemize}
\tightlist{}
\item
  \textbf{tickets}, associati agli eventi
\item
  \textbf{conversations}, per la comunicazione tra utenti
\item
  \textbf{interactions}, che aggregano like, review e follow
\item
  \textbf{notifications,} associate agli utenti
\end{itemize}

Questo approccio permette di mantenere un modello REST coerente e
facilmente estendibile.

\subsubsection{/users}\label{users}

La risorsa /users rappresenta gli utenti del sistema ed è modellata
secondo l'archetipo \textbf{collection}, mentre /users/\{userId\}
rappresenta una risorsa \textbf{document}.

Ad ogni utente sono associate anche ulteriori \textbf{sotto-risorse}
(altre \textbf{collection}), tra cui:

\begin{itemize}
\tightlist{}
\item
  /users/\{userId\}/likes
\item
  /users/\{userId\}/reviews
\item
  /users/\{userId\}/followers
\item
  /users/\{userId\}/following
\item
  /users/\{userId\}/events
\item
  /users/\{userId\}/conversations
\item
  /users/\{userId\}/notifications
\end{itemize}

Alcuni endpoint seguono l'archetipo del \textbf{controller} in quanto
escono dal classico paradigma RESTfull indicando delle azioni, ad
esempio:~--- /users/login~--- /users/register

\subsubsection{/events}\label{events}

La risorsa /events rappresenta l'insieme degli eventi disponibili nel
sistema (archetipo \textbf{collection}), mentre /events/\{eventId\}
rappresenta un singolo evento (\textbf{document}).

Ad ogni evento sono associate anche ulteriori \textbf{sotto-risorse}
(altre \textbf{collection}), tra cui:

\begin{itemize}
\tightlist{}
\item
  /events/\{eventId\}/participants
\item
  /events/\{eventId\}/likes
\item
  /events/\{eventId\}/reviews
\item
  /events/\{eventId\}/tickets
\end{itemize}

Su alcuni endpoint vengono utilizzati dei query params per filtrare i
dati delle collections ed effettuare una richiesta paginata tramite
limit e offset, ad esempio:

\begin{itemize}
\tightlist{}
\item
  /events/search?query=sunset\&limit=10\&offset=0\&orderBy=date
\end{itemize}

\subsection{3.5~--- Design delle API}\label{design-delle-api}

Le API sono state progettate seguendo i principi del REST API Design,
con particolare attenzione all'uso corretto dei metodi HTTP,
all'utilizzo di sostantivi negli url e al rispetto degli archetipi REST\@.

Alcuni esempi di API implementate seguendo i principi citati:

\begin{figure}[H]
\centering
\pandocbounded{\includegraphics[keepaspectratio,alt={Alcune API per la collection Users}]{/api/user-api.png}}
\caption{Alcune API per la collection Users}
\end{figure}

\begin{figure}[H]
\centering
\pandocbounded{\includegraphics[keepaspectratio,alt={Alcune API per la collection Events}]{/api/events-api.png}}
\caption{Alcune API per la collection Events}
\end{figure}

\begin{figure}[H]
\centering
\pandocbounded{\includegraphics[keepaspectratio,alt={Alcune API per la collection Ticket-Types}]{/api/ticket_types-api.png}}
\caption{Alcune API per la collection Ticket-Types}
\end{figure}

Ogni microservizio espone una documentazione \textbf{Swagger/OpenAPI},
che descrive endpoint, input e possibili risposte HTTP\@.

Swagger è stato utilizzato sia come strumento di documentazione che come
supporto al testing manuale delle API\@. Di seguito il link
\url{https://eventonight.github.io/EvenToNight/openAPI/}

\subsection{3.6~--- Architettura}\label{architettura}

L'architettura del sistema è basata su un insieme di microservizi
indipendenti, ciascuno responsabile di un sotto-dominio applicativo
specifico.

Il frontend rappresenta il punto di accesso per gli utenti e comunica
esclusivamente con i servizi backend tramite API REST e socket, senza
accesso diretto ai database.

Ogni servizio infatti possiede la propria logica di business, entità
comuni a più servizi (es. gli eventi) vengono rappresentati in maniera
diversa in ognuno di essi.

Ogni servizio è containerizzato tramite Docker, comunica con gli altri
servizi principalmente tramite messaggi asincroni ed espone un insieme
coerente di API REST\@.

\begin{figure}[H]
\centering
\pandocbounded{\includegraphics[keepaspectratio,alt={Panoramica dell' architectura}]{/architecture/architecture-overview.png}}
\caption{Panoramica dell' architectura}
\end{figure}

Le risorse individuate nella fase precedente sono state organizzate e
distribuite nei vari servizi, di seguito una breve descizione.

\begin{figure}[H]
\centering
\pandocbounded{\includegraphics[keepaspectratio,alt={Archiettura del servizio Users}]{/architecture/users-service.png}}
\caption{Archiettura del servizio Users}
\end{figure}

Il servizio \emph{users} è responsabile della gestione delle risorse
utente e di una parte delle relative sotto-risorse. Per ogni utente, il
sistema gestisce l'auenticazione e le informazioni riguardanti l'accoun
(e.g username, email, interessi) e il profilo (e.g.~nome, bio).

\begin{figure}[H]
\centering
\pandocbounded{\includegraphics[keepaspectratio,alt={Archiettura del servizio Events}]{/architecture/events-service.png}}
\caption{Archiettura del servizio Events}
\end{figure}

Il servizio \emph{events} è responsabile della gestione delle risorse
eventi e delle loro informazioni. Gestisce sia gli aspetti di creazione
degli eventi sia il recupero degli eventi filtrati in base a
caratteristiche specificate (e.g.popolari, interessi).

\begin{figure}[H]
\centering
\pandocbounded{\includegraphics[keepaspectratio,alt={Archiettura del servizio Interactions}]{/architecture/interactions-service.png}}
\caption{Archiettura del servizio Interactions}
\end{figure}

Il servizio \emph{interactions} è responsabile di alcune sotto-risorse
degli utenti e degli eventi. In particolare, gestisce tutte le
informazioni riguardanti le interazioni che un utente può avere con un
evento o con un altro utente. Tra le interazioni con gli eventi sono
stati implementati i likes, le reviews e le partecipazioni, mentre con
gli altri utenti è stato implementato un sistema di following.

\begin{figure}[H]
\centering
\pandocbounded{\includegraphics[keepaspectratio,alt={Archiettura del servizio Chat}]{/architecture/chat-service.png}}
\caption{Archiettura del servizio Chat}
\end{figure}

Il servizio \emph{chat} è responsabile di alcune sotto-risorse degli
utenti. In particolare, gestisce tutte le informazioni riguardanti le
conversazioni che l'utente può avere con altri utenti.

\begin{figure}[H]
\centering
\pandocbounded{\includegraphics[keepaspectratio,alt={Archiettura del servizio Payments}]{/architecture/payments-service.png}}
\caption{Archiettura del servizio Payments}
\end{figure}

Il servizio payments è responsabile di alcune risorse degli utenti e
degli eventi, in particolare della gestione dei biglietti che gli utenti
possono acquistare per partecipare agli eventi.

\begin{figure}[H]
\centering
\pandocbounded{\includegraphics[keepaspectratio,alt={Archiettura del servizio Notifications}]{/architecture/notifications-service.png}}
\caption{Archiettura del servizio Notifications}
\end{figure}

Il servizio \emph{notifications} è responsabile di alcune sotto-risorse
degli utenti, in particolare gestisce tutte le informazioni riguardanti
le notifiche che un utente riceve. Le notifiche possono essere di tipi
diversi, quali like ricevuto, review ricevuta, nuovo follower o nuovo
evento creato da un'organizzazione seguita. Si occupa anche di
notificare l'arrivo di un nuovo messaggio, gestito poi dal servizio
chat.

\subsection{3.7~--- Comunicazione: RabbitMQ e
Socket.IO}\label{comunicazione-rabbitmq-e-socket.io}

Per la comunicazione tra servizi abbiamo utilizzato un broker di
messaggi, in particolare RabbitMQ\@. Ogni servizio definisce alcuni
\emph{domain event} \emph{(e.g.~nuovo evento pubblicato, like aggiunto,
utente creato)} e li comunica sull'exchange \emph{eventonight}, al quale
sono connesse e in ascolto le code dei vari servizi.

Per funzionalità che richiedono aggiornamento in tempo reale (come le
notifiche e le chat) abbiamo utilizzato una comunicazione basata su
socket centralizzandola nel servizio notifications in modo da aprire una
sola connessione con ciascun client. Il servizio notifications tramite
RabbitMQ ascolta tutti i \emph{domain event} che vogliamo comunicare
real-time, li elabora e quando necessario li inoltra nei canali degli
utenti connessi interessati.


\chapter{Tecnologie}
\section{4~--- Tecnologie}\label{tecnologie}

Durante lo sviluppo del progetto sono state utilizzate molteplici
tecnologie, sia per esigenze implementative sia per scopi di
apprendimento.

Per realizzare il frontend è stato utilizzato il framework Vue.js.

Per la realizzazione del backend invece, sono stati utilizzati diversi
framework. Oltre allo stack MEVN infatti abbiamo utilizzato NestJS per
la gestione di alcuni servizi, Cask per implementare le API in Scala,
\href{http://Socket.IO}{Socket.IO} per la comunicazione server-client,
RabbitMQ per la comunicazione tra servizi.

Per la gestione dei pagamenti è stato utilizzato Stripe (in modalità
SandBox), per il salvataggio delle immagini è stato utilizzato MinIO\@.

Per il deploy è stato utilizzato Docker, e tramite i tunnel di
CloudFlare è stato esposto in rete.

\textbf{Tecnologie:}

\begin{itemize}
\tightlist%
\item
  Vue.js
\item
  MongoDB
\item
  Express.js
\item
  Node.js
\item
  Socket.IO
\item
  Gradle
\item
  Docker
\item
  RabbitMQ
\item
  Traefik
\item
  NestJS
\item
  MinIO
\item
  Cask
\item
  Swagger~--- OpenAPI
\end{itemize}

\textbf{Servizi esterni}

\begin{itemize}
\tightlist%
\item
  Keycloak
\item
  Stripe
\item
  CloudFlare
\end{itemize}

\textbf{Linguaggi}

\begin{itemize}
\tightlist%
\item
  Typescrip
\item
  Scala
\end{itemize}


\chapter{Codice}
\section{5~--- Codice}\label{codice}

L'architettura del progetto \textbf{EvenToNight} si basa su una
\textbf{Single Page Application (SPA)} per il frontend e un layer di
\textbf{microservizi backend} esposti attraverso \textbf{Traefik} come
reverse-proxy/API gateway.

\subsection{5.1~--- Frontend}\label{frontend}

Il frontend è stato sviluppato con \textbf{Vue 3} utilizzando la
\textbf{Composition API} e \textbf{TypeScript}, sfruttando il framework
\textbf{Quasar} per i componenti UI\@.

L'applicazione fa uso delle principali funzionalità di Vue come
\texttt{provide/\allowbreak{} inject}, \texttt{props/\allowbreak{} emit}, \texttt{defineModel},
\texttt{defineExpose}, \texttt{watchers} e in alcune situazioni la
direttiva \texttt{:key} per innescare il refresh dei componenti.

\subsubsection{Struttura del Progetto}\label{struttura-del-progetto}

\begin{Shaded}
\begin{Highlighting}[]
\NormalTok{/src/}
\NormalTok{├── api/                 \# Layer di astrazione API con supporto mock}
\NormalTok{│   ├── adapters/        \# Adapter per allinare i dati ricevuti dalle API}
\NormalTok{│   ├── mock{-}services/   \# Implementazioni mock per sviluppo}
\NormalTok{│   ├── services/        \# Implementazioni API dei servizi}
\NormalTok{│   └── client.ts        \# Client HTTP con gestione JWT}
\NormalTok{├── components/          \# Componenti Vue riutilizzabili}
\NormalTok{├── composables/         \# Funzionalità riutilizzabili dai vari componenti}
\NormalTok{├── i18n/                \# File di lingua}
\NormalTok{├── layouts/             \# Layout condivisi}
\NormalTok{├── router/              \# Configurazione routing e guards}
\NormalTok{├── stores/              \# Pinia store}
\NormalTok{└── views/               \# Pagine dell\textquotesingle{}applicazione}
\end{Highlighting}
\end{Shaded}

\subsubsection{Gestione dello Stato con
Pinia}\label{gestione-dello-stato-con-pinia}

Lo store Pinia gestisce l'autenticazione dell'utente e i token JWT\@. Il
sistema implementa il refresh automatico dei token prima della scadenza:

\begin{Shaded}
\begin{Highlighting}[]
\KeywordTok{interface}\NormalTok{ Tokens \{}
\NormalTok{  accesToken}\OperatorTok{:}\NormalTok{ AccessToken}\OperatorTok{,}
\NormalTok{  refreshToken}\OperatorTok{:}\NormalTok{ RefreshToken}\OperatorTok{,}
\NormalTok{  refreshExpiresAt}\OperatorTok{:} \DataTypeTok{number}\OperatorTok{,}
\NormalTok{\}}

\ImportTok{export} \KeywordTok{const}\NormalTok{ useAuthStore }\OperatorTok{=} \FunctionTok{defineStore}\NormalTok{(}\StringTok{\textquotesingle{}auth\textquotesingle{}}\OperatorTok{,}\NormalTok{ () }\KeywordTok{=\textgreater{}}\NormalTok{ \{}
  \KeywordTok{const}\NormalTok{ user }\OperatorTok{=} \FunctionTok{ref}\OperatorTok{\textless{}}\NormalTok{User }\OperatorTok{|} \DataTypeTok{null}\OperatorTok{\textgreater{}}\NormalTok{(}\KeywordTok{null}\NormalTok{)}
  \KeywordTok{const}\NormalTok{ tokens }\OperatorTok{=}\NormalTok{ ref}\OperatorTok{\textless{}}\NormalTok{Tokens }\OperatorTok{|} \DataTypeTok{null}\OperatorTok{\textgreater{}}\NormalTok{ (}\KeywordTok{null}\NormalTok{)}

  \KeywordTok{const}\NormalTok{ isAuthenticated }\OperatorTok{=} \FunctionTok{computed}\NormalTok{(() }\KeywordTok{=\textgreater{}}\NormalTok{ \{}
    \ControlFlowTok{if}\NormalTok{ (}\OperatorTok{!}\NormalTok{tokens}\OperatorTok{.}\AttributeTok{value} \OperatorTok{||} \OperatorTok{!}\NormalTok{user}\OperatorTok{.}\AttributeTok{value}\NormalTok{) }\ControlFlowTok{return} \KeywordTok{false}
    \ControlFlowTok{return}\NormalTok{ tokens}\OperatorTok{.}\AttributeTok{refreshExpiresAt}\OperatorTok{.}\AttributeTok{value} \OperatorTok{\textgreater{}} \BuiltInTok{Date}\OperatorTok{.}\FunctionTok{now}\NormalTok{()}
\NormalTok{  \})}

  \CommentTok{// Set data to local storage to avoid data loss on page refresh}
  \KeywordTok{const}\NormalTok{ setAuthData }\OperatorTok{=} \KeywordTok{async}\NormalTok{ (authData}\OperatorTok{:}\NormalTok{ LoginResponse) }\KeywordTok{=\textgreater{}}\NormalTok{ \{}
    \FunctionTok{setTokens}\NormalTok{(authData)}
    \FunctionTok{setUser}\NormalTok{(authData}\OperatorTok{.}\AttributeTok{user}\NormalTok{)}
    \FunctionTok{setupAutoRefresh}\NormalTok{()}
    \ControlFlowTok{await}\NormalTok{ api}\OperatorTok{.}\AttributeTok{notifications}\OperatorTok{.}\FunctionTok{connect}\NormalTok{(authData}\OperatorTok{.}\AttributeTok{user}\OperatorTok{.}\AttributeTok{id}\OperatorTok{,}\NormalTok{ authData}\OperatorTok{.}\AttributeTok{accessToken}\NormalTok{) }\CommentTok{//Connect to Socket}
\NormalTok{  \}}

  \CommentTok{// Restores auth data from session storage (if any)}
  \KeywordTok{const}\NormalTok{ refreshCurrentSessionUserData }\OperatorTok{=}\NormalTok{ () }\KeywordTok{=\textgreater{}}\NormalTok{ \{\}}

  \CommentTok{// Auto{-}refresh 5 minutes before expiration}
  \KeywordTok{const}\NormalTok{ setupAutoRefresh }\OperatorTok{=}\NormalTok{ () }\KeywordTok{=\textgreater{}}\NormalTok{ \{}
    \KeywordTok{const}\NormalTok{ refreshTime }\OperatorTok{=}\NormalTok{ tokens}\OperatorTok{.}\AttributeTok{value}\OperatorTok{.}\AttributeTok{refreshExpiresAt} \OperatorTok{{-}} \BuiltInTok{Date}\OperatorTok{.}\FunctionTok{now}\NormalTok{() }\OperatorTok{{-}} \DecValTok{5} \OperatorTok{*} \DecValTok{60} \OperatorTok{*} \DecValTok{1000}
    \ControlFlowTok{if}\NormalTok{ (refreshTime }\OperatorTok{\textgreater{}} \DecValTok{0}\NormalTok{) \{}
      \PreprocessorTok{setTimeout}\NormalTok{(() }\KeywordTok{=\textgreater{}} \FunctionTok{refreshAccessToken}\NormalTok{()}\OperatorTok{,}\NormalTok{ refreshTime)}
\NormalTok{    \}}
\NormalTok{  \}}
\NormalTok{\})}
\end{Highlighting}
\end{Shaded}

\subsubsection{Layer API e Mocking
Strategy}\label{layer-api-e-mocking-strategy}

Un aspetto fondamentale dello sviluppo è stato il \textbf{layer di
astrazione API}, che ha permesso di prototipare il frontend
indipendentemente dalla disponibilità dei microservizi backend.

Il sistema utilizza una variabile d'ambiente per utilizzare API reali o
mock:

\begin{Shaded}
\begin{Highlighting}[]
\KeywordTok{const}\NormalTok{ useRealApi}\OperatorTok{:} \DataTypeTok{boolean} \OperatorTok{=}\NormalTok{ import}\OperatorTok{.}\AttributeTok{meta}\OperatorTok{.}\AttributeTok{env}\OperatorTok{.}\AttributeTok{VITE\_USE\_MOCK\_API} \OperatorTok{===} \StringTok{\textquotesingle{}false\textquotesingle{}}

\ImportTok{export} \KeywordTok{const}\NormalTok{ api }\OperatorTok{=}\NormalTok{ \{}
\NormalTok{  events}\OperatorTok{:}\NormalTok{ useRealApi }\OperatorTok{?} \FunctionTok{createEventsApi}\NormalTok{(}\FunctionTok{createEventsClient}\NormalTok{()) }\OperatorTok{:}\NormalTok{ mockEventsApi}\OperatorTok{,}
\NormalTok{  chat}\OperatorTok{:}\NormalTok{ useRealApi }\OperatorTok{?} \FunctionTok{createChatApi}\NormalTok{(}\FunctionTok{createChatClient}\NormalTok{()) }\OperatorTok{:}\NormalTok{ mockChatApi}\OperatorTok{,}
\NormalTok{  notifications}\OperatorTok{:}\NormalTok{ useRealApi}
    \OperatorTok{?} \FunctionTok{createNotificationsApi}\NormalTok{(}\FunctionTok{createNotificationsClient}\NormalTok{())}
    \OperatorTok{:}\NormalTok{ mockNotificationsApi}\OperatorTok{,}
  \CommentTok{// other services}
\NormalTok{\}}
\end{Highlighting}
\end{Shaded}

Il client HTTP gestisce centralmente l'injection del token JWT e il refresh automatico in caso di 401:

\begin{Shaded}
\begin{Highlighting}[]
\KeywordTok{const}\NormalTok{ token }\OperatorTok{=}\NormalTok{ tokenProvider}\OperatorTok{?.}\NormalTok{()}
\ControlFlowTok{if}\NormalTok{ (token) \{}
\NormalTok{  headers[}\StringTok{\textquotesingle{}Authorization\textquotesingle{}}\NormalTok{] }\OperatorTok{=} \VerbatimStringTok{\textasciigrave{}Bearer }\SpecialCharTok{$\{}\NormalTok{token}\SpecialCharTok{\}}\VerbatimStringTok{\textasciigrave{}}
\NormalTok{\}}

\CommentTok{// Refresh after receiving 401}
\ControlFlowTok{if}\NormalTok{ (response}\OperatorTok{.}\AttributeTok{status} \OperatorTok{===} \DecValTok{401} \OperatorTok{\&\&} \OperatorTok{!}\NormalTok{isRetry }\OperatorTok{\&\&}\NormalTok{ onTokenExpired) \{}
  \KeywordTok{const}\NormalTok{ refreshed }\OperatorTok{=} \ControlFlowTok{await} \FunctionTok{onTokenExpired}\NormalTok{()}
  \ControlFlowTok{if}\NormalTok{ (refreshed) \{}
    \ControlFlowTok{return} \KeywordTok{this}\OperatorTok{.}\FunctionTok{request}\NormalTok{(endpoint}\OperatorTok{,}\NormalTok{ options}\OperatorTok{,} \KeywordTok{true}\NormalTok{)}
\NormalTok{  \}}
\NormalTok{\}}
\end{Highlighting}
\end{Shaded}

Questo approccio, seguendo il principio \textbf{Dependency Inversion
(DIP)}, ha permesso di sviluppare componenti UI indipendenti
dall'implementazione concreta delle API e ha facilitato il testing.

\subsubsection{Router e Navigation
Guards}\label{router-e-navigation-guards}

Il sistema di routing utilizza due livelli: un livello root per gestire
redirect e rotte speciali, e un livello nested sotto \texttt{/:locale}
per tutte le rotte localizzate. Questa separazione è stata necessaria
perché alcune rotte (come quella codificata nel QR code nei biglietti)
non richiedono il prefisso della lingua.

\begin{Shaded}
\begin{Highlighting}[]

\KeywordTok{const}\NormalTok{ router }\OperatorTok{=} \FunctionTok{createRouter}\NormalTok{(\{}
\NormalTok{  history}\OperatorTok{:} \FunctionTok{createWebHistory}\NormalTok{(import}\OperatorTok{.}\AttributeTok{meta}\OperatorTok{.}\AttributeTok{env}\OperatorTok{.}\AttributeTok{BASE\_URL}\NormalTok{)}\OperatorTok{,}
\NormalTok{  routes}\OperatorTok{:}\NormalTok{ [}
\NormalTok{    \{}
\NormalTok{      path}\OperatorTok{:} \StringTok{\textquotesingle{}/\textquotesingle{}}\OperatorTok{,}
\NormalTok{      redirect}\OperatorTok{:}\NormalTok{ () }\KeywordTok{=\textgreater{}} \VerbatimStringTok{\textasciigrave{}/}\SpecialCharTok{$\{}\FunctionTok{getInitialLocale}\NormalTok{()}\SpecialCharTok{\}}\VerbatimStringTok{\textasciigrave{}}\OperatorTok{,}
\NormalTok{    \}}\OperatorTok{,}
\NormalTok{    \{}
\NormalTok{      path}\OperatorTok{:} \StringTok{\textquotesingle{}/verify/:ticketId\textquotesingle{}}\OperatorTok{,}
\NormalTok{      redirect}\OperatorTok{:}\NormalTok{ (to) }\KeywordTok{=\textgreater{}} \VerbatimStringTok{\textasciigrave{}/}\SpecialCharTok{$\{}\FunctionTok{getInitialLocale}\NormalTok{()}\SpecialCharTok{\}}\VerbatimStringTok{/verify/}\SpecialCharTok{$\{}\NormalTok{to}\OperatorTok{.}\AttributeTok{params}\OperatorTok{.}\AttributeTok{ticketId}\SpecialCharTok{\}}\VerbatimStringTok{\textasciigrave{}}\OperatorTok{,}
\NormalTok{    \}}\OperatorTok{,}
\NormalTok{    \{}
\NormalTok{      path}\OperatorTok{:} \StringTok{\textquotesingle{}/:locale\textquotesingle{}}\OperatorTok{,}
\NormalTok{      component}\OperatorTok{:}\NormalTok{ LocaleWrapper}\OperatorTok{,}
\NormalTok{      children}\OperatorTok{:}\NormalTok{ [}
\NormalTok{        \{ path}\OperatorTok{:} \StringTok{\textquotesingle{}\textquotesingle{}}\OperatorTok{,}\NormalTok{ name}\OperatorTok{:} \StringTok{\textquotesingle{}home\textquotesingle{}}\OperatorTok{,}\NormalTok{ component}\OperatorTok{:}\NormalTok{ Home \}}\OperatorTok{,}
\NormalTok{        \{ path}\OperatorTok{:} \StringTok{\textquotesingle{}login\textquotesingle{}}\OperatorTok{,}\NormalTok{ name}\OperatorTok{:} \StringTok{\textquotesingle{}login\textquotesingle{}}\OperatorTok{,}\NormalTok{ component}\OperatorTok{:}\NormalTok{ () }\KeywordTok{=\textgreater{}} \ImportTok{import}\NormalTok{(}\StringTok{\textquotesingle{}../views/AuthView.vue\textquotesingle{}}\NormalTok{)}\OperatorTok{,}\NormalTok{ beforeEnter}\OperatorTok{:}\NormalTok{ requireGuest \}}\OperatorTok{,}
\NormalTok{        \{ path}\OperatorTok{:} \StringTok{\textquotesingle{}events/:id\textquotesingle{}}\OperatorTok{,}\NormalTok{ name}\OperatorTok{:} \StringTok{\textquotesingle{}event{-}details\textquotesingle{}}\OperatorTok{,}\NormalTok{ component}\OperatorTok{:}\NormalTok{ () }\KeywordTok{=\textgreater{}} \ImportTok{import}\NormalTok{(}\StringTok{\textquotesingle{}../views/EventDetailsView.vue\textquotesingle{}}\NormalTok{)}\OperatorTok{,}\NormalTok{ beforeEnter}\OperatorTok{:}\NormalTok{ requireNotDraft \}}\OperatorTok{,}
\NormalTok{        \{ path}\OperatorTok{:} \StringTok{\textquotesingle{}create{-}event\textquotesingle{}}\OperatorTok{,}\NormalTok{ name}\OperatorTok{:} \StringTok{\textquotesingle{}create{-}event\textquotesingle{}}\OperatorTok{,}\NormalTok{ component}\OperatorTok{:}\NormalTok{ () }\KeywordTok{=\textgreater{}} \ImportTok{import}\NormalTok{(}\StringTok{\textquotesingle{}../views/CreateEventView.vue\textquotesingle{}}\NormalTok{)}\OperatorTok{,}\NormalTok{ beforeEnter}\OperatorTok{:} \FunctionTok{requireRole}\NormalTok{(}\StringTok{\textquotesingle{}organization\textquotesingle{}}\NormalTok{) \}}\OperatorTok{,}
        \CommentTok{// ... other routes}
\NormalTok{      ]}\OperatorTok{,}
\NormalTok{    \}}\OperatorTok{,}
\NormalTok{  ]}\OperatorTok{,}
\NormalTok{\})}

\CommentTok{// Global guard for i18n synchronization}
\NormalTok{router}\OperatorTok{.}\FunctionTok{beforeEach}\NormalTok{((to}\OperatorTok{,}\NormalTok{ \_from}\OperatorTok{,}\NormalTok{ next) }\KeywordTok{=\textgreater{}}\NormalTok{ \{}
  \KeywordTok{const}\NormalTok{ locale }\OperatorTok{=}\NormalTok{ to}\OperatorTok{.}\AttributeTok{params}\OperatorTok{.}\AttributeTok{locale} \ImportTok{as} \DataTypeTok{string}

  \CommentTok{// Redirect if locale is not supported}
  \ControlFlowTok{if}\NormalTok{ (locale }\OperatorTok{\&\&} \OperatorTok{!}\NormalTok{SUPPORTED\_LOCALES}\OperatorTok{.}\FunctionTok{includes}\NormalTok{(locale)) \{}
    \ControlFlowTok{return} \FunctionTok{next}\NormalTok{(}\VerbatimStringTok{\textasciigrave{}/}\SpecialCharTok{$\{}\NormalTok{DEFAULT\_LOCALE}\SpecialCharTok{\}$\{}\NormalTok{to}\OperatorTok{.}\AttributeTok{path}\OperatorTok{.}\FunctionTok{substring}\NormalTok{(locale}\OperatorTok{.}\AttributeTok{length} \OperatorTok{+} \DecValTok{1}\NormalTok{)}\SpecialCharTok{\}}\VerbatimStringTok{\textasciigrave{}}\NormalTok{)}
\NormalTok{  \}}

  \CommentTok{// Restore saved locale preference on navigation}
  \KeywordTok{const}\NormalTok{ savedLocale }\OperatorTok{=}\NormalTok{ localStorage}\OperatorTok{.}\FunctionTok{getItem}\NormalTok{(}\StringTok{\textquotesingle{}user{-}locale\textquotesingle{}}\NormalTok{)}
  \ControlFlowTok{if}\NormalTok{ (savedLocale }\OperatorTok{\&\&}\NormalTok{ locale }\OperatorTok{\&\&}\NormalTok{ locale }\OperatorTok{!==}\NormalTok{ savedLocale }\OperatorTok{\&\&}\NormalTok{ SUPPORTED\_LOCALES}\OperatorTok{.}\FunctionTok{includes}\NormalTok{(savedLocale)) \{}
    \ControlFlowTok{return} \FunctionTok{next}\NormalTok{(\{}
\NormalTok{      name}\OperatorTok{:}\NormalTok{ to}\OperatorTok{.}\AttributeTok{name} \ImportTok{as} \DataTypeTok{string}\OperatorTok{,}
\NormalTok{      params}\OperatorTok{:}\NormalTok{ \{ }\OperatorTok{...}\NormalTok{to}\OperatorTok{.}\AttributeTok{params}\OperatorTok{,}\NormalTok{ locale}\OperatorTok{:}\NormalTok{ savedLocale \}}\OperatorTok{,}
\NormalTok{      query}\OperatorTok{:}\NormalTok{ to}\OperatorTok{.}\AttributeTok{query}\OperatorTok{,}
\NormalTok{      replace}\OperatorTok{:} \KeywordTok{true}\OperatorTok{,}
\NormalTok{    \})}
\NormalTok{  \}}

  \CommentTok{// Sync i18n with URL locale}
  \ControlFlowTok{if}\NormalTok{ (locale }\OperatorTok{\&\&}\NormalTok{ i18n}\OperatorTok{.}\AttributeTok{global}\OperatorTok{.}\AttributeTok{locale}\OperatorTok{.}\AttributeTok{value} \OperatorTok{!==}\NormalTok{ locale) \{}
\NormalTok{    i18n}\OperatorTok{.}\AttributeTok{global}\OperatorTok{.}\AttributeTok{locale}\OperatorTok{.}\AttributeTok{value} \OperatorTok{=}\NormalTok{ locale }\ImportTok{as}\NormalTok{ Locale}
\NormalTok{    localStorage}\OperatorTok{.}\FunctionTok{setItem}\NormalTok{(}\StringTok{\textquotesingle{}user{-}locale\textquotesingle{}}\OperatorTok{,}\NormalTok{ locale)}
\NormalTok{  \}}

  \FunctionTok{next}\NormalTok{()}
\NormalTok{\})}
\end{Highlighting}
\end{Shaded}

Le \textbf{navigation guards} proteggono le rotte in base
all'autenticazione e ai ruoli:

\begin{Shaded}
\begin{Highlighting}[]
\ImportTok{export} \KeywordTok{const}\NormalTok{ requireAuth }\OperatorTok{=}\NormalTok{ (to}\OperatorTok{,}\NormalTok{ \_from}\OperatorTok{,}\NormalTok{ next) }\KeywordTok{=\textgreater{}}\NormalTok{ \{}
  \KeywordTok{const}\NormalTok{ authStore }\OperatorTok{=} \FunctionTok{useAuthStore}\NormalTok{()}
  \ControlFlowTok{if}\NormalTok{ (}\OperatorTok{!}\NormalTok{authStore}\OperatorTok{.}\AttributeTok{isAuthenticated}\NormalTok{) \{}
    \FunctionTok{next}\NormalTok{(\{ name}\OperatorTok{:}\NormalTok{ LOGIN\_ROUTE\_NAME}\OperatorTok{,}\NormalTok{ query}\OperatorTok{:}\NormalTok{ \{ redirect}\OperatorTok{:}\NormalTok{ to}\OperatorTok{.}\AttributeTok{fullPath}\NormalTok{ \} \})}
\NormalTok{  \} }\ControlFlowTok{else}\NormalTok{ \{}
    \FunctionTok{next}\NormalTok{()}
\NormalTok{  \}}
\NormalTok{\}}

\ImportTok{export} \KeywordTok{const}\NormalTok{ requireRole }\OperatorTok{=}\NormalTok{ (role}\OperatorTok{:} \DataTypeTok{string}\NormalTok{) }\KeywordTok{=\textgreater{}}\NormalTok{ \{}
  \ControlFlowTok{return}\NormalTok{ (to}\OperatorTok{,} \ImportTok{from}\OperatorTok{,}\NormalTok{ next) }\KeywordTok{=\textgreater{}}\NormalTok{ \{}
    \KeywordTok{const}\NormalTok{ authStore }\OperatorTok{=} \FunctionTok{useAuthStore}\NormalTok{()}
    \ControlFlowTok{if}\NormalTok{ (authStore}\OperatorTok{.}\AttributeTok{user}\OperatorTok{?.}\AttributeTok{role} \OperatorTok{!==}\NormalTok{ role) \{}
      \FunctionTok{next}\NormalTok{(\{ name}\OperatorTok{:}\NormalTok{ FORBIDDEN\_ROUTE\_NAME \})}
\NormalTok{    \} }\ControlFlowTok{else}\NormalTok{ \{}
      \FunctionTok{next}\NormalTok{()}
\NormalTok{    \}}
\NormalTok{  \}}
\NormalTok{\}}
\end{Highlighting}
\end{Shaded}

Le guards guidano l'utente nell'utilizzo dell'applicazione, impedendo
l'accesso a pagine non autorizzate e reindirizzandolo automaticamente
(ad esempio, se un organizzazione vuole creare un evento ma non ha
effettuato l'accesso, prima si ha un redirect alla pagina di login).

\subsubsection{Comunicazione Real-time con
WebSocket}\label{comunicazione-real-time-con-websocket}

La comunicazione in tempo reale è gestita tramite
\href{http://socket.io/}{\textbf{Socket.IO}} per le notifiche e i
messaggi chat:

\begin{Shaded}
\begin{Highlighting}[]
\NormalTok{socket }\OperatorTok{=} \FunctionTok{io}\NormalTok{(url}\OperatorTok{,}\NormalTok{ \{}
\NormalTok{  auth}\OperatorTok{:}\NormalTok{ \{ token}\OperatorTok{,}\NormalTok{ userId \}}\OperatorTok{,}
\NormalTok{  reconnection}\OperatorTok{:} \KeywordTok{true}\OperatorTok{,}
\NormalTok{  reconnectionAttempts}\OperatorTok{:} \DecValTok{5}\OperatorTok{,}
\NormalTok{  transports}\OperatorTok{:}\NormalTok{ [}\StringTok{\textquotesingle{}websocket\textquotesingle{}}\OperatorTok{,} \StringTok{\textquotesingle{}polling\textquotesingle{}}\NormalTok{]}\OperatorTok{,}
\NormalTok{\})}

\NormalTok{socket}\OperatorTok{.}\FunctionTok{on}\NormalTok{(}\StringTok{\textquotesingle{}connect\textquotesingle{}}\OperatorTok{,}\NormalTok{ () }\KeywordTok{=\textgreater{}}\NormalTok{ \{}
\NormalTok{  handlers}\OperatorTok{.}\FunctionTok{forEach}\NormalTok{((\{ handler}\OperatorTok{,}\NormalTok{ eventType \}) }\KeywordTok{=\textgreater{}}\NormalTok{ \{}
\NormalTok{    socket}\OperatorTok{?.}\FunctionTok{on}\NormalTok{(eventType}\OperatorTok{,}\NormalTok{ handler)}
\NormalTok{  \})}
\NormalTok{\})}
\end{Highlighting}
\end{Shaded}

Gli eventi gestiti includono:

\begin{itemize}
\tightlist{}
\item
  \texttt{user-online} / \texttt{user-offline}~--- Stato online degli
  utenti
\item
  \texttt{new-message}~--- Nuovi messaggi in chat
\item
  \texttt{like-received} / \texttt{follow-received}~--- Notifiche di
  interazione
\item
  \texttt{new-event-published}~--- Nuovi eventi pubblicati da utenti
  seguiti
\end{itemize}

\subsubsection{Composables
Riutilizzabili}\label{composables-riutilizzabili}

I composables incapsulano logiche riutilizzabili tra i componenti. Un
esempio significativo è \texttt{useInfiniteScroll} per la paginazione:

\begin{Shaded}
\begin{Highlighting}[]
\ImportTok{export} \KeywordTok{function} \FunctionTok{useInfiniteScroll}\OperatorTok{\textless{}}\NormalTok{R}\OperatorTok{\textgreater{}}\NormalTok{(config}\OperatorTok{:}\NormalTok{ InfiniteScrollConfiguration}\OperatorTok{\textless{}}\NormalTok{R}\OperatorTok{\textgreater{}}\NormalTok{) \{}
  \KeywordTok{const}\NormalTok{ items}\OperatorTok{:}\NormalTok{ Ref}\OperatorTok{\textless{}}\NormalTok{R[]}\OperatorTok{\textgreater{}} \OperatorTok{=} \FunctionTok{ref}\NormalTok{([])}
  \KeywordTok{const}\NormalTok{ hasMore }\OperatorTok{=} \FunctionTok{ref}\NormalTok{(}\KeywordTok{true}\NormalTok{)}
  \KeywordTok{const}\NormalTok{ loading }\OperatorTok{=} \FunctionTok{ref}\NormalTok{(}\KeywordTok{true}\NormalTok{)}

  \KeywordTok{const}\NormalTok{ onLoad }\OperatorTok{=} \KeywordTok{async}\NormalTok{ (\_index}\OperatorTok{:} \DataTypeTok{number}\OperatorTok{,}\NormalTok{ done}\OperatorTok{:}\NormalTok{ (stop}\OperatorTok{?:} \DataTypeTok{boolean}\NormalTok{) }\KeywordTok{=\textgreater{}} \DataTypeTok{void}\NormalTok{) }\KeywordTok{=\textgreater{}}\NormalTok{ \{}
    \ControlFlowTok{if}\NormalTok{ (}\OperatorTok{!}\NormalTok{hasMore}\OperatorTok{.}\AttributeTok{value}\NormalTok{) \{}
      \FunctionTok{done}\NormalTok{(}\KeywordTok{true}\NormalTok{)}
      \ControlFlowTok{return}
\NormalTok{    \}}
    \ControlFlowTok{await} \FunctionTok{loadItems}\NormalTok{(}\KeywordTok{true}\NormalTok{)}
    \FunctionTok{done}\NormalTok{(}\OperatorTok{!}\NormalTok{hasMore}\OperatorTok{.}\AttributeTok{value}\NormalTok{)}
\NormalTok{  \}}

  \ControlFlowTok{return}\NormalTok{ \{ items}\OperatorTok{,}\NormalTok{ hasMore}\OperatorTok{,}\NormalTok{ loading}\OperatorTok{,}\NormalTok{ onLoad}\OperatorTok{,}\NormalTok{ reload \}}
\NormalTok{\}}
\end{Highlighting}
\end{Shaded}

Altri composables includono:

\begin{itemize}
\tightlist{}
\item
  \texttt{useUserProfile}~--- Logica profilo utente (isOwnProfile,
  isOrganization)
\item
  \texttt{useDarkMode}~--- Gestione tema chiaro/scuro con persistenza
\item
  \texttt{useTranslation}~--- Traduzioni con prefisso automatico
\end{itemize}

\subsubsection{Layout Condivisi}\label{layout-condivisi}

Sono stati definiti anche layout riutilizzabili per garantire
consistenza nell'interfaccia:

\begin{itemize}
\tightlist{}
\item
  \textbf{NavigationWithSearch}: Utilizzato in home ed explore, serve a
  gestire la comparsa della barra di ricerca nella barra di navigazione
  dal momento che esce dalla viewport e viceversa.
\item
  \textbf{TwoColumnLayout}: Utilizzato in chat e impostazioni, con
  supporto mobile che mostra una colonna alla volta
\end{itemize}

\subsubsection{Internazionalizzazione
(i18n)}\label{internazionalizzazione-i18n}

L'applicazione supporta 5 lingue (en, es, fr, it, de). Le traduzioni
sono generate automaticamente in CI a partire dal sorgente inglese:

\begin{Shaded}
\begin{Highlighting}[]
\KeywordTok{const}\NormalTok{ localeModules }\OperatorTok{=}\NormalTok{ import}\OperatorTok{.}\AttributeTok{meta}\OperatorTok{.}\FunctionTok{glob}\NormalTok{(}\StringTok{\textquotesingle{}./locales/*.ts\textquotesingle{}}\OperatorTok{,}\NormalTok{ \{ eager}\OperatorTok{:} \KeywordTok{true}\NormalTok{ \})}

\KeywordTok{const}\NormalTok{ i18n }\OperatorTok{=} \FunctionTok{createI18n}\NormalTok{(\{}
\NormalTok{  legacy}\OperatorTok{:} \KeywordTok{false}\OperatorTok{,}
\NormalTok{  locale}\OperatorTok{:}\NormalTok{ DEFAULT\_LOCALE}\OperatorTok{,}
\NormalTok{  fallbackLocale}\OperatorTok{:}\NormalTok{ DEFAULT\_LOCALE}\OperatorTok{,}
\NormalTok{  messages}\OperatorTok{,}
\NormalTok{\})}
\end{Highlighting}
\end{Shaded}

L'utilizzo nei componenti avviene tramite il composable useTranslation:

\begin{Shaded}
\begin{Highlighting}[]

\KeywordTok{const}\NormalTok{ \{ t \} }\OperatorTok{=} \FunctionTok{useTranslation}\NormalTok{(}\StringTok{\textquotesingle{}components.cards.EventCard\textquotesingle{}}\NormalTok{)}
\end{Highlighting}
\end{Shaded}

Inoltre il selettore delle lingue nelle impostazioni del profilo
utilizza l'API nativa \texttt{Intl.DisplayNames} per mostrare ogni
lingua nel proprio nome nativo (es. ``Italiano'', ``Français'',
``Deutsch''), migliorando l'accessibilità per gli utenti:

\begin{Shaded}
\begin{Highlighting}[]
\KeywordTok{const}\NormalTok{ getLanguageInfo }\OperatorTok{=}\NormalTok{ (code}\OperatorTok{:} \DataTypeTok{string}\NormalTok{)}\OperatorTok{:}\NormalTok{ LanguageOption }\KeywordTok{=\textgreater{}}\NormalTok{ \{}
  \KeywordTok{const}\NormalTok{ nativeNames }\OperatorTok{=} \KeywordTok{new} \BuiltInTok{Intl}\OperatorTok{.}\FunctionTok{DisplayNames}\NormalTok{([code]}\OperatorTok{,}\NormalTok{ \{ type}\OperatorTok{:} \StringTok{\textquotesingle{}language\textquotesingle{}}\NormalTok{ \})}
  \ControlFlowTok{return}\NormalTok{ \{}
\NormalTok{    code}\OperatorTok{,}
\NormalTok{    nativeName}\OperatorTok{:}\NormalTok{ nativeNames}\OperatorTok{.}\FunctionTok{of}\NormalTok{(code) }\OperatorTok{||}\NormalTok{ code}\OperatorTok{.}\FunctionTok{toUpperCase}\NormalTok{()}\OperatorTok{,}
\NormalTok{    flag}\OperatorTok{:} \FunctionTok{getFlagEmoji}\NormalTok{(code)}\OperatorTok{,}
\NormalTok{  \}}
\NormalTok{\}}
\end{Highlighting}
\end{Shaded}

L'internazionalizzazione si estende anche al backend: il servizio
Payments genera i \textbf{biglietti PDF nella lingua dell'utente}, con
traduzioni dedicate.

Il backend predispone inoltre la gestione dei prezzi con le valute e un
sistema di conversione con caching, attualmente non utilizzato dal
frontend ma pronto per future estensioni:

\begin{Shaded}
\begin{Highlighting}[]
\ImportTok{export} \KeywordTok{class}\NormalTok{ CurrencyConverter \{}
  \KeywordTok{private} \KeywordTok{static} \KeywordTok{readonly}\NormalTok{ CACHE\_DURATION }\OperatorTok{=} \DecValTok{24} \OperatorTok{*} \DecValTok{60} \OperatorTok{*} \DecValTok{60} \OperatorTok{*} \DecValTok{1000} \CommentTok{// 24 hours}

  \KeywordTok{static} \KeywordTok{async} \FunctionTok{convertAmount}\NormalTok{(amount}\OperatorTok{:} \DataTypeTok{number}\OperatorTok{,}\NormalTok{ fromCurrency}\OperatorTok{:} \DataTypeTok{string}\OperatorTok{,}\NormalTok{ toCurrency}\OperatorTok{:} \DataTypeTok{string}\NormalTok{)}\OperatorTok{:} \BuiltInTok{Promise}\OperatorTok{\textless{}}\DataTypeTok{number}\OperatorTok{\textgreater{}}\NormalTok{ \{}
    \KeywordTok{const}\NormalTok{ fromRates }\OperatorTok{=} \ControlFlowTok{await} \KeywordTok{this}\OperatorTok{.}\FunctionTok{fetchRatesForCurrency}\NormalTok{(}\ImportTok{from}\NormalTok{)}
    \ControlFlowTok{return} \KeywordTok{this}\OperatorTok{.}\AttributeTok{converter}\OperatorTok{.}\FunctionTok{convert}\NormalTok{(amount}\OperatorTok{,} \ImportTok{from}\OperatorTok{,}\NormalTok{ to}\OperatorTok{,}\NormalTok{ fromRates)}
\NormalTok{  \}}
\NormalTok{\}}
\end{Highlighting}
\end{Shaded}

\textbf{Limiti attuali}: I contenuti user-generated come le descrizioni
degli eventi non sono attualmente tradotti e vengono memorizzati nella
lingua in cui sono stati inseriti dall'organizzazione. Un'estensione
futura potrebbe prevedere la possibilità di inserire descrizioni in più
lingue, con fallback alla lingua originale quando la traduzione non è
disponibile.

\subsubsection{Geolocalizzazione degli
eventi}\label{geolocalizzazione-degli-eventi}

Per l'inserimento della posizione durante la creazione di un evento è
stata utilizzata l'API pubblica di \textbf{Nominatim} (OpenStreetMap)
per la ricerca e il geocoding degli indirizzi. La scelta di
OpenStreetMap è stata dettata dalla sua natura open-source e
dall'assenza di costi di utilizzo.

Poiché la maggior parte degli utenti utilizza Google Maps per la
navigazione, i dati ricevuti da Nominatim vengono elaborati per generare
link compatibili con Google Maps, permettendo all'utente di cliccare
sulla posizione dell'evento e aprirla direttamente nell'applicazione di
navigazione.

\begin{Shaded}
\begin{Highlighting}[]
\ImportTok{export} \KeywordTok{const}\NormalTok{ extractLocationMapsLink }\OperatorTok{=}\NormalTok{ (location}\OperatorTok{:}\NormalTok{ LocationData)}\OperatorTok{:} \DataTypeTok{string} \KeywordTok{=\textgreater{}}\NormalTok{ \{}
  \KeywordTok{const}\NormalTok{ query }\OperatorTok{=} \VerbatimStringTok{\textasciigrave{}}\SpecialCharTok{$\{}\NormalTok{location}\OperatorTok{.}\AttributeTok{name}\SpecialCharTok{\}}\VerbatimStringTok{,}\SpecialCharTok{$\{}\NormalTok{location}\OperatorTok{.}\AttributeTok{road}\SpecialCharTok{\}}\VerbatimStringTok{,}\SpecialCharTok{$\{}\NormalTok{location}\OperatorTok{.}\AttributeTok{city}\SpecialCharTok{\}}\VerbatimStringTok{,}\SpecialCharTok{$\{}\NormalTok{location}\OperatorTok{.}\AttributeTok{country}\SpecialCharTok{\}}\VerbatimStringTok{\textasciigrave{}}
  \ControlFlowTok{return} \VerbatimStringTok{\textasciigrave{}https://www.google.com/maps/search/?api=1\&query=}\SpecialCharTok{$\{}\PreprocessorTok{encodeURIComponent}\NormalTok{(query)}\SpecialCharTok{\}}\VerbatimStringTok{\textasciigrave{}}
\NormalTok{\}}
\end{Highlighting}
\end{Shaded}

\subsection{\texorpdfstring{\textbf{5.2
Backend}}{5.2 Backend}}\label{backend}

Il backend è composto da \textbf{7 microservizi}: 2 sviluppati in
\textbf{Scala 3} con il framework \textbf{Cask} (Users ed Events), 3 in
\textbf{NestJS} (Interactions, Chat e Payments) e 2 in
\textbf{Express.js} (Notifications e Media).

Ogni servizio ha la propria istanza \textbf{MongoDB} dedicata e comunica
con gli altri tramite \textbf{RabbitMQ} con topic exchange. Il servizio
Notifications gestisce le connessioni WebSocket tramite
\textbf{Socket.IO} per le notifiche real-time e il tracciamento dello
stato online degli utenti.

\subsubsection{Struttura del Progetto}\label{struttura-del-progetto-1}

Essendo i microservizi eterogenei segue una descrizione di massima
rappresentativa della struttura delle varie implementazioni, viene usata
la terminologia dello stack MEVN ma alcuni componenti potrebbero avere
un nome/implementazione diversa (e.i. in Nest il concetto di router e
controller viene unito in un unico componente rispetto ad express).

\begin{Shaded}
\begin{Highlighting}[]
\NormalTok{/src/}
\NormalTok{├── presentation/        \# Layer di ingresso al servizio}
\NormalTok{├── application/         \# Layer contenenti le logiche applicative}
\NormalTok{├── domain/              \# Modello dei dati}
\NormalTok{└── infrastructure/      \# Dipendenze del dominio}
\end{Highlighting}
\end{Shaded}

Più nel dettaglio, nel primo layer di presentazione troviamo i router
che definiscono le rotte (path + metodo http) supportate da ciascun
servizio. In application sono specificati i diversi DTO usati per
validare i dati in ingresso agli endpoint e strutturare i dati di
risposta degli stessi e i controller delle varie rotte. Nel dominio sono
definiti i modelli delle entità di interesse per il particolare servizio
e in infrastructure troviamo le varie dipendenze esterne, tipicamente le
implementazioni dei connectors a database e message-broker

\subsubsection{Autenticazione JWT}\label{autenticazione-jwt}

L'autenticazione è basata su \textbf{JWT}. Il servizio Users integra
\textbf{Keycloak} come identity provider per la gestione delle
credenziali utente (registrazione, login, gestione delle sessioni).
Keycloak genera i token JWT firmati e il servizio Users espone un
endpoint \texttt{/public-keys} che restituisce le chiavi pubbliche
necessarie per la validazione.

Gli altri microservizi recuperano le chiavi pubbliche da questo endpoint
e le salvano in cache localmente. Quando ricevono una richiesta
autenticata, estraggono il token dall'header \texttt{Authorization}, ne
verificano la firma e decodificano i claim.

Oltre alle REST API, l'autenticazione è stata implementata anche per le
connessioni WebSocket: il token viene passato durante l'handshake della
connessione Socket.IO e validato prima di stabilire il canale. Questo
permette di inviare notifiche con dati (oltre che solo segnali) in modo
sicuro.

\subsubsection{Express Middlewares}\label{express-middlewares}

Per gestire le richieste HTTP in \textbf{Express} sono stati utilizzati
i middleware, in particolare è stato definito un middleware per
l'autenticazione:

\begin{Shaded}
\begin{Highlighting}[]
\ImportTok{export} \KeywordTok{function} \FunctionTok{createAuthMiddleware}\NormalTok{(options}\OperatorTok{:}\NormalTok{ \{ optional}\OperatorTok{?:} \DataTypeTok{boolean}\NormalTok{ \} }\OperatorTok{=}\NormalTok{ \{\}) \{}
  \ControlFlowTok{return}\NormalTok{ (req}\OperatorTok{:}\NormalTok{ Request}\OperatorTok{,}\NormalTok{ res}\OperatorTok{:}\NormalTok{ Response}\OperatorTok{,}\NormalTok{ next}\OperatorTok{:}\NormalTok{ NextFunction)}\OperatorTok{:} \DataTypeTok{void} \KeywordTok{=\textgreater{}}\NormalTok{ \{}
    \KeywordTok{const}\NormalTok{ token }\OperatorTok{=}\NormalTok{ req}\OperatorTok{.}\AttributeTok{headers}\OperatorTok{.}\AttributeTok{authorization}\OperatorTok{?.}\FunctionTok{replace}\NormalTok{(}\StringTok{"Bearer "}\OperatorTok{,} \StringTok{""}\NormalTok{)}

    \ControlFlowTok{if}\NormalTok{ (}\OperatorTok{!}\NormalTok{token) \{}
      \ControlFlowTok{if}\NormalTok{ (options}\OperatorTok{.}\AttributeTok{optional}\NormalTok{) }\ControlFlowTok{return} \FunctionTok{next}\NormalTok{()}
      \ControlFlowTok{return}\NormalTok{ res}\OperatorTok{.}\FunctionTok{status}\NormalTok{(}\DecValTok{401}\NormalTok{)}\OperatorTok{.}\FunctionTok{json}\NormalTok{(\{ error}\OperatorTok{:} \StringTok{"No token provided"}\NormalTok{ \})}
\NormalTok{    \}}

    \ControlFlowTok{try}\NormalTok{ \{}
      \KeywordTok{const}\NormalTok{ payload }\OperatorTok{=} \ControlFlowTok{await}\NormalTok{ JwtService}\OperatorTok{.}\FunctionTok{verifyToken}\NormalTok{(token)}
\NormalTok{      req}\OperatorTok{.}\AttributeTok{userId} \OperatorTok{=}\NormalTok{ payload}\OperatorTok{.}\AttributeTok{user\_id}
      \FunctionTok{next}\NormalTok{()}
\NormalTok{    \} }\ControlFlowTok{catch}\NormalTok{ (error) \{}
\NormalTok{      res}\OperatorTok{.}\FunctionTok{status}\NormalTok{(}\DecValTok{401}\NormalTok{)}\OperatorTok{.}\FunctionTok{json}\NormalTok{(\{ error}\OperatorTok{:} \StringTok{"Authentication failed"}\NormalTok{ \})}
\NormalTok{    \}}
\NormalTok{  \}}
\NormalTok{\}}

\ImportTok{export} \KeywordTok{const}\NormalTok{ authMiddleware }\OperatorTok{=} \FunctionTok{createAuthMiddleware}\NormalTok{()}
\ImportTok{export} \KeywordTok{const}\NormalTok{ optionalAuthMiddleware }\OperatorTok{=} \FunctionTok{createAuthMiddleware}\NormalTok{(\{ optional}\OperatorTok{:} \KeywordTok{true}\NormalTok{ \})}
\end{Highlighting}
\end{Shaded}

Le routes utilizzano poi questo middleware per proteggere gli endpoint:

\begin{Shaded}
\begin{Highlighting}[]
\ImportTok{export} \KeywordTok{function} \FunctionTok{createNotificationRoutes}\NormalTok{(controller}\OperatorTok{:}\NormalTok{ NotificationController)}\OperatorTok{:}\NormalTok{ Router \{}
  \KeywordTok{const}\NormalTok{ router }\OperatorTok{=} \FunctionTok{Router}\NormalTok{()}
\NormalTok{  router}\OperatorTok{.}\FunctionTok{use}\NormalTok{(authMiddleware)}

\NormalTok{  router}\OperatorTok{.}\FunctionTok{get}\NormalTok{(}\StringTok{"/"}\OperatorTok{,}\NormalTok{ (req}\OperatorTok{,}\NormalTok{ res}\OperatorTok{,}\NormalTok{ next) }\KeywordTok{=\textgreater{}}
\NormalTok{    controller}\OperatorTok{.}\FunctionTok{getNotificationsByUserId}\NormalTok{(req}\OperatorTok{,}\NormalTok{ res}\OperatorTok{,}\NormalTok{ next))}
\NormalTok{  router}\OperatorTok{.}\FunctionTok{get}\NormalTok{(}\StringTok{"/unread{-}count"}\OperatorTok{,}\NormalTok{ (req}\OperatorTok{,}\NormalTok{ res}\OperatorTok{,}\NormalTok{ next) }\KeywordTok{=\textgreater{}}
\NormalTok{    controller}\OperatorTok{.}\FunctionTok{getUnreadCount}\NormalTok{(req}\OperatorTok{,}\NormalTok{ res}\OperatorTok{,}\NormalTok{ next))}
    \CommentTok{//other routes...}

  \ControlFlowTok{return}\NormalTok{ router}
\NormalTok{\}}
\end{Highlighting}
\end{Shaded}

A livello applicativo vengono inoltre utilizzati i middleware standard
di Express:~\texttt{cors()}~per gestire le richieste cross-origin
(necessario per le chiamate dal frontend) e~\texttt{express.json()}~per
il parsing automatico del body JSON.

\subsubsection{Authenticated WebSocket Gateway con
Socket.IO}\label{authenticated-websocket-gateway-con-socket.io}

Il gateway \textbf{Socket.IO} gestisce l'autenticazione delle
connessioni WebSocket e la distribuzione delle notifiche:

\begin{Shaded}
\begin{Highlighting}[]
\ImportTok{export} \KeywordTok{class}\NormalTok{ SocketIOGateway }\KeywordTok{implements}\NormalTok{ NotificationGateway \{}
  \KeywordTok{private}\NormalTok{ userSockets}\OperatorTok{:} \BuiltInTok{Map}\OperatorTok{\textless{}}\DataTypeTok{string}\OperatorTok{,} \BuiltInTok{Set}\OperatorTok{\textless{}}\DataTypeTok{string}\OperatorTok{\textgreater{}\textgreater{}} \OperatorTok{=} \KeywordTok{new} \BuiltInTok{Map}\NormalTok{()}

  \KeywordTok{private} \FunctionTok{setupAuthMiddleware}\NormalTok{()}\OperatorTok{:} \DataTypeTok{void}\NormalTok{ \{}
    \KeywordTok{this}\OperatorTok{.}\AttributeTok{io}\OperatorTok{.}\FunctionTok{use}\NormalTok{(}\KeywordTok{async}\NormalTok{ (socket}\OperatorTok{:} \BuiltInTok{Socket}\OperatorTok{,}\NormalTok{ next) }\KeywordTok{=\textgreater{}}\NormalTok{ \{}
      \KeywordTok{const}\NormalTok{ token }\OperatorTok{=}\NormalTok{ socket}\OperatorTok{.}\AttributeTok{handshake}\OperatorTok{.}\AttributeTok{auth}\OperatorTok{.}\AttributeTok{token} \OperatorTok{||}
\NormalTok{                    socket}\OperatorTok{.}\AttributeTok{handshake}\OperatorTok{.}\AttributeTok{headers}\OperatorTok{.}\AttributeTok{authorization}\OperatorTok{?.}\FunctionTok{replace}\NormalTok{(}\StringTok{"Bearer "}\OperatorTok{,} \StringTok{""}\NormalTok{)}

      \ControlFlowTok{if}\NormalTok{ (}\OperatorTok{!}\NormalTok{token) \{}
        \ControlFlowTok{return} \FunctionTok{next}\NormalTok{(}\KeywordTok{new} \BuiltInTok{Error}\NormalTok{(}\StringTok{"Authentication error: No token provided"}\NormalTok{))}
\NormalTok{      \}}

      \ControlFlowTok{try}\NormalTok{ \{}
        \KeywordTok{const}\NormalTok{ payload }\OperatorTok{=} \ControlFlowTok{await}\NormalTok{ JwtService}\OperatorTok{.}\FunctionTok{verifyToken}\NormalTok{(token)}
\NormalTok{        socket}\OperatorTok{.}\AttributeTok{data}\OperatorTok{.}\AttributeTok{userId} \OperatorTok{=}\NormalTok{ payload}\OperatorTok{.}\AttributeTok{user\_id}
        \FunctionTok{next}\NormalTok{()}
\NormalTok{      \} }\ControlFlowTok{catch}\NormalTok{ (err) \{}
        \FunctionTok{next}\NormalTok{(}\KeywordTok{new} \BuiltInTok{Error}\NormalTok{(}\StringTok{"Authentication error: Invalid token"}\NormalTok{))}
\NormalTok{      \}}
\NormalTok{    \})}
\NormalTok{  \}}

  \FunctionTok{sendNotificationToUser}\NormalTok{(userId}\OperatorTok{:} \DataTypeTok{string}\OperatorTok{,}\NormalTok{ notification}\OperatorTok{:} \DataTypeTok{any}\NormalTok{)}\OperatorTok{:} \BuiltInTok{Promise}\OperatorTok{\textless{}}\DataTypeTok{void}\OperatorTok{\textgreater{}}\NormalTok{ \{}
    \KeywordTok{const}\NormalTok{ topic }\OperatorTok{=} \KeywordTok{this}\OperatorTok{.}\FunctionTok{getTopicFromNotificationType}\NormalTok{(notification}\OperatorTok{.}\AttributeTok{type}\NormalTok{)}
    \KeywordTok{this}\OperatorTok{.}\AttributeTok{io}\OperatorTok{.}\FunctionTok{to}\NormalTok{(}\VerbatimStringTok{\textasciigrave{}user:}\SpecialCharTok{$\{}\NormalTok{userId}\SpecialCharTok{\}}\VerbatimStringTok{\textasciigrave{}}\NormalTok{)}\OperatorTok{.}\FunctionTok{emit}\NormalTok{(topic}\OperatorTok{,}\NormalTok{ notification)}
    \ControlFlowTok{return} \BuiltInTok{Promise}\OperatorTok{.}\FunctionTok{resolve}\NormalTok{()}
\NormalTok{  \}}

  \FunctionTok{broadcastUserOnline}\NormalTok{(userId}\OperatorTok{:} \DataTypeTok{string}\NormalTok{)}\OperatorTok{:} \DataTypeTok{void}\NormalTok{ \{}
    \KeywordTok{this}\OperatorTok{.}\AttributeTok{io}\OperatorTok{.}\FunctionTok{except}\NormalTok{(}\VerbatimStringTok{\textasciigrave{}user:}\SpecialCharTok{$\{}\NormalTok{userId}\SpecialCharTok{\}}\VerbatimStringTok{\textasciigrave{}}\NormalTok{)}\OperatorTok{.}\FunctionTok{emit}\NormalTok{(}\StringTok{"user{-}online"}\OperatorTok{,}\NormalTok{ \{}
\NormalTok{      userId}\OperatorTok{,}
\NormalTok{      timestamp}\OperatorTok{:} \KeywordTok{new} \BuiltInTok{Date}\NormalTok{()}
\NormalTok{    \})}
\NormalTok{  \}}

  \FunctionTok{broadcastUserOffline}\NormalTok{(userId}\OperatorTok{:} \DataTypeTok{string}\NormalTok{)}\OperatorTok{:} \DataTypeTok{void}\NormalTok{ \{}
    \KeywordTok{this}\OperatorTok{.}\AttributeTok{io}\OperatorTok{.}\FunctionTok{emit}\NormalTok{(}\StringTok{"user{-}offline"}\OperatorTok{,}\NormalTok{ \{}
\NormalTok{      userId}\OperatorTok{,}
\NormalTok{      timestamp}\OperatorTok{:} \KeywordTok{new} \BuiltInTok{Date}\NormalTok{()}
\NormalTok{    \})}
\NormalTok{  \}}
\NormalTok{\}}
\end{Highlighting}
\end{Shaded}

\subsubsection{Endpoint con autenticazione
opzionale}\label{endpoint-con-autenticazione-opzionale}

Alcuni endpoint utilizzano l'\textbf{autenticazione opzionale}: sono
accessibili sia da utenti autenticati che non, in questo caso la
risposta può variare pur mantenendo il formato consistente. Ad esempio,
l'endpoint per ottenere il profilo di un utente restituisce solo i dati
pubblici (username e informazioni del profilo) se la richiesta non è
autenticata, mentre include anche i dati privati dell'account
(e.g.~email, interessi) se l'utente autenticato sta richiedendo le
proprie informazioni:

\begin{Shaded}
\begin{Highlighting}[]
\NormalTok{@cask}\OperatorTok{.}\FunctionTok{get}\OperatorTok{(}\StringTok{"/:userId"}\OperatorTok{)}
  \KeywordTok{def} \FunctionTok{getUser}\OperatorTok{(}\NormalTok{userId}\OperatorTok{:} \ExtensionTok{String}\OperatorTok{,}\NormalTok{ req}\OperatorTok{:} \ExtensionTok{Request}\OperatorTok{):} \ExtensionTok{Response}\OperatorTok{[}\ExtensionTok{String}\OperatorTok{]} \OperatorTok{=}
\NormalTok{    userService}\OperatorTok{.}\FunctionTok{getUserById}\OperatorTok{(}\NormalTok{userId}\OperatorTok{)} \ControlFlowTok{match}
      \ControlFlowTok{case} \FunctionTok{Left}\OperatorTok{(}\NormalTok{err}\OperatorTok{)} \OperatorTok{=\textgreater{}} \ExtensionTok{Response}\OperatorTok{(}\NormalTok{err}\OperatorTok{,} \DecValTok{404}\OperatorTok{)}
      \ControlFlowTok{case} \FunctionTok{Right}\OperatorTok{(}\NormalTok{role}\OperatorTok{,}\NormalTok{ user}\OperatorTok{)} \OperatorTok{=\textgreater{}}
        \KeywordTok{val}\NormalTok{ isOwner}\OperatorTok{:} \ExtensionTok{Boolean} \OperatorTok{=} \FunctionTok{authenticateAndAuthorize}\OperatorTok{(}\NormalTok{req}\OperatorTok{,}\NormalTok{ userId}\OperatorTok{).}\NormalTok{isRight}
        \KeywordTok{val}\NormalTok{ json }\OperatorTok{=} \ControlFlowTok{if}\NormalTok{ isOwner then}
\NormalTok{            user}\OperatorTok{.}\FunctionTok{toOwnedUserDTO}\OperatorTok{(}\NormalTok{userId}\OperatorTok{,}\NormalTok{ role}\OperatorTok{).}\NormalTok{asJson}
        \ControlFlowTok{else}
\NormalTok{          user}\OperatorTok{.}\FunctionTok{toUserDTO}\OperatorTok{(}\NormalTok{userId}\OperatorTok{,}\NormalTok{ role}\OperatorTok{).}\NormalTok{asJson}
        \ExtensionTok{Response}\OperatorTok{(}\NormalTok{json}\OperatorTok{.}\NormalTok{spaces2}\OperatorTok{,} \DecValTok{200}\OperatorTok{,} \BuiltInTok{Seq}\OperatorTok{(}\StringTok{"Content{-}Type"} \OperatorTok{{-}\textgreater{}} \StringTok{"application/json"}\OperatorTok{))}
\end{Highlighting}
\end{Shaded}


\chapter{Test}
\section{6~--- Test}\label{test}

Il frontend prodotto è stato testato principalmente su Chrome sfruttando
i Chrome DevTools per testare la responsività del layout su diversi
dispositivi ma anche direttamente da mobile una volta messo online il
sito. Per quanto riguarda il backend sono stati effettuati sia unit test
per i singoli componenti sia test e2e principalmente da Swagger e in
integrazione con il frontend.

\subsection{6.1~--- Accessibilità (a11y)}\label{accessibilituxe0-a11y}

Per garantire l'accessibilità dell'interfaccia sono stati adottati due
approcci complementari: \textbf{analisi statica} durante lo sviluppo e
\textbf{test automatizzati} sulle pagine web a runtime.

\subsubsection{Analisi Statica con
ESLint}\label{analisi-statica-con-eslint}

Il progetto integra \textbf{eslint-plugin-vuejs-\allowbreak%
accessibility} nella configurazione ESLint, identificando potenziali violazioni delle linee guida WCAG direttamente durante lo sviluppo:

\begin{Shaded}
\begin{Highlighting}[]
\ImportTok{import}\NormalTok{ vuejsAccessibility }\ImportTok{from} \StringTok{\textquotesingle{}eslint{-}plugin{-}vuejs{-}accessibility\textquotesingle{}}

\ImportTok{export} \ImportTok{default} \FunctionTok{defineConfig}\NormalTok{([}
  \CommentTok{// ... altre configurazioni}
  \OperatorTok{...}\NormalTok{vuejsAccessibility}\OperatorTok{.}\AttributeTok{configs}\NormalTok{[}\StringTok{\textquotesingle{}flat/recommended\textquotesingle{}}\NormalTok{]}\OperatorTok{,}
\NormalTok{])}
\end{Highlighting}
\end{Shaded}

Il plugin verifica automaticamente la presenza di attributi \texttt{alt}
nelle immagini, la corretta associazione di label ai form e l'uso
appropriato di ruoli \texttt{ARIA} segnalando le violazioni come warning
o errori durante il linting.

\subsubsection{Test Automatizzati con Puppeteer e
Lighthouse}\label{test-automatizzati-con-puppeteer-e-lighthouse}

Oltre all'analisi statica, il sistema implementa test di accessibilità
dinamici tramite \textbf{Puppeteer} e \textbf{Lighthouse}. Lo script
\texttt{a11y-check-multi.js} esegue un audit automatizzato sulle pagine
specificate in fase di configurazione:

\begin{Shaded}
\begin{Highlighting}[]
\KeywordTok{const}\NormalTok{ config }\OperatorTok{=}\NormalTok{ \{}
\NormalTok{  baseUrl}\OperatorTok{:} \BuiltInTok{process}\OperatorTok{.}\AttributeTok{env}\OperatorTok{.}\AttributeTok{BASE\_URL} \OperatorTok{||} \StringTok{\textquotesingle{}http://localhost:5173/it\textquotesingle{}}\OperatorTok{,}
\NormalTok{  pages}\OperatorTok{:}\NormalTok{ [}
\NormalTok{    \{ name}\OperatorTok{:} \StringTok{\textquotesingle{}Home\textquotesingle{}}\OperatorTok{,}\NormalTok{ path}\OperatorTok{:} \StringTok{\textquotesingle{}/\textquotesingle{}}\NormalTok{ \}}\OperatorTok{,}
\NormalTok{    \{ name}\OperatorTok{:} \StringTok{\textquotesingle{}Explore\textquotesingle{}}\OperatorTok{,}\NormalTok{ path}\OperatorTok{:} \StringTok{\textquotesingle{}/explore\textquotesingle{}}\NormalTok{ \}}\OperatorTok{,}
\NormalTok{    \{ name}\OperatorTok{:} \StringTok{\textquotesingle{}Login\textquotesingle{}}\OperatorTok{,}\NormalTok{ path}\OperatorTok{:} \StringTok{\textquotesingle{}/login\textquotesingle{}}\NormalTok{ \}}\OperatorTok{,}
\NormalTok{    \{ name}\OperatorTok{:} \StringTok{\textquotesingle{}Register\textquotesingle{}}\OperatorTok{,}\NormalTok{ path}\OperatorTok{:} \StringTok{\textquotesingle{}/register\textquotesingle{}}\NormalTok{ \}}\OperatorTok{,}
\NormalTok{    \{ name}\OperatorTok{:} \StringTok{\textquotesingle{}Create Event\textquotesingle{}}\OperatorTok{,}\NormalTok{ path}\OperatorTok{:} \StringTok{\textquotesingle{}/create{-}event\textquotesingle{}}\NormalTok{ \}}\OperatorTok{,}
\NormalTok{    \{ name}\OperatorTok{:} \StringTok{\textquotesingle{}Event Details\textquotesingle{}}\OperatorTok{,}\NormalTok{ path}\OperatorTok{:} \StringTok{\textquotesingle{}/events/547a3b27{-}344a{-}4318{-}b17e{-}edf7cd14aee3\textquotesingle{}}\NormalTok{ \}}\OperatorTok{,}
\NormalTok{    \{}
\NormalTok{      name}\OperatorTok{:} \StringTok{\textquotesingle{}Organization Profile [Published Events Tab]\textquotesingle{}}\OperatorTok{,}
\NormalTok{      path}\OperatorTok{:} \StringTok{\textquotesingle{}/users/7dee946f{-}3ab9{-}41b5{-}92e9{-}ea6264d9dd35\#publishedEvents\textquotesingle{}}\OperatorTok{,}
\NormalTok{    \}}\OperatorTok{,}
\NormalTok{  ]}\OperatorTok{,}
\NormalTok{  minScore}\OperatorTok{:} \PreprocessorTok{parseInt}\NormalTok{(}\BuiltInTok{process}\OperatorTok{.}\AttributeTok{env}\OperatorTok{.}\AttributeTok{MIN\_A11Y\_SCORE} \OperatorTok{||} \StringTok{\textquotesingle{}80\textquotesingle{}}\NormalTok{)}\OperatorTok{,}
\NormalTok{  themeMode}\OperatorTok{:} \BuiltInTok{process}\OperatorTok{.}\AttributeTok{env}\OperatorTok{.}\AttributeTok{THEME\_MODE} \OperatorTok{||} \StringTok{\textquotesingle{}both\textquotesingle{}}\OperatorTok{,} \CommentTok{// \textquotesingle{}light\textquotesingle{}, \textquotesingle{}dark\textquotesingle{}, \textquotesingle{}both\textquotesingle{}}
\NormalTok{\}}
\end{Highlighting}
\end{Shaded}

Lo script:

\begin{enumerate}
\def\labelenumi{\arabic{enumi}.}
\tightlist{}
\item
  \textbf{Lancia Chrome in modalità
  headless}~tramite~\texttt{chrome-launcher}
\item
  \textbf{Connette Puppeteer}~per controllare il browser e impostare il
  tema
\item
  \textbf{Esegue Lighthouse}~sulla categoria accessibilità per ogni
  pagina
\item
  \textbf{Testa entrambi i temi}~(light e dark mode) per garantire
  l'accessibilità in tutte le condizioni
\item
  \textbf{Genera report dettagliati}~in formato HTML e un summary
  testuale
\end{enumerate}

\begin{Shaded}
\begin{Highlighting}[]
\KeywordTok{async} \KeywordTok{function} \FunctionTok{runLighthouseOnPage}\NormalTok{(url}\OperatorTok{,}\NormalTok{ port}\OperatorTok{,}\NormalTok{ themeName) \{}
  \KeywordTok{const}\NormalTok{ options }\OperatorTok{=}\NormalTok{ \{}
\NormalTok{    logLevel}\OperatorTok{:} \StringTok{\textquotesingle{}error\textquotesingle{}}\OperatorTok{,}
\NormalTok{    output}\OperatorTok{:}\NormalTok{ [}\StringTok{\textquotesingle{}json\textquotesingle{}}\OperatorTok{,} \StringTok{\textquotesingle{}html\textquotesingle{}}\NormalTok{]}\OperatorTok{,}
\NormalTok{    onlyCategories}\OperatorTok{:}\NormalTok{ [}\StringTok{\textquotesingle{}accessibility\textquotesingle{}}\NormalTok{]}\OperatorTok{,}
\NormalTok{    port}\OperatorTok{:}\NormalTok{ port}\OperatorTok{,}
\NormalTok{  \}}
  \KeywordTok{const}\NormalTok{ runnerResult }\OperatorTok{=} \ControlFlowTok{await} \FunctionTok{lighthouse}\NormalTok{(url}\OperatorTok{,}\NormalTok{ options)}
  \ControlFlowTok{return}\NormalTok{ \{}
\NormalTok{    lhr}\OperatorTok{:}\NormalTok{ runnerResult}\OperatorTok{.}\AttributeTok{lhr}\OperatorTok{,}
\NormalTok{    reportHtml}\OperatorTok{:}\NormalTok{ runnerResult}\OperatorTok{.}\AttributeTok{report}\NormalTok{[}\DecValTok{1}\NormalTok{]}\OperatorTok{,}
\NormalTok{    theme}\OperatorTok{:}\NormalTok{ themeName}\OperatorTok{,}
\NormalTok{  \}}
\NormalTok{\}}
\end{Highlighting}
\end{Shaded}

Il report generato include:

\begin{itemize}
\tightlist{}
\item
  \textbf{Score di accessibilità}~per ogni pagina (scala 0--100)
\item
  \textbf{Violazioni critiche}~con conteggio degli elementi coinvolti
\item
  \textbf{Top 5 problemi più frequenti}~nell'intera applicazione
\item
  \textbf{Status pass/fail}~basato sulla soglia minima configurata
  (default 80/100)
\end{itemize}

Per lanciare i test è stato necessario predisporre una modalità ad hoc,
eseguibile con \texttt{npm\ run\ prod:a11y} che prevede login automatico
come organizzazione e disabilita le guardie lato frontend così da
permettere la corretta visualizzazione di tutte le pagine.

Lo script può poi essere eseguito tramite \texttt{npm\ run\ a11y:multi}
e restituisce un exit code non-zero in caso di pagine sotto la soglia
minima, permettendo l'integrazione in pipeline CI/CD\@.

Al momento, dato che il processo di seed iniziale genera ogni volta id
casuali, è necessario ricontrollare i link nella configuazione ad ogni
deploy con seeding.

\subsection{6.2~--- Euristiche di Nielsen}\label{euristiche-di-nielsen}

Per offrire una migliore usabilità e user experience, il sistema è stato
sottoposto alla valutazione delle euristiche di Nielsen. Di seguito le
considerazioni emerse:

\begin{enumerate}
\def\labelenumi{\arabic{enumi}.}
\tightlist{}
\item
  \textbf{Visibilità dello stato del sistema}: per segnalare attività in
  corso viene mostrata un'icona animata di caricamento. Lo stato online
  degli utenti è visibile in tempo reale nella chat tramite un apposito
  label, e le notifiche push informano l'utente di nuovi messaggi, like
  ricevuti o eventi pubblicati dalle organizzazioni seguite.
\item
  \textbf{Corrispondenza tra sistema e mondo reale}: la terminologia e
  le icone utilizzate sono in linea con le convenzioni dei social
  network (cuore per i like, fumetto per la chat, campanella per le
  notifiche). I concetti di ``evento'', ``biglietto'' e
  ``organizzazione'' rispecchiano il dominio reale.
\item
  \textbf{Controllo e libertà per l'utente}: l'applicazione non presenta
  percorsi obbligati. L'utente può navigare liberamente tra le sezioni,
  tornare indietro in qualsiasi momento e le operazioni critiche (come
  l'eliminazione di un evento) richiedono conferma esplicita.
\item
  \textbf{Consistenza e standard}: l'applicazione mantiene un linguaggio
  visivo uniforme grazie al framework Quasar. I pulsanti primari,
  secondari e di pericolo hanno stili distintivi e coerenti in tutto il
  sistema. La palette colori è stata definita in fase di design e
  applicata uniformemente, inclusa la modalità dark.
\item
  \textbf{Prevenzione dall'errore}: i form implementano validazione in
  tempo reale (formato di email, password e campi obbligatori in
  generale) con feedback immediato. Le navigation guards impediscono
  l'accesso a pagine non autorizzate, reindirizzando automaticamente
  l'utente (es. alla pagina di login se necessaria autenticazione).
\item
  \textbf{Riconoscimento anziché ricordo}: i layout sono consistenti tra
  le pagine. La barra di navigazione e i tab mantengono la stessa
  posizione, le card degli eventi hanno struttura uniforme e le icone
  sono autoesplicative senza necessità di tooltip.
\item
  \textbf{Flessibilità ed efficienza d'uso}: per gli utenti esperti sono
  disponibili scorciatoie da tastiera globali: \texttt{Ctrl/Cmd\ +\ D}
  per il toggle della dark mode, \texttt{Ctrl/Cmd\ +\ H} per tornare
  alla home, \texttt{Ctrl/Cmd\ +\ P} per aprire il profilo e
  \texttt{Ctrl/Cmd\ +\ E} per accedere alla creazione eventi.
\item
  \textbf{Design e estetica minimalista}: il design segue il principio
  KISS con approccio mobile-first. Ogni pagina presenta poche azioni ben
  distinte: la home mostra gli eventi in evidenza, la pagina explore
  permette la ricerca, il profilo gestisce le informazioni personali.
\item
  \textbf{Aiuto all'utente}: i messaggi di errore sono contestuali e
  descrittivi. Le notifiche toast informano l'utente dell'esito delle
  operazioni e i campi dei form mostrano label di errore specifiche
  (e.g.~``Email non valida'', ``Password troppo corta'').
\item
  \textbf{Documentazione}: vista la semplicità d'uso derivante dalle
  scelte di design, non è stata necessaria documentazione esterna. Le
  scorciatoie da tastiera disponibili (\texttt{Ctrl/Cmd\ +\ D/H/P/E})
  sono documentate internamente nel
  composable~\texttt{useKeyboardShortcuts}, pronte per essere esposte in
  una futura sezione ``Keyboard Shortcuts'' nelle impostazioni.
\end{enumerate}

\subsection{6.3~--- Test di Usabilità}\label{test-di-usabilituxe0}

Durante lo sviluppo dell'applicazione, questa è stata fatta provare a
diversi utenti, per ricevere di volta in volta dei feedback utili a
migliorare lo sviluppo e la resa finale del prodotto.

Ai soggetti del test non è stata fornita nessuna linea guida, ci si è
limitati a richiedere una particolare azione da svolgere al fine di
osservare come gli utenti avrebbero cercato di eseguire i compiti
richiesti vedendo per la prima volta il sistema.

\textbf{Test per Organizzazioni}: fra i compiti assegnati hanno figurato
la creazione di eventi mediante l'apposito editor, che si è rivelato
semplice ed intuitivo.

\textbf{Test per Membri}: fra i compiti figurava la ricerca degli
eventi, l'acquisto dei biglietti, la modifica del profilo e di
effettuare una recensione per un evento.

In generale gli utenti hanno trovato la disposizione dei vari elementi
sostanzialmente adeguata. Le osservazioni sollevate erano perlopiù su
aspetti stilistici e sono state incorporate nelle successive iterazioni
di sviluppo.


\chapter{Deployment}
\section{7~--- Deployment}\label{deployment}

Il deploy dell'applicazione è stato automatizzato tramite Docker e
Docker Compose: tutti i servizi vengono eseguiti come container
indipendenti e lanciati tramite uno script centralizzato.

\subsection{7.1~--- Installazione}\label{installazione}

\subsubsection{1. Clonare il repository}\label{clonare-il-repository}

\begin{Shaded}
\begin{Highlighting}[]
\FunctionTok{git}\NormalTok{ clone https://github.com/EvenToNight/EvenToNight.git}
\BuiltInTok{cd}\NormalTok{ EvenToNight}
\end{Highlighting}
\end{Shaded}

\subsubsection{2. Configurare le variabili
d'ambiente}\label{configurare-le-variabili-dambiente}

\begin{Shaded}
\begin{Highlighting}[]
\FunctionTok{cp}\NormalTok{ .env.template .env}
\CommentTok{\# Modificare .env e compilare tutti i campi richiesti}
\CommentTok{\# Nota: Se si utilizza il flag {-}{-}no{-}deps, le chiavi Stripe possono contenere valori arbitrari. Tutti i valori devono essere compilati.}
\end{Highlighting}
\end{Shaded}

\subsubsection{3. Avviare l'applicazione}\label{avviare-lapplicazione}

\paragraph{Opzione A:\@ Utilizzo di immagini pre-compilate da ghcr.io
(Consigliato)}\label{opzione-a-utilizzo-di-immagini-pre-compilate-da-ghcr.io-consigliato}

\textbf{Download delle immagini:}

\begin{Shaded}
\begin{Highlighting}[]
\ExtensionTok{./scripts/composeApplication.sh}\NormalTok{ pull}
\end{Highlighting}
\end{Shaded}

\textbf{Download delle immagini con seeding del database:}

\begin{Shaded}
\begin{Highlighting}[]
\ExtensionTok{./scripts/composeApplication.sh} \AttributeTok{{-}{-}init{-}db}\NormalTok{ pull}
\end{Highlighting}
\end{Shaded}

\textbf{Deploy dell'applicazione:}

\begin{Shaded}
\begin{Highlighting}[]
\ExtensionTok{./scripts/composeApplication.sh}\NormalTok{ up }\AttributeTok{{-}d} \AttributeTok{{-}{-}wait}
\end{Highlighting}
\end{Shaded}

\textbf{Deploy con seeding del database:}

\begin{Shaded}
\begin{Highlighting}[]
\ExtensionTok{./scripts/composeApplication.sh} \AttributeTok{{-}{-}init{-}db}\NormalTok{ up }\AttributeTok{{-}d} \AttributeTok{{-}{-}wait}
\end{Highlighting}
\end{Shaded}

\textbf{Deploy in modalità sviluppo} (con porte mappate sull'host e
dashboard per database/RabbitMQ/Traefik):

\begin{Shaded}
\begin{Highlighting}[]
\ExtensionTok{./scripts/composeApplication.sh} \AttributeTok{{-}{-}init{-}db} \AttributeTok{{-}{-}dev}\NormalTok{ up }\AttributeTok{{-}d} \AttributeTok{{-}{-}wait}
\end{Highlighting}
\end{Shaded}

\paragraph{Opzione B:\@ Build locale}\label{opzione-b-build-locale}

Aggiungere il flag \texttt{-\/-build} per compilare i servizi localmente
invece di utilizzare le immagini pre-compilate:

\begin{Shaded}
\begin{Highlighting}[]
\CommentTok{\# Build e deploy}
\ExtensionTok{./scripts/composeApplication.sh}\NormalTok{ up }\AttributeTok{{-}{-}build} \AttributeTok{{-}d} \AttributeTok{{-}{-}wait}

\CommentTok{\# Build e deploy con seeding}
\ExtensionTok{./scripts/composeApplication.sh} \AttributeTok{{-}{-}init{-}db}\NormalTok{ up }\AttributeTok{{-}{-}build} \AttributeTok{{-}d} \AttributeTok{{-}{-}wait}
\end{Highlighting}
\end{Shaded}

\paragraph{Flag aggiuntivi}\label{flag-aggiuntivi}

\textbf{\texttt{-\/-no-deps}}: Esclude le dipendenze esterne (Stripe)

È possibile aggiungere \texttt{-\/-no-deps} a qualsiasi comando di
deploy per escludere i servizi esterni:

\begin{Shaded}
\begin{Highlighting}[]
\CommentTok{\# Deploy senza dipendenze esterne}
\ExtensionTok{./scripts/composeApplication.sh} \AttributeTok{{-}{-}no{-}deps}\NormalTok{ up }\AttributeTok{{-}d} \AttributeTok{{-}{-}wait}

\CommentTok{\# Deploy con seeding ma senza dipendenze esterne}
\ExtensionTok{./scripts/composeApplication.sh} \AttributeTok{{-}{-}init{-}db} \AttributeTok{{-}{-}no{-}deps}\NormalTok{ up }\AttributeTok{{-}d} \AttributeTok{{-}{-}wait}
\end{Highlighting}
\end{Shaded}

\textbf{Nota:} Quando si utilizza \texttt{-\/-no-deps}, le chiavi Stripe \textbf{(STRIPE\_SECRET\_KEY, STRIPE\_PUBLISHABLE\_KEY, STRIPE\_WEBHOOK\_SECRET)} in \texttt{.env} possono contenere valori arbitrari.

\paragraph{Configurazione Stripe}\label{configurazione-stripe}

\textbf{Per i pagamenti Stripe in ambiente locale} (richiesto solo se
NON si utilizza \texttt{-\/-no-deps}):

\begin{Shaded}
\begin{Highlighting}[]
\ExtensionTok{./services/payments/scripts/local{-}webhooks.sh}
\end{Highlighting}
\end{Shaded}

Questo script deve essere eseguito per inoltrare i webhook di Stripe
all'ambiente locale.

Per maggiori informazioni sull'utilizzo della modalità sandbox,
consultare la \href{https://docs.stripe.com/testing}{documentazione
Stripe}.

\subsubsection{Setup alternativo}\label{setup-alternativo}

Utilizzare Gradle per configurare l'intero ambiente con seeding e
listener Stripe:

\begin{Shaded}
\begin{Highlighting}[]
\ExtensionTok{./gradlew}\NormalTok{ setupApplicationEnvironment}
\end{Highlighting}
\end{Shaded}

\subsubsection{Teardown}\label{teardown}

\textbf{Arresto dell'applicazione:}

\begin{Shaded}
\begin{Highlighting}[]
\ExtensionTok{./scripts/composeApplication.sh}\NormalTok{ down}
\end{Highlighting}
\end{Shaded}

\textbf{Arresto e rimozione dei volumi:}

\begin{Shaded}
\begin{Highlighting}[]
\ExtensionTok{./scripts/composeApplication.sh}\NormalTok{ down }\AttributeTok{{-}v}
\end{Highlighting}
\end{Shaded}

\begin{center}\rule{0.5\linewidth}{0.5pt}\end{center}

Come ulteriore alternativa, è possibile visualizzare il sito già in
produzione al link \url{https://eventonight.site/it}


\chapter{Conclusioni}
\input{sections/conclusioni.tex}

\end{document}
